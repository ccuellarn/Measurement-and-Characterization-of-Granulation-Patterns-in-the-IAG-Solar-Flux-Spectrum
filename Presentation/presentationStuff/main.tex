\documentclass[aspectratio=169]{beamer}
\usepackage[style=ieee,backend=bibtex]{biblatex}
\usepackage{amsmath, tgpagella, eulerpx, eucal}
\usepackage{tikz}
\usepackage{graphicx}
\usepackage{xcolor}
\usepackage{ragged2e}
\usepackage{subcaption}
\usepackage{physics}
\usepackage{amssymb}
\usepackage[T1]{fontenc}
\usepackage[font=scriptsize]{caption}

\renewcommand{\footnotesize}{\tiny}

\addbibresource{referenceAll.bib}

\usefonttheme{serif}
\usefonttheme{professionalfonts}

\usetikzlibrary{positioning}

\definecolor{Miku}{HTML}{168282}
\definecolor{Ado}{HTML}{000E4F}  

\newcommand{\MyFrameTitle}{}

\setbeamertemplate{navigation symbols}[default]
\setbeamertemplate{caption}[numbered]

%Headline definition:
\setbeamercolor{headline}{bg=Miku, fg=white}
\setbeamertemplate{headline}{
  \begin{beamercolorbox}[wd = \paperwidth, ht=1cm, dp=0.2cm]{headline}
    \hspace*{0.3cm}\includegraphics[height=0.8cm]{images/Logo_White.png}%
    \hfill
    \raisebox{0.2cm}{\large\textcolor{white}{\MyFrameTitle}}%
    \hspace*{0.3cm}
  \end{beamercolorbox}
}


% Footline definition:
\newcommand{\myauthor}{Cuellar A.}
\newcommand{\myfootertitle}{Medida y caracterización del patrón de granulación en el IAG Solar Flux Atlas}

\setbeamercolor{footline}{bg=Miku, fg=white}
\setbeamertemplate{footline}{
  \leavevmode%
  \hbox{%
    \begin{beamercolorbox}[wd=\paperwidth,ht=2.5ex,dp=1ex,leftskip=1em,rightskip=1em]{footline}
      \usebeamerfont{footline}
      \color{white} 
      \insertframenumber{} / \inserttotalframenumber\hfill
      \myauthor\hfill
      \myfootertitle
    \end{beamercolorbox}%
  }%
  \vskip0pt%
}

\begin{document}

\begin{frame}[plain] % Title section.
  \begin{tikzpicture}[remember picture,overlay]
    \node[anchor=north east, xshift = -0.5cm, yshift = -0.15cm] at (current page.north east) {
      \begin{minipage}[t]{0.45\textwidth}
        \raggedleft
        
        {\Large \textcolor{Ado}{Universidad de los Andes}}\\
        {\small \textcolor{Ado}{Departamento de Física}}
      \end{minipage}
    };
    \node[anchor = south, xshift = -5cm , yshift = 0.10cm] at (current page.south) {
      \begin{minipage}{0.8\textwidth}
        \centering
        \small
        $\ast$\tiny\textcolor{Ado}{c.cuellarn@uniandes.edu.co} \quad 
        $\ast\ast$\tiny\textcolor{Ado}{boostra@uniandes.edu.co}
      \end{minipage}
    };
  \end{tikzpicture}
  \vspace{0.25cm}
  \begin{flushright}
    {\bfseries\LARGE \textcolor{Ado}{Medida y caracterización de los patrones de granulación del espectro solar IAG}}
    \noindent\color{Miku}\rule{\linewidth}{2pt}
    \par\vspace{0.5cm}
      
    \bfseries\textcolor{Ado}{Claudia Alejandra Cuellar Nieto}$^{\ast}$ \\
    \bfseries\textcolor{Ado}{Benjamin Oostra Vannopen}$^{\ast\ast}$
  \end{flushright}
\end{frame}

%-------------------------------- CONTENT
\renewcommand{\MyFrameTitle}{¿Qué es un patrón de granulación?}
\begin{frame}
  \justifying
Es todo aquel comportamiento en el espectro producto del movimiento de convección.
Ya sea asimetrías, Doppler shifts, o \textbf{Cromodependencia anómala}.

Dicho movimiento de convección se ve presente en la granulación.
  % \footnotetext[3]{\fullcite{arfelli_low-dose_1998}}
  % \footnotetext[4]{\fullcite{endrizzi_x-ray_2018}}
\end{frame}

\renewcommand{\MyFrameTitle}{¿Qué es la granulación?}
\begin{frame}
  \justifying
Son celdas convectivas creadas en el movimiento de convección de plasma en la fotosfera solar.
Esto hace que se genere cierto patrón visual en forma de panal, aunque yo prefiero verlo como...
  % \footnotetext[3]{\fullcite{arfelli_low-dose_1998}}
  % \footnotetext[4]{\fullcite{endrizzi_x-ray_2018}}
\end{frame}

\renewcommand{\MyFrameTitle}{La natilla solar}
\begin{frame}
  \justifying
Foto recreada en casa, tranquilos, no es plasma
Se puede entender como natilla, es un "gas" super denso el cual está sometido a un gradiente de temperatura "en escala".

El cual genera que se forme el patrón característico de material enfriándose y calentándose en continuo movimiento.
Este movimiento específico se llama convección donde las burbujas de natilla son celdas convectivas.
  % \footnotetext[3]{\fullcite{arfelli_low-dose_1998}}
  % \footnotetext[4]{\fullcite{endrizzi_x-ray_2018}}
\end{frame}

\renewcommand{\MyFrameTitle}{¿Cuál es el recorrido de la celda convectiva?}
\begin{frame}
  \justifying
Sube caliente, baja fría pero emitiendo radiación
Solo que la relación de tamaños es muy grande 

OJO: Tratamos granulos, no manchas.

  % \footnotetext[3]{\fullcite{arfelli_low-dose_1998}}
  % \footnotetext[4]{\fullcite{endrizzi_x-ray_2018}}
\end{frame}

\renewcommand{\MyFrameTitle}{¿Cómo afecta la convección al espectro?}
\begin{frame}
  \justifying
Si sube caliente emite blueshift
Se enfría, emite redshift
Baja de nuevo con menos blueshift
Pero como cuando sube el brillo es tanto, se ve más blueshift que redshift


Se presentará un bluesfhift convectivo adicional.

  % \footnotetext[3]{\fullcite{arfelli_low-dose_1998}}
  % \footnotetext[4]{\fullcite{endrizzi_x-ray_2018}}
\end{frame}

\renewcommand{\MyFrameTitle}{¿Cómo identificar si su estrella sufre de convección?}
\begin{frame}
  \justifying
Receta para identificar si su estrella presenta convección (by David Gray)
Tambien conocida en el mundo académico como los tres signos de convección:
  \begin{itemize}
    \item Asímetría en la línea (perfil de línea curvo)
    \item Ensanchamiento de línea
    \item Dependencia de la profundidad de línea con el blueshift convectivo
  \end{itemize}

El sol presenta estas tres, y de manera más pronunciada que otras estrellas,
dada su cercanía con la tierra, tenemos mejor resolución para la estrella.
  % \footnotetext[3]{\fullcite{arfelli_low-dose_1998}}
  % \footnotetext[4]{\fullcite{endrizzi_x-ray_2018}}
\end{frame}

\renewcommand{\MyFrameTitle}{¿Cuál fue mi proyecto?}
\begin{frame}
  \justifying
Medir los patrones de granulación basados en los tres signos de convección a partir del espectro
de flujo solar del Intituto de Astrofísica y Geofísica de Göttingen IAG, para caracterizar el fenómeno de
cromodependencia anómala.
  % \footnotetext[3]{\fullcite{arfelli_low-dose_1998}}
  % \footnotetext[4]{\fullcite{endrizzi_x-ray_2018}}
\end{frame}

\renewcommand{\MyFrameTitle}{IAG solar flux atlas}
\begin{frame}
  \justifying
Espectro muy bueno y preciso tomado por Reiners en 2014
  % \footnotetext[3]{\fullcite{arfelli_low-dose_1998}}
  % \footnotetext[4]{\fullcite{endrizzi_x-ray_2018}}
\end{frame}

\renewcommand{\MyFrameTitle}{IAG spatially resolved quiet sun atlas}
\begin{frame}
  \justifying
Espectro muy bueno y preciso tomado por Elwarth en 2023
  % \footnotetext[3]{\fullcite{arfelli_low-dose_1998}}
  % \footnotetext[4]{\fullcite{endrizzi_x-ray_2018}}
\end{frame}

\renewcommand{\MyFrameTitle}{¿Por qué hay dos atlas?}
\begin{frame}
  \justifying
Nuestro objetivo es tomar al sol como cualquier otra estrella para que la caracterización que se haga sea 
aplicable a otras estrellas de las cuales no se tiene tan buena resolución.
PERO para entender dicha caracterización y hallar razones al compartamiento, es necesario usar el de Elwwarth, quien quita el efecto de 
oscurecimiento de borde.
  % \footnotetext[3]{\fullcite{arfelli_low-dose_1998}}
  % \footnotetext[4]{\fullcite{endrizzi_x-ray_2018}}
\end{frame}

\renewcommand{\MyFrameTitle}{¿Qué líneas se usaron para calibrar?}
\begin{frame}
  \justifying
Se utilizó la lista de Nave de Fe I, el cual nos sirve por muchas cosas importantes,
esta lista digamos que fue depurada con líneas que no son mezclas en el sol y que pertenecen al sol.

  % \footnotetext[3]{\fullcite{arfelli_low-dose_1998}}
  % \footnotetext[4]{\fullcite{endrizzi_x-ray_2018}}
\end{frame}

\renewcommand{\MyFrameTitle}{Metodología}
\begin{frame}
  \justifying
Agarro espectro y lineas limpias
Identifico y tomo una ventada de 10mA
Hallo mi longitud de onda observada con un fit de polinomio de grado 4
Calculo cosas
  % \footnotetext[3]{\fullcite{arfelli_low-dose_1998}}
  % \footnotetext[4]{\fullcite{endrizzi_x-ray_2018}}
\end{frame}

\renewcommand{\MyFrameTitle}{¿Por qué 10mA?}
\begin{frame}
  \justifying
Porque lo probamos y es la que genera una ventana óptima de observación, cosa que no se hacia antes.

  % \footnotetext[3]{\fullcite{arfelli_low-dose_1998}}
  % \footnotetext[4]{\fullcite{endrizzi_x-ray_2018}}
\end{frame}

\renewcommand{\MyFrameTitle}{¿Qué se encontró?}
\begin{frame}
  \justifying
Se eoncontró cromodepdendencia hasta en la natilla, hay varias cosas sin razon alguna
Leggamos a una caracterizacion mediocre por el tiempo
NO logre completar casi nada GENIAL :D
  % \footnotetext[3]{\fullcite{arfelli_low-dose_1998}}
  % \footnotetext[4]{\fullcite{endrizzi_x-ray_2018}}
\end{frame}

\renewcommand{\MyFrameTitle}{Conclusiones}
\begin{frame}
  \justifying
Tenemos mejorías en el método pero sin explicaciones a nada
  % \footnotetext[3]{\fullcite{arfelli_low-dose_1998}}
  % \footnotetext[4]{\fullcite{endrizzi_x-ray_2018}}
\end{frame}
\renewcommand{\MyFrameTitle}{¿Preguntas?}
\begin{frame}
  \justifying
NO
  % \footnotetext[3]{\fullcite{arfelli_low-dose_1998}}
  % \footnotetext[4]{\fullcite{endrizzi_x-ray_2018}}
\end{frame}
\renewcommand{\MyFrameTitle}{Bibliography}
\begin{frame}[allowframebreaks]
  \printbibliography
  \nocite{*}
\end{frame}

\end{document}