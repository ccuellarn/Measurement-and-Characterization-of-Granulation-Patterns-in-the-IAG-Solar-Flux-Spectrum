\documentclass[aspectratio=169]{beamer}
\usepackage[style=ieee,backend=bibtex]{biblatex}
\usepackage{amsmath, tgpagella, eulerpx, eucal}
\usepackage{tikz}
\usepackage{graphicx}
\usepackage{xcolor}
\usepackage{ragged2e}
\usepackage{subcaption}
\usepackage{physics}
\usepackage{amssymb}
\usepackage[T1]{fontenc}
\usepackage[font=scriptsize]{caption}

\renewcommand{\footnotesize}{\tiny}

\addbibresource{referenceAll.bib}

\usefonttheme{serif}
\usefonttheme{professionalfonts}

\usetikzlibrary{positioning}

\definecolor{BlueTitle}{HTML}{002855}

\definecolor{MyOrange}{HTML}{FF6B35}    
\definecolor{MyPink}{HTML}{FF4D94}      
\definecolor{MyPurple}{HTML}{8A2BE2}  


\newcommand{\MyFrameTitle}{}

\setbeamertemplate{navigation symbols}[default]
\setbeamertemplate{caption}[numbered]

%Headline definition:
\setbeamercolor{headline}{bg=MyPurple, fg=white}
\setbeamertemplate{headline}{
  \begin{beamercolorbox}[wd = \paperwidth, ht=1cm, dp=0.2cm]{headline}
    \hspace*{0.3cm}\includegraphics[height=0.8cm]{images/HEP_Logo_White.png}%
    \hfill
    \raisebox{0.2cm}{\large\textcolor{white}{\MyFrameTitle}}%
    \hspace*{0.3cm}
  \end{beamercolorbox}
}


% Footline definition:
\newcommand{\myauthor}{Andrade T.}
\newcommand{\myfootertitle}{Implementation of PCI by Edge Illumination for Clinical X-ray Sources}

\setbeamercolor{footline}{bg=BlueTitle, fg=white}
\setbeamertemplate{footline}{
  \leavevmode%
  \hbox{%
    \begin{beamercolorbox}[wd=\paperwidth,ht=2.5ex,dp=1ex,leftskip=1em,rightskip=1em]{footline}
      \usebeamerfont{footline}
      \color{white} 
      \insertframenumber{} / \inserttotalframenumber\hfill
      \myauthor\hfill
      \myfootertitle
    \end{beamercolorbox}%
  }%
  \vskip0pt%
}

\begin{document}

\begin{frame}[plain] % Title section.
  \begin{tikzpicture}[remember picture,overlay]
    \node[anchor=north east, xshift = -0.5cm, yshift = -0.15cm] at (current page.north east) {
      \begin{minipage}[t]{0.45\textwidth}
        \raggedleft
        % \includegraphics[width=2cm]{images/escudo-uniandes-gris.png}\par\vspace{0.3cm}
        {\Large \textcolor{BlueTitle}{Universidad de los Andes}}\\
        {\small \textcolor{BlueTitle}{Department of Physics}}
      \end{minipage}
    };
    \node[anchor = south, xshift = -5cm , yshift = 0.10cm] at (current page.south) {
      \begin{minipage}{0.8\textwidth}
        \centering
        \small
        $\ast$\tiny\textcolor{BlueTitle}{t.andrade@uniandes.edu.co} \quad 
        $\ast\ast$\tiny\textcolor{BlueTitle}{cavila@uniandes.edu.co}
      \end{minipage}
    };
  \end{tikzpicture}
  \vspace{0.25cm}
  \begin{flushright}
    {\bfseries\LARGE \textcolor{BlueTitle}{X-ray Sources}}
    \noindent\color{BlueTitle}\rule{\linewidth}{2pt}
    \par\vspace{0.5cm}
      
    \bfseries\textcolor{BlueTitle}{Thomas Andrade Hernández}$^{\ast}$ \\
    \bfseries\textcolor{BlueTitle}{PhD. Carlos Ávila Bernal}$^{\ast\ast}$
  \end{flushright}
\end{frame}

\renewcommand{\MyFrameTitle}{Research Motivation}
\begin{frame}
  \begin{columns}
    \begin{column}{0.5\textwidth}
      \begin{itemize}
        \item About 12\% of women worldwide are at risk of developing breast cancer\footnotemark[1].
        \item Clinical experience remains the most important factor for a reliable diagnosis.
      \end{itemize}
    \end{column}
    \hfill
    \begin{column}{0.5\textwidth}
        \begin{figure}
        \centering
            \includegraphics[width=1\textwidth]{images/sampleMammogram.png}
            \caption{\justifying Mammography image example with micro-calcifications. Image from [2].}
        \end{figure}
    \end{column}
  \end{columns}
    % Citar al Breast Imaging Physics Part I. y meter una imagen de una mamografía, mostrando una microcalcificación.
  \footnotetext[1]{\fullcite{breast_images}}
  \footnotetext[2]{\fullcite{mammoImage}}
\end{frame}

\renewcommand{\MyFrameTitle}{X-ray Phase Contrast Imaging}
\begin{frame}
  \justifying
  XPCI exploits a property different from the well-known attenuation\footnotemark[3]; 
  this property refers to the \textbf{phase}. To understand it, we introduce the concept of 
  a \textbf{complex refractive index}:
  \begin{equation}
    n = 1 - \delta + i\beta
  \end{equation}
  \justifying
  Here, $\beta$ represents the attenuation effects, while $\delta$ accounts for the phase effects\footnotemark[4]. 
  In soft tissue, this difference is about three orders of magnitude.
  \footnotetext[3]{\fullcite{arfelli_low-dose_1998}}
  \footnotetext[4]{\fullcite{endrizzi_x-ray_2018}}
\end{frame}

\begin{frame}
  \begin{figure}[htbp]
    \centering
    \includegraphics[width=0.7\textwidth]{images/coeffComparison.png}
    \caption{\justifying Complex index of refraction for water and bone as a function of X-ray energy. Image from [4].}
    \label{fig:coeffComparison}
  \end{figure}
  \footnotetext[4]{\fullcite{endrizzi_x-ray_2018}}
\end{frame}

\renewcommand{\MyFrameTitle}{Dark Field}
\begin{frame}
  \begin{columns}
    \begin{column}{0.5\textwidth}
      Dark-field imaging captures ultrasmall-angle X-ray scattering caused by microstructural inhomogeneities 
      smaller than the system resolution. It is particularly sensitive to features such as fibers, pores, or 
      fine tissue structures, producing strong signals in samples like composite materials, bone, and lung.
    \end{column}
    \hfill
    \begin{column}{0.5\textwidth}
        \begin{figure}
        \centering
            \includegraphics[width=0.6\textwidth]{images/darkField.png}
            \caption{\justifying Dark field measurements of an inhomogeneous wedge that gets thicker as moving to the right. 
            Image from [5].}
        \end{figure}
    \end{column}
  \end{columns}
  
  \footnotetext[5]{\fullcite{PhysRevApplied.19.054042}}
  % Edge-Illumination X-Ray Dark-Field Tomography Adam Doherty , 1,* Savvas Savvidis,1 Carlos Navarrete-León ,
\end{frame}

\renewcommand{\MyFrameTitle}{Known Problems of XPCI}
\begin{frame}
  \justifying Most XPCI methods require at least one of the following conditions to be properly implemented\footnotemark[6]:
  \begin{enumerate}
    \item Considerable beam coherence, being achieved by a synchrotron source or considerable big source to
    detector distance. \textbf{(Propagation Based, Grating Based Interferometry)}
    \item Detectors with high resolution. \textbf{(Analizer Based Interferometry)}
    \item Microtube X-ray sources with focal spots of $\approx 10\mu$m.
  \end{enumerate} 
  \footnotetext[6]{\fullcite{zamir_recent_2017}}
\end{frame}

\renewcommand{\MyFrameTitle}{Edge Illumination Method}
\begin{frame}
  \justifying This method is characterized by the presence of \textbf{two masks}: one creates fringes, 
  while the other selectively filters them.  
  \begin{figure}[htbp] 
    \centering
    \includegraphics[width=0.6\textwidth]{images/doubleMask.png}
    \caption{Conventional edge illumination setup. Image from [7].}
    \label{fig:doubleMask}
  \end{figure}
  \footnotetext[7]{\fullcite{olivo_edge-illumination_2021}}
\end{frame}

\renewcommand{\MyFrameTitle}{Single Mask Modality}
\begin{frame}
  Edge illumination can be achieved by using only \textbf{one mask} positioned to illuminate
  a specific region between pixels.
  \begin{figure}[htbp]
    \centering
    \includegraphics[width=0.6\textwidth]{images/singleMask.png}
    \caption{Single mask modality of the edge illumination setup. Image from [8].}
    \label{fig:doubleMask}
  \end{figure}
  \footnotetext[8]{\fullcite{yuan_comparing_nodate}}
\end{frame}

\renewcommand{\MyFrameTitle}{Objectives}
\begin{frame}
Study the feasibility of implementing XPCI with Edge Illumination in the single-mask modality for 
clinical X-ray setups. \\

  \textbf{Specific Objectives:}
  \begin{itemize}
    \item Identify the key setup parameters required for effective Edge Illumination (EI).
    \item Develop an open-source code for phase retrieval using EI.
    \item Perform computational simulations with PEPI to evaluate spatial resolution and contrast for various clinical X-ray setups.
    \item Conduct experimental studies with clinical setups and medical phantoms to validate the computational results.
  \end{itemize}
\end{frame}

\renewcommand{\MyFrameTitle}{Previous work with XPCI by EI}
\begin{frame}
  \begin{columns}
    \begin{column}{0.5\textwidth}
      Astolfo et al.\footnotemark[9] investigated how the focal spot size influences phase retrieval 
      and dark-field measurements. They found that all signal values remained consistent across 
      different effective focal spot sizes, except for scatter, which showed a slight increase with 
      larger focal spots.
    \end{column}
 
    \begin{column}{0.5\textwidth}
        \begin{figure}
        \centering
            \includegraphics[width=0.6\textwidth]{images/astolfoImages.png}
            \caption{Measurements of attenuation, phase and dark field of a phantom. Image from [9]}
        \end{figure}
    \end{column}
  \end{columns}

  \footnotetext[9]{\fullcite{astolfo_effect_2022}}
\end{frame}

\renewcommand{\MyFrameTitle}{Transport of Intensity Model by Yuan \& Das}
\begin{frame}
  Jingchuang Yuan and Mini Das\footnotemark[10] proposed a method for simultaneous \textbf{phase contrast} and \textbf{dark-field} imaging. 
  Their approach is based on the \textbf{Transport of Intensity Equation} (TIE):
  \begin{equation}
    I(z,\vec{r}) = I(0,\vec{r}) - \frac{z}{k} \Big( \nabla_{\perp} I(0,\vec{r}) \cdot \nabla_{\perp} \phi(\vec{r}) + I(0,\vec{r}) \nabla_{\perp}^{2} \phi(\vec{r}) \Big)
  \end{equation}

  In the Propagation-Based formulation, the first term is said to be negligible 
  ($\nabla_{\perp} I(0,\vec{r}) \cdot \nabla_{\perp} \phi(\vec{r}) \approx 0$), so that:
  \begin{equation}
    I(z,\vec{r}) = I(0,\vec{r}) - \frac{z}{k} I(0,\vec{r}) \nabla_{\perp}^{2} \phi(\vec{r})
  \end{equation}

  \footnotetext[10]{\fullcite{yuan_single-shot_2025}}
\end{frame}


\renewcommand{\MyFrameTitle}{Transport of Intensity Model}
\begin{frame}
  For the mask case, the assumption of minimal variation no longer holds, so the first term 
  cannot be neglected. Assuming
  $I(z, \vec{r}) = T(\vec{r}) M(x)$:
  \begin{equation}
    \nabla_{\perp} I(0,\vec{r}) = T \nabla_{\perp} M + M \nabla_{\perp} T \approx T \partial_x M
  \end{equation}

  The TIE then becomes, after integration over the pixel regions:
  \begin{equation}
    I_n = \int_{x_n}^{x_{n+1}} T \cdot M \, dx 
    - \frac{z}{k} \int_{x_n}^{x_{n+1}} T \cdot \partial_x M \cdot \partial_x \phi \, dx 
    - \frac{z}{k} \int_{x_n}^{x_{n+1}} T \cdot M \cdot \nabla_\perp^2 \phi \, dx
  \end{equation}

  Finally, assuming $M(x)$ is a square-wave function with coefficients $C_m$ depending on the mask properties\footnotemark[11]:
  \begin{equation}
    M(x) = \sum_m C_m \cos \left( \frac{2 \pi m x}{2 \rho} \right)
  \end{equation}
  \footnotetext[11]{\fullcite{yuan_transport--intensity_2024}}
\end{frame}


 \renewcommand{\MyFrameTitle}{Phase Acquisition}
\begin{frame}
  \begin{columns}
    \begin{column}{0.4\textwidth}
      The case of only phase retrieval ends up being:
      \begin{equation}
        T_n(1-L_n) \approx \frac{\bar{I}_n+\bar{I}_{n+1}}{2} 
      \end{equation}
      \begin{equation}
        \frac{\alpha}{w_e}D_n \approx \frac{\bar{I}_n-\bar{I}_{n+1}}{\bar{I}_n+\bar{I}_{n+1}}(-1)^n
      \end{equation}
    \end{column}
 
    \begin{column}{0.6\textwidth}
        \begin{figure}
        \centering
            \includegraphics[width=1\textwidth]{images/DPC}
            \caption{Required setup for the phase contrast map acquisition. Image from [10].}
        \end{figure}
    \end{column}
  \end{columns}  
    Here $\bar{I}_{n} = I^{(S)}_{n}/I^{(M)}_{n}$. $I^{(S)}_{n}$ are the intensities of the image with de mask
    and object, while $I^{(M)}_{n}$ the ones with only the mask. 
  \footnotetext[10]{\fullcite{yuan_single-shot_2025}}
\end{frame}

\renewcommand{\MyFrameTitle}{Dark-Field Acquisition}
\begin{frame}
  \begin{columns}
    \begin{column}{0.4\textwidth}
      For dark-field-only retrieval, the equations become:
      \begin{equation}
        T_n(1-L_n) = \frac{I_n^{(S)} + I_{n+1}^{(S)}}{I_n^{(M)} + I_{n+1}^{(M)}}
      \end{equation}
      \begin{equation}
        \frac{\alpha_3}{\alpha_1} S_n = 1 - \frac{I_n^{(S)} - I_{n+1}^{(S)}}{I_n^{(M)} - I_{n+1}^{(M)}} 
        \cdot \frac{I_n^{(M)} + I_{n+1}^{(M)}}{I_n^{(S)} + I_{n+1}^{(S)}}
      \end{equation}
    \end{column}
    
    \begin{column}{0.6\textwidth}
      \begin{figure}
        \centering
        \includegraphics[width=1\textwidth]{images/DF.png}
        \caption{Required setup for the dark-field acquisition. Image from \cite{yuan_single-shot_2025}.}
      \end{figure}
    \end{column}
  \end{columns}
\footnotetext[10]{\fullcite{yuan_single-shot_2025}}
\end{frame}

 \renewcommand{\MyFrameTitle}{Simultaneous Acquisition of Phase and Dark Field}
\begin{frame}
  For simultaneous retrieval, the equations are:

  \begin{equation}
    T_n (1 - L_n) = \frac{I^{(S)}_{n-1} + I^{(S)}_n + I^{(S)}_{n+1}}
    {I^{(M)}_{n-1} + I^{(M)}_n + I^{(M)}_{n+1}}
  \end{equation}

  \begin{equation}
    \frac{2\alpha_2}{w_e + \tfrac{1}{2}\alpha_1} D_n = 
    \left(\frac{I^{(S)}_{n-1}}{I^{(M)}_{n-1}} - \frac{I^{(S)}_n}{I^{(M)}_n}\right)
    \frac{I^{(M)}_{n-1} + I^{(M)}_n + I^{(M)}_{n+1}}
    {I^{(S)}_{n-1} + I^{(S)}_n + I^{(S)}_{n+1}}
  \end{equation}

  \begin{equation}
    \frac{2\alpha_3}{w_e + 2\alpha_1} S_n = 
    \left[ 1 - \frac{I^{(S)}_{n-1} + I^{(S)}_n - I^{(S)}_{n+1}}
    {I^{(M)}_{n-1} + I^{(M)}_n - I^{(M)}_{n+1}}
    \frac{I^{(M)}_{n-1} + I^{(M)}_n + I^{(M)}_{n+1}}
    {I^{(S)}_{n-1} + I^{(S)}_n + I^{(S)}_{n+1}} \right]
  \end{equation}

  \vspace{0.3cm}
  Here, $T_n$, $D_n$, and $S_n$ correspond to the transmission, differential phase, and dark-field signals, respectively.
\end{frame}

\begin{frame}
  \begin{figure}
    \centering
    \includegraphics[width=1\textwidth]{images/BOTH.png}
    \caption{Required setup for the simultaneous acquisition. Image from \cite{yuan_single-shot_2025}.}
  \end{figure}
  \footnotetext[10]{\fullcite{yuan_single-shot_2025}}
\end{frame}

\renewcommand{\MyFrameTitle}{Geant4 Framework}
\begin{frame}{Simulation Framework}
  This project uses the Geant4 framework developed at CERN, which has been adapted for 
  phase-contrast imaging simulations\footnotemark[12]. 
  Additionally, Luca Brombal at the University of Trieste has released an open-access code called 
  Photon-counting Edge-illumination Phase-contrast Imaging (PEPI)\footnotemark[13].
  \footnotetext[12]{\fullcite{brombal_geant4_2022}}
  \footnotetext[13]{\fullcite{brombal_pepi_2023}}
\end{frame}

\begin{frame}
  \begin{figure}
    \centering
    \includegraphics[width=0.4\textwidth]{images/ATLAS.png}
    \caption{Image of the ATLAS detector at CERN simulated using Geant4. Image taken from the ATLAS official website.}
  \end{figure}
\end{frame}

\renewcommand{\MyFrameTitle}{Considered Setup and Restrictions}
\begin{frame}
  \begin{figure}[htbp]
    \centering
    \includegraphics[width=0.6\textwidth]{images/singleMask.png}
    \caption{Single mask modality of the edge illumination setup. Image from [8].}
    \label{fig:doubleMask}
  \end{figure}
  \footnotetext[8]{\fullcite{yuan_comparing_nodate}}
\end{frame}

\begin{frame}{Simulation Parameters}
  The following parameters were considered throughout the simulations:
  \begin{enumerate}
    \item Source-to-detector distance: 70 cm.
    \item Mask-to-detector distances: 50 cm, 46 cm, 42 cm, ..., 18 cm, 14 cm, 10 cm.
    \item Focal spot sizes: 10 $\mu$m, 100 $\mu$m, 200 $\mu$m, and 300 $\mu$m.
  \end{enumerate}

  The chosen configuration follows the study by Fico et al.\footnotemark[1], which describes 
  the characteristics of the mammography setup.

  \footnotetext[1]{\fullcite{breast_images}}
\end{frame}

\renewcommand{\MyFrameTitle}{Mask Characterization}
\begin{frame}
  \begin{itemize}
    \item For each distance, a simulation with $1 \times 10^{6}$ events was performed. 
    The X-ray beam was projected onto a detector of $1024 \times 1024$ pixels, 
    each with a size of 10$\mu$m.
  \end{itemize}
\end{frame}

\begin{frame}
  \begin{figure}
    \centering
    \begin{subfigure}{0.4\textwidth}
      \includegraphics[width=\textwidth]{images/Step1.png}
    \end{subfigure}
    \hfill
    \begin{subfigure}{0.4\textwidth}
      \includegraphics[width=\textwidth]{images/Step1_.png}
    \end{subfigure}
    \caption{From left to right: measurements with 10 $\mu$m and 100 $\mu$m focal spots. Same measurement setup: mask-to-detector distance 46 cm, mask aperture 90 $\mu$m.}
  \end{figure}
\end{frame}

\begin{frame}
  \begin{itemize}
    \item The intensity was projected along the $x$-axis, and only measurements exceeding 
    a threshold of 5\% of the maximum intensity were considered.
  \end{itemize}
\end{frame}

\begin{frame}
  \begin{figure}
    \centering
    \begin{subfigure}{0.4\textwidth}
      \includegraphics[width=\textwidth]{images/Step2.png}
    \end{subfigure}
    \hfill
    \begin{subfigure}{0.4\textwidth}
      \includegraphics[width=\textwidth]{images/Step2_.png}
    \end{subfigure}
    \caption{From left to right: measurements with 10 $\mu$m and 100 $\mu$m focal spots. Same measurement setup: mask-to-detector distance 46 cm, mask aperture 90 $\mu$m.}
  \end{figure}
\end{frame}

\begin{frame}
  \begin{figure}
    \centering
    \begin{subfigure}{0.4\textwidth}
      \includegraphics[width=\textwidth]{images/Step3.png}
    \end{subfigure}
    \hfill
    \begin{subfigure}{0.4\textwidth}
      \includegraphics[width=\textwidth]{images/Step3_.png}
    \end{subfigure}
    \caption{From left to right: measurements with 10 $\mu$m and 100 $\mu$m focal spots. Same measurement setup: mask-to-detector distance 46 cm, mask aperture 90 $\mu$m.}
  \end{figure}
\end{frame}

\begin{frame}
  \begin{itemize}
    \item The mask aperture size was then varied to study how the beams are magnified, 
      repeating the procedure for multiple distances.
  \end{itemize}
\end{frame}

\begin{frame}
  \begin{figure}
    \centering
    \begin{subfigure}{0.45\textwidth}
      \includegraphics[width=\textwidth]{images/results/10umMult.pdf}
    \end{subfigure}
    \hfill
    \begin{subfigure}{0.45\textwidth}
      \includegraphics[width=\textwidth]{images/results/100umMult.pdf}
    \end{subfigure}
    \caption{From left to right: measurements with 10 $\mu$m and 100 $\mu$m focal spots. Same measurement setup: mask-to-detector distance 46 cm, mask aperture 90 $\mu$m.}
  \end{figure}
\end{frame}

\begin{frame}
  These magnifications were measured and show a trend that can be explained 
  using the geometric magnification formula of a cone-beam source\footnotemark[14]:

  \begin{align*}
    p_{\text{detector}} = M p_{\text{mask}} & = \frac{d_{SD}}{d_{SM}} p_{\text{mask}} 
    = \frac{d_{SD}}{d_{SD} - d_{MD}} p_{\text{mask}} \\
    \frac{p_{\text{detector}}}{p_{\text{mask}}} & = \frac{1}{1 - \frac{d_{MD}}{d_{SD}}} 
    = M(d_{MD}, d_{SD})
  \end{align*}

  For a non-ideal scenario, the formula can be generalized as:
  \begin{equation}
    M(d_{MD}, d_{SD}, FS, E) = \frac{1}{1 - \frac{d_{MD}}{\alpha(d_{SD}, FS, E)}}
  \end{equation}
  \footnotetext[14]{\fullcite{paganin_x-ray_2021}}
\end{frame}

\begin{frame}
  The $M(d_{MD}, d_{SD}, FS, E)$ can be visualized as follows:
  \begin{equation}
    \delta = \frac{M + 1}{2}p_{\text{mask}}
  \end{equation}
    \begin{figure}[htbp]
    \centering
    \includegraphics[width=0.6\textwidth]{images/Esquema.jpg}
    \caption{Sketch of what I mean.}
    \label{fig:doubleMask}
  \end{figure}
\end{frame}

\renewcommand{\MyFrameTitle}{Mask Characterization Results}
\begin{frame}
  \begin{figure}[htbp]
    \centering
    \includegraphics[width=0.6\textwidth]{images/Temporal.png}
    \caption{Magnification of the beamlet as a function of the mask-detector distance for several focal spots.}
    \label{fig:doubleMask}
  \end{figure}
  \footnotetext[8]{\fullcite{yuan_comparing_nodate}}
\end{frame}

\renewcommand{\MyFrameTitle}{Yuan \& Das Retreieval Method Results}
\begin{frame}
  \begin{figure}[htbp]
    \centering
    \includegraphics[width=1\textwidth]{images/PhaseIdealTrial.png}
    \caption{Ideal scenario only phase retrieval.}
    \label{fig:doubleMask}
  \end{figure}
\end{frame}

\begin{frame}
  \begin{figure}[htbp]
    \centering
    \includegraphics[width=1\textwidth]{images/Heya.png}
    \caption{Non ideal scenario only dark field.}
    \label{fig:doubleMask}
  \end{figure}
\end{frame}

\renewcommand{\MyFrameTitle}{Overall Progress}
\begin{frame}
  The overall progress of the project has been:
  \begin{itemize}
    \item Identify the key setup parameters required for effective Edge Illumination (EI) (70\%).
    \item Develop an open-source code for phase retrieval using EI (100\%).
    \item Perform computational simulations with PEPI to evaluate spatial resolution and contrast for various clinical X-ray setups (20\%).
    \item Conduct experimental studies with clinical setups and medical phantoms to validate the computational results (0\%).
  \end{itemize}
\end{frame}

\renewcommand{\MyFrameTitle}{Remaining Tasks}
\begin{frame}
  \begin{itemize}
    \item Need to closely observe the behavior for 200 $\mu$m and 300 $\mu$m focal spots, which appear unusual. \\
    \item Once these cases are analyzed, proceed to examine other scenarios for large focal spots (Yuan \& Das). \\
    \item Assess how the CNR and SNR change for the same setup with different focal spot sizes.
  \end{itemize}
\end{frame}

\renewcommand{\MyFrameTitle}{Potential Questions}
\begin{frame}
  \begin{itemize}
    \item What is the equivalence between the simulated events and the radiation dose received by a person? (Try to estimate.)
    \item What happens if the X-ray source energy is increased?
    \item What results do you expect based on the current simulations?
  \end{itemize}
\end{frame}
\renewcommand{\MyFrameTitle}{Bibliography}
\begin{frame}[allowframebreaks]
  \printbibliography
  \nocite{*}
\end{frame}

\end{document}