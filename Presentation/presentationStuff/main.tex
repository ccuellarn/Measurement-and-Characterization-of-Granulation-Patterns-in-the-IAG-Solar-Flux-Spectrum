\documentclass[aspectratio=169]{beamer}
\usepackage[style=ieee,backend=bibtex]{biblatex}
\usepackage{amsmath, tgpagella, eulerpx, eucal}
\usepackage{tikz}
\usepackage{graphicx}
\usepackage{xcolor}
\usepackage{ragged2e}
\usepackage{subcaption}
\usepackage{physics}
\usepackage{amssymb}
\usepackage[T1]{fontenc}
\usepackage[font=scriptsize]{caption}

\renewcommand{\footnotesize}{\tiny}

\addbibresource{referenceAll.bib}

\usefonttheme{serif}
\usefonttheme{professionalfonts}

\usetikzlibrary{positioning}

\definecolor{Miku}{HTML}{168282}
\definecolor{Ado}{HTML}{000E4F}  

\newcommand{\MyFrameTitle}{}

\setbeamertemplate{navigation symbols}[default]
\setbeamertemplate{caption}[numbered]

%Headline definition:
\setbeamercolor{headline}{bg=Miku, fg=white}
\setbeamertemplate{headline}{
  \begin{beamercolorbox}[wd = \paperwidth, ht=1cm, dp=0.2cm]{headline}
    \hspace*{0.3cm}\includegraphics[height=0.8cm]{images/Logo_White.png}%
    \hfill
    \raisebox{0.2cm}{\large\textcolor{white}{\MyFrameTitle}}%
    \hspace*{0.3cm}
  \end{beamercolorbox}
}


% Footline definition:
\newcommand{\myauthor}{Cuellar A.}
\newcommand{\myfootertitle}{Medida y caracterización del patrón de granulación en el IAG Solar Flux Atlas}

\setbeamercolor{footline}{bg=Miku, fg=white}
\setbeamertemplate{footline}{
  \leavevmode%
  \hbox{%
    \begin{beamercolorbox}[wd=\paperwidth,ht=2.5ex,dp=1ex,leftskip=1em,rightskip=1em]{footline}
      \usebeamerfont{footline}
      \color{white} 
      \insertframenumber{} / \inserttotalframenumber\hfill
      \myauthor\hfill
      \myfootertitle
    \end{beamercolorbox}%
  }%
  \vskip0pt%
}

\begin{document}

\begin{frame}[plain] % Title section.
  \begin{tikzpicture}[remember picture,overlay]
    \node[anchor=north east, xshift = -0.5cm, yshift = -0.15cm] at (current page.north east) {
      \begin{minipage}[t]{0.45\textwidth}
        \raggedleft
        
        {\Large \textcolor{Ado}{Universidad de los Andes}}\\
        {\small \textcolor{Ado}{Departamento de Física}}
      \end{minipage}
    };
    \node[anchor = south, xshift = -5cm , yshift = 0.10cm] at (current page.south) {
      \begin{minipage}{0.8\textwidth}
        \centering
        \small
        $\ast$\tiny\textcolor{Ado}{c.cuellarn@uniandes.edu.co} \quad 
        $\ast\ast$\tiny\textcolor{Ado}{boostra@uniandes.edu.co}
      \end{minipage}
    };
  \end{tikzpicture}
  \vspace{0.25cm}
  \begin{flushright}
    {\bfseries\LARGE \textcolor{Ado}{Medida y caracterización del patrón de granulación en el IAG Solar Flux Atlas}}
    \noindent\color{Miku}\rule{\linewidth}{2pt}
    \par\vspace{0.5cm}
      
    \bfseries\textcolor{Ado}{Claudia Alejandra Cuellar Nieto}$^{\ast}$ \\
    \bfseries\textcolor{Ado}{Benjamin Oostra Vannopen}$^{\ast\ast}$
  \end{flushright}
\end{frame}


\renewcommand{\MyFrameTitle}{X-ray Phase Contrast Imaging}
\begin{frame}
  \justifying
  XPCI exploits a property different from the well-known attenuation\footnotemark[3]; 
  this property refers to the \textbf{phase}. To understand it, we introduce the concept of 
  a \textbf{complex refractive index}:
  \begin{equation}
    n = 1 - \delta + i\beta
  \end{equation}
  \justifying
  Here, $\beta$ represents the attenuation effects, while $\delta$ accounts for the phase effects\footnotemark[4]. 
  In soft tissue, this difference is about three orders of magnitude.
  % \footnotetext[3]{\fullcite{arfelli_low-dose_1998}}
  % \footnotetext[4]{\fullcite{endrizzi_x-ray_2018}}
\end{frame}




\renewcommand{\MyFrameTitle}{Overall Progress}
\begin{frame}
  The overall progress of the project has been:
  \begin{itemize}
    \item Identify the key setup parameters required for effective Edge Illumination (EI) (70\%).
    \item Develop an open-source code for phase retrieval using EI (100\%).
    \item Perform computational simulations with PEPI to evaluate spatial resolution and contrast for various clinical X-ray setups (20\%).
    \item Conduct experimental studies with clinical setups and medical phantoms to validate the computational results (0\%).
  \end{itemize}
\end{frame}


\renewcommand{\MyFrameTitle}{Bibliography}
\begin{frame}[allowframebreaks]
  \printbibliography
  \nocite{*}
\end{frame}

\end{document}