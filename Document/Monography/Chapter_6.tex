\begin{appendices}
\chapter{Z-score Standardization}\label{ap:zscore}
In the process of calculate the four order polynomial fit the function np.poly.fit() presents an overstimation on the coefficients, due to the large difference of magnitude order between axis.
To deal with this difference a z-score standardization was used on the selected bins of wavelengths around the observed wavelength. 
This process helps to avoid the dominance of certain features over other due to diferences in their scales \cite{Boyd_2014}.

The follow up for the standardization was applied the relation \eqref{z score} on the selected bins for wavelength.

\begin{equation}
\lambda_{scaled}= \frac{\lambda-\mu(\lambda)}{\sigma(\lambda)}
\label{z score}
\end{equation}

Where $\mu(\lambda)$ refers to the mean and $\sigma(\lambda)$ to the standard deviation of the wavelength range. 
As the wavelength was scaled, in terms of calculated derivates for the first and the second signature, a re-scaled for this values was necessary.
Based on the definition for the standardization, the derivates follow the relation \eqref{re scaled derivates}.

\begin{equation}
 \frac{d}{d \lambda} = \frac{1}{\sigma(\lambda)} \frac{d}{d \lambda_{scaled}}
\label{re scaled derivates}
\end{equation}

Taking the derivate of the expresion \eqref{z score} a factor related to the standard deviation appear.
With this, the original values for derivates evaluated in the observed wavelength are expresed in equation \eqref{2 and 3 scaled derivate}

\begin{equation}
 \frac{d^2}{d \lambda^2} = \frac{1}{\sigma(\lambda)^2} \frac{d^2}{d \lambda_{scaled}^2} \quad \quad \frac{d^3}{d \lambda^3} = \frac{1}{\sigma(\lambda)^3} \frac{d^3}{d \lambda_{scaled}^3}
\label{2 and 3 scaled derivate}
\end{equation}

This improved considerably the precision in the fit and there over the precision on the observed wavelength calculated.

\chapter{The third derivate relation}\label{ap:third derivate}

Or called the bisector slope. 
It was multiplied by the relation ($\frac{c}{\lambda}$) to see each clear in the graphic.

\chapter{Visualizer for outliers}\label{ap:visualizer}

For the process of the blend-free list was created an app using the interface Tkinter with the objective for help the visualization of outliers.
Two version of the visualizer were created.
One just show the line core and forthh order polynomial fit as shown in the figure ().

This helps for a first process where far separated lines were discarted.
Then, we can perform the different calculations and use the second version of the visualizer (see figure ()).

In this version the vision of the thre signature of conevction and the line core with the fit.
Moreover, was resalted the correspond Fe I line on each graph to corroborated the behavior.
Thanks to this the time expended seeing lines was reduced signficantly.


\end{appendices}
