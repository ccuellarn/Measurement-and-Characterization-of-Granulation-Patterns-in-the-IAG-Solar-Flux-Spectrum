As the spectral analysis is complicated due to  convective motion this project takes a computational approach leading to an observational requirement, by the experience of professor Benjamin and previous research.
For the computational part, the programming language Python and Jupyter Notebook was used.
All analysis data and code used in the process was uploaded to a GitHub repository \cite{GitHub}, allowing anyone interested to reproduce the results and verify the authenticity of the conclusions presented. 
Moreover, proper credit will be given to all previous work from other researchers.

We follow the methodology established in previous studies (see \cite{Dravins_1981} and references therein), which utilized a selected list of Fe I lines. 
As mentioned before these lines are ideal for this calibration due to their minimal thermal broadening and reduced susceptibility to other atmospheric affectations.

\section{Computational approach}
The computational aspect focuses on identifying the granulation pattern within the solar spectrum by calculating relative velocities using the wavelengths of Fe I. 

The line up was to identify the Fe I lines in the IAG Solar Flux Atlas and the IAG Spatially Resolved Quiet Sun Atlas using the blend-free list of Fe I lines.
Second, fit a fourth-grade polynomial fit due to the c-curved line profile bisector and find the observed wavelength. For an optimal fit was used a z-score standardization on each line core, which is explained in appendix A.
Then calculate the observed wavelength, Doppler velocity and convective blueshift based on the polynomial fit.
Finally, find the values for the line core curvature (see equation \eqref{core curvature}) and the line core bisector slope (see equation \eqref{bisector slope}).

\section{Blend-free Fe I line list}
The previous line up implemented the Nave et. al list of laboratory measured Fe I lines \cite{Nave_1994}. 
This list classifies lines with a quality rating (A,B,C,D) with A be the most and best quality.
However, not all the listed lines are clearly present in the solar spectrum, and within the near-infrared range, many lines are severely mixed.

In collaboration with Professor Benjamin and Manuel Fuentes, we refined this list using different approaches. 

\subsection{Selection method for Fe I lines}
The process of selection has two different approaches: Computational and visual.
First, we selected only quality A lines for the Nave list and then performed statistical filters according to the statement of having a fourth order polynomial fit for the line core.

The filters consist of three general aspects.
First, the coefficient of the fourth grade term needs to be positive and not too small because it denotes a wide curve. Specially, this can be for a line in the infrared due the big core curvatures are related to atmospheric lines.
Second, the difference between extreme points of the fit needs to be less than half of the distance. This filter helps identify lines which are not slopes and have a core.
Finally, the absolute difference between the wavelength observed and the emitted (from Nave list) needs to be less to $0.025\mathring{A}$.

\subsection{Observational requirement}
The last part of the selection was a visual inspection to discard line mixes or absent from the solar spectrum. The use of a visualizer application (explained in detail in the appendix C) was useful to see simultaneously the graphic for the line core and the behavior in the line core curvature, line bisector slope and granulation pattern plots.
The discard part was using the geometry of the curve, behavior on plots and the guidance of professor Benjamin.

The final part of the methodology was to perform different plots for our analysis. This is based on the requirement for the characterization.
