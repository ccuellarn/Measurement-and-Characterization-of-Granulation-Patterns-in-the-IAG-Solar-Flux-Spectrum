Given the complexity introduced by convective motions on the spectrum, this project takes a computational approach leading, guided by Professor Benjamin and previous research.
The analysis was conducted using the Python programming language.
All data and code have been uploaded in a \href{https://github.com/ccuellarn/Final-Project}{GitHub repository}, allowing anyone interested to reproduce the results and verify the authenticity of the conclusions presented. 

This project follows the methodology established in previous studies (see~\cite{Dravins_1981} and references therein), utilizing a selected list of Fe I lines from Nave~\cite{Nave_1994}. 
As previously discussed, these lines are ideal for the calibration process due to their high abundance, minimal thermal broadening, limited isotopic variation and availability of accurate laboratory measurements of natural wavelengths.
\section{Computational approach}
The computational approach focuses on identifying the solar granulation patterns by calculating relative velocities from Fe I lines. 

First, Fe I lines in the IAG solar flux atlas and the IAG spatially resolved quiet Sun atlas using the Nave blend-free list of Fe I lines were identified~\cite{Nave_1994}.
For each identified line were selected bins of $\SI{0.1}{\angstrom}$ around the closest minimal point to the associated value from the Nave Fe I list.
Beware, these points are not the observed wavelengths, just a reference to generate an observation window.

Then, a fourth-order polynomial fit was adjusted due to the wavelength window and found the minimal point, whose represent the observed wavelength. 
The efficacy of this approach for quantifying asymmetry via the bisector slope is supported by Allende et al. research~\cite{Allende_Garcia_1998}.
For an optimal fit was used a z-score standardization on each line core, which is explained in appendix\ref{ap:zscore}.

The observed wavelength from the polynomial fit was used to calculate the relative velocity.
Finally, the values for the line core curvature (see Equation~\eqref{eq:core curvature}) and the line core bisector slope (see Equation~\eqref{eq:third derivative relation}) were found.

To maintain consistency when comparing results with other authors, the visible and near infrared range given by Reiners et al.~was modified.
The visible range is considered from $\SI{4000}{\angstrom}$ to $\SI{7500}{\angstrom}$; and the near infrared from $\SI{7500}{\angstrom}$ to $\SI{23000}{\angstrom}$.

\subsection{Statistical analysis}
The selection for the width of the window around the line core for analysis is not statistical justified on the previous research.
This project presents several results for the statistical methodology accuracy.  

The variance and standard deviation of the observed wavelength was analyzed by altering the number of points used in the fourth-order polynomial fit aorund the line core. 
Figure~\ref{fig:variance} shows that these parameters set performs well in the visible spectrum. 
However, its performance degrades in the infrared, where the number of data points defining the line core is reduced compared to a typical line in the visible range.

\begin{figure}[H]
     \centering
     \begin{subfigure}{1.0\textwidth}
         \includegraphics[width=\textwidth]{Images/Results/variance VIS.pdf}
         \caption{Visible range for the IAG solar flux atlas.}

     \end{subfigure}
\hfill
     \begin{subfigure}{1.0\textwidth}

         \includegraphics[width=\textwidth]{Images/Results/variance NIR.pdf}
         \caption{Near infrared range for the IAG solar flux atlas.}

     \end{subfigure}

        \caption{Variance and standard deviation for several observed wavelength altering the number of points on the fourth-order polynomial fit.}\label{fig:variance}
\end{figure}

Then, a plot of variance and standard deviation across wavelength for the number of points taken was realized (see Figure~\ref{fig:variance for window}). 
\begin{figure}[H]
     \centering
     \begin{subfigure}{1.0\textwidth}
         \includegraphics[width=\textwidth]{Images/Results/variance VIS.pdf}
         \caption{Standard deviation against wavelength for the number of points taken on the fourth-order polynomial fit.}

     \end{subfigure}
\hfill
     \begin{subfigure}{1.0\textwidth}

         \includegraphics[width=\textwidth]{Images/Results/variance NIR.pdf}
         \caption{Variance against wavelength for the number of points taken on the fourth-order polynomial fit.}

     \end{subfigure}

        \caption{Variance and standard deviation across wavelength for the number of points, the resalted dashed line indicates the window used on the project.}\label{fig:variance for window}
\end{figure}
The resalted dashed line in the Figure~\ref{fig:variance for window} indicates the window used on the project.
It can be seen that is a constant minimal point along all the wavelength range, this aspect is important due the consistency for accuracy on all the spectrum.
The window of $\SI{0.1}{\angstrom}$ around the line core ensures the minimal standard deviation and accuracy for all the spectrum, thereby a consistency on the error for wavelength values.

\section{Blend-free Fe I line Nave list}
The mentioned methodology implements the Nave list of laboratory-measured Fe I lines~\cite{Nave_1994}. 
This list classifies lines with a quality grading ($A,B,C,D$), based on four wavenumber uncertainties $(\Delta \sigma)$ and the corresponding wavelength uncertainties $(\Delta \lambda)$.
Where $A$ denotes lines with error in wavenumbers less than $0.005$ $\text{cm}^{-1}$; grade $B$ less than $0.01$ $\text{cm}^{-1}$; grade $C$ less than $0.02$ $\text{cm}^{-1}$; and grade $D$ greater than $0.02$ $\text{cm}^{-1}$.
All known blended lines and all lines measured only in grating spectra have been assigned the grade $D$~\cite{Nave_1994}.

However, not all the cataloged lines with grade $A$ are clearly present in the solar spectrum, and within the near infrared range, many lines are severely blended.
In collaboration with Professor Benjamin and Manuel Fuentes, we refined this initial list using different approaches to select the most reliable values. 

\subsection{Selection process for blend-free Fe I lines}
The line selection process employed two approaches: Computational and manual.
Initially, only quality A lines for the Nave list were selected. 
This group were subjected to four computational filters based on the properties of having a fourth-order polynomial fit for the line core.

The first filter takes into account the form of the C-curved line profile bisector.
Lines whose c-curved bisectors exhibited excessive scatter, indicating a profile dominated by noise rather than a convective signature, were rejected.

The second filter selected lines whose curvature sign was consistent with an absorption line; this implies the coefficient of the fourth-order term needs to be positive. 

The third filter ensures the selected lines represents an absorption curve by discarding closest points that more closely resemble slopes or continuum noise.
The condition was discard lines which difference between extreme points of the fit were less than half of the line depth. 
This describe an absorption curve, and not a slope or noise.

The final filter required the absolute difference between the observed wavelength and the measured laboratory wavelength to be less than $\SI{0.025}{\angstrom}$. 
Larger discrepancies suggest misidentification or severe blending, rendering the calculated relative velocity unreliable.
This threshold is empirically supported by observations throughout the selection process and the statement for the mean velocity of the sun varies from $200$  m/s to $600$  m/s.

The final part of the selection process was a manual inspection to discard lines that were blended or absent from the solar spectrum. 
A custom visualizer (detailed in appendix\ref{ap:visualizer}) was instrumental for this process, allowing the simultaneously display of graphics related to dynamics aspects. 
This was particularly useful for selection lines in the near infrared range.
The visual requirement focused on the geometry of the spectral line profile, its behavior on the three signatures plots. 

The concluding part of the methodology was performed the plots which represents the three signatures of convection.

First of all, the third signature plot, line core curvatures and line profile bisector slopes against line depth was performed to all ranges in both atlases with the objective to observe the three signatures of convection.
Subsequently, various parameters were plotted against line depth to specifically characterize the granulation.
