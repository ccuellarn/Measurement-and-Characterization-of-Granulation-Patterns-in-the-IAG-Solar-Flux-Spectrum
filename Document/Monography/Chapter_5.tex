This project has successfully characterized the convective blueshift in solar absorption lines and the anomalous chromodependence present across the three established signatures of solar granulation: line broadening, line asymmetry, and the line-depth dependence of convective blueshift. A foundational outcome of this work is the production of a refined, blend-free list of Fe I lines, from which solar granulation patterns were derived with minimal scatter. We recommend this list, complete with measured parameters such as convective blueshift, curvature, bisector slope, and line depth for each line, be published to serve as a community resource. Furthermore, our analysis statistically justified a window of $\SI{1}{\miliangstrom}$ around the line core as the optimal width for examination, a finding that underscores the accuracy and unprecedented precision of the IAG solar flux atlas.

A major contribution of this study is the detailed description of how granulation patterns vary with wavelength, leading to the establishment of color-dependent standard granulation curves for the entire spectrum. We confirmed the expected trend where deeper lines are generally located at shorter wavelengths. More significantly, we identified and characterized several anomalous instances of chromodependence: for a given line depth, short-wavelength lines exhibit greater convective blueshifts, higher (more positive or less negative) core bisector slopes, and narrower line profiles. A key finding is that solar rotation is not the cause of this anomalous chromodependence observed in the third signature, as the velocity shift was present and even greater in the disk-center spectrum where rotational broadening is negligible. For the near-infrared range, the chromodependence is well-described by a general trend line, with the line bisector slope shifting linearly with depth. In the more complex visible range, we identified individual granulation curves and described their governing coefficients.

Delving into specific signatures, the analysis of the line core curvature for shallow lines revealed a clear linear trend in the visible range. The slope of this relationship was found to be $(1.83 \pm 0.08)\times 10^{10}$ $\text{m}^{-1}$ for the disk-integrated flux and a significantly higher $(2.94\pm 0.16)\times 10^{10}$ $\text{m}^{-1}$ for the disk-center spectrum. The absence of wavelength dependence in this relationship indicates that line core curvatures are primarily governed by velocities (thermal, convective, and rotational), with negligible influence from atomic effects. The steeper slope at the disk-center confirms that rotation is a major contributor to line broadening in the integrated spectrum. By applying the theoretical model from Equation~\eqref{eq:Theory slope}, we derived values for $\langle v_{\text{conv}}^2 \rangle$ of $3.15$ $(\text{km/s})^2$ for the disk-integrated flux and $2.20$ $(\text{km/s})^2$ for the disk-center spectrum. The fact that the convective velocity appears lower at the disk center, where it should be most fully observable, points to an unresolved effect that merits further investigation.

Regarding line asymmetry, we focused on the middle range of line depths ($0.3-0.6$), where the convection effect is less pronounced. A clear wavelength dependence was observed in the line core bisector slopes for the visible range, in contrast to the poor dependence and significant scatter found in the near-infrared. For a fixed line depth, the bisector slopes in the integrated flux spectrum decrease with wavelength. While the same general behavior was found in the disk-center spectrum, the magnitude of the displacement in bisector slope values was consistently smaller. This difference highlights the role of limb darkening in producing non-uniform shifts in the integrated spectrum. The analysis of wavelength shifts at fixed velocities revealed that for the disk-integrated spectrum, the shift is not uniform but fluctuates, while the disk-center spectrum exhibits a uniform, quadratically increasing behavior.

Finally, the investigation into the line depth-dependence on excitation potential revealed that the phenomenon is most pronounced for high-excitation potentials ($2.5$ to $5.0$ eV) and is best modeled in the visible range due to non-uniform behavior in the near-infrared. The resulting coefficients depend strongly on wavelength and weakly on excitation potential. A notable finding is the difference in the velocity of maximum displacement: it occurs at $-200$ m/s for the disk-integrated flux but shifts to $0$ m/s for the spatially resolved quiet-Sun atlas.

For future work, it is imperative to study atomic broadening effects and develop more sophisticated models of convection to investigate the roles of temperature and density fluctuations in driving the anomalous chromodependencies uncovered in this study. The standard granulation curves and the comprehensive line list provided here offer a robust foundation for such endeavors.

% %Por cada objetivo va una conclusion

% In conclusion, this project characterized the convective blueshift in solar absorption lines and the anomalous chromodependence present across the three signatures of convection. 
% First of all, we produced a refined, blend-free list of Fe I lines and derived solar granulation patterns with minimal scatter. 
% This list should be published, with the convective blueshift, curvture, bisector slope and dept of each line.
% Alsoinclude other parameters such as ionization potential, probability of transition, and damping parameters.
% The analysis statistically justified a window of $\SI{1}{\miliangstrom}$ around the line core as the optimal width for examination, underscoring the accuracy and unprecedented precision of the IAG solar flux atlas. 

% Furthermore, we provided a detailed description of how granulation patterns vary with wavelength, establishing color-dependent standard granulation curves for the entire spectrum. 
% We found several manifestations of wavelength dependence: The expected fact that, in general, deeper lines are located at shorter wavelengths.
% But also several anomalous instances of chromodependence: For a given line depth, short-wavelength-lines have greater convective blueshifts, higher (more positive or less negative) core bisector slopes, and norrower profiles (or more precisely, sharper, cores).
% Our results determine that rotation is not the cause of the anomalous chromodependence observed in the third signature plot, and we established a general characterization for the full wavelength range of the atlas. 
% For the near-infrared, a general trend line describes the chromodependence, while the line bisector slope shift along the line depth is linear. 
% In the visible range, we identified individual granulation curves and a description for the coefficients. 


% For future work, it is necessary to study atomic broadening effects and model convection to investigate the role of temperature and density fluctuations.

% \subsection{Line depth-dependence of line core curvature}
% Furthermore, the visible range of the IAG solar flux atlas exhibits a clear linear trend for shallow lines.
% the slope of the relationship has a value of $(1.83 \pm 0.08)\times 10^{10}$ $\text{m}^{-1}$.
% The slope for the spatially shows a value of $(2.94\pm 0.16)\times 10^{10}$ $\text{m}^{-1}$ for the slope.
% The absence of wavelength-dependence in this spectral range indicates that line core curvatures have a net dependence on velocities (thermal, convective and rotational), with negligible influence from atomic effects.
% The result of finding a greater slope in the center-disk confirms that rotation is an important cause for line broadening.
% In the IAG spatially resolved quiet sun spectrum the curvatures are greater, meaning that the broadening is smaller.
% This allows deduce the variance of convection speed, knowing the thermal velocity. 
% As mention before, we can induce from here the $\langle v_{\text{conv}}^2 \rangle$ as we the other parameters described in Equation~\eqref{eq:Theory slope} and confirm that rotation is an important cause. 
% Using the theorical values reported and the value from the linear fit applied on shallow lines for both atlases, we found values for $\langle v_{\text{conv}}^2 \rangle$ of $3.15$ $(\text{km/s})^2$ for the disk-integrated flux spectrum and $2.20$ $(\text{km/s})^2$ for the disk-center spectrum.
% This is not according to the theoretical behavior, cause in the center-disk spectrum the rotation is neligible and the convection can be seen in his totality.
% The reason is unknown but make this effect important

% \subsection{The line core bisector slope}
% With the goal of describing the middle range values for line depth which shows a non pronounced convection effect, we conducted a separate analysis on this subset.
% Then, a linear fit was applied to the line core bisector slope data in the line depth range of $(0.3-0.6)$ to quantify this transition, as shown in Figure~\ref{fig:bisector slope}.
% In Figure~\ref{fig:bisector slope} both plots shows an evident wavelength-dependence along the line core bisector slopes.
% Figure~\ref{fig:bisector slope VIS}, for the visible range, shows an evident wavelength-dependence along the line core bisector slopes.
% In contrast, Figure~\ref{fig:bisector slope NIR}, for the near infrared range, illustrates a poor wavelength dependence, along with a greater amount of scatter in the dataset.
% Due to this difference, we focused our analysis of the anomalous chromodependence presented through line asymmetry, only to the visible range.
% For each bin in both atlases, a linear fit was applied to characterize the wavelength dependence.
% For the integrated flux spectrum, was found that for a line depth fixed window, the slopes decrease with wavelength (see Figure~\ref{fig:velocity bins bisector VIS}).
% The same analysis was performed on the disk-center flux, revealing the same general behavior (see Figure~\ref{fig:velocity bins bisector SPA}).
% However, a quantitative difference was found: The displacement in the line core bisector slope values is smaller across all line depths in the disk-center data.
% Despite this overall difference, the values for the middle depth range of $0.45 \pm 0.05$ are relative closer.
% This indicates that the same physical behavior predominantly affects the line profiles within this specific range.
% Furthermore, the shift in every linear fit at a fixed velocity expresses the granulation relation. 
% For the disk-integrated flux spectrum, this relation is not uniform but instead describes a fluctuating displacement around $-200$ m/s, indicating that wavelength displacements increase for values less than this value and decrease for greater values. 
% In contrast, the disk-center spectrum shows a uniform, quadratically increasing behavior across wavelength shifts. 
% Since limb darkening is negligible in this last atlas, the conclusion is that this effect produces nonuniform shifts. 
% These relations are presented in the H and J bands of the near-infrared range for the IAG solar flux atlas. 

% \subsection{Line depth-dependence of wavelength shifts}
% As shown in Figure~\ref{fig:Granulation solar flux}, the convective velocity depends on line depth, as described in literature.
% Moreover, the anomalous chromodependence makes its presence known in the fact that this \textit{granulation curve} is not unique or universal, but depends on the wavelength range.
% To further characterize this trend, we performed an analysis of line depth versus wavelength at fixed velocities. 
% The coefficients from each linear fit applied to the velocity bins were plotted against velocity to identify patterns between the atlases. 
% However, as there is no disk-center spectrum for the near-infrared range available for comparison, the coefficients are presented in Figure~\ref{fig:coeff plot vel NIR} but are not analyzed comparatively.
% Therefore, the wavelength shift is uniform in the disk-center spectrum; the wavelenght shift in the disk-flux spectrum shows the same behavior as the granulation curves; and for the infrared the trend follows the IAG spatially resolves quiet sun atlas. 
% Although there is no theoretical explanation for the change in velocity displacement along the third signature plot, a key conclusion can be drawn: A phenomenon exists that generates the anomalous chromodependence, and it is not negligible in the disk-integrated spectrum.
% The initial hypothesis was that solar rotation would be the cause, but only if a velocity shift was present in the disk-integrated spectrum and absent at the disk-center. 
% However, the velocity shift was observed in both spectral datasets, with the magnitude of the shift from the disk-center spectrum being greater than in the IAG solar flux atlas. 
% This was unexpected, as rotational Doppler broadening is negligible at the disk center, thereby discarding rotation as the cause.

% \subsection{Line depth-dependence on excitation potential}
% The distribution of wavelength along excitation potential is according to literature, showing that lower wavelength have less excitation potential on the lowest energy level than high wavelength.
% Figure~\ref{fig:velocity bins energy plot} explicitly shows the dependence on the highest values $(2.5$ to $5.0)$ eV for the excitation potential of lower energy levels across line depth, which can be modeled with a lineal fit.
% Specifically, Figure~\ref{fig:velocity bins energy VIS} and Figure~\ref{fig:velocity bins energy NIR} shows the lineal fit for each velocity bins with a range of $(2.5-5.0)$ eV for excitation potential.
% However, the near infrared range don't shows a uniform behavior across excitation potential. 
% Then, we limit the line depth-dependence on excitation potential analysis to the visible range.

% Regarding line depth-dependence on excitation potential, the coefficients curve depends strongly on wavelength and weakly on excitation potential. 
% The maximum displacement for the disk-integrated flux occurs at  $-200$ m/s, while for the IAG spatially resolved quiet Sun, it occurs at $0$ m/s. 
% The standard granulation curves for the entire wavelength range in the IAG solar flux atlas, described by Equation~\eqref{eq:standard line NIR} and Equation~\eqref{eq:standard line VIS} , provide a description of the anomalous chromodependence, as both depend on wavelength and the velocity shift. 
% Finally, the line core bisector has a shift along line depth that is linear, as it is determined by a slope. 
% This finding agrees with the literature, as these shifts are due to Fe I line displacement.


