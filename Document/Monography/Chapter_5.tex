%Por cada objetivo va una conclusion
In conclusion, this project characterized the convective blueshift in solar absorption lines and its dependence on both line depth and wavelength range.

Specifically, was produced a refined, blend-free list of Fe I lines and derived solar granulation patterns with minimal scatter. 
Also provided a detailed description of how the granulation pattern varies with wavelength, establishing a new, color-dependent standard curve.


For wavelength shifts at a fixed velocity rotation is not the cause of deformation and the disk-integrated flux presents a fluctuates behavior around the relative velocity of the sun $-200$ m/s.

For the disk-integrated flux the maximum displacement of excitation along line depth occurs at $-200$ m/s, the relative velocity of the Sun.
While for the IAG spatially resolved quiet sun the maximum excitation potential displacement occurs at $0$ m/s.

We establish a general characterization for all wavelength range in the IAG solar flux atlas.
The near infrared have a general line to describe the anomaly chromodependence.
For the visible range we find each granulation curve and a description for the coefficients.