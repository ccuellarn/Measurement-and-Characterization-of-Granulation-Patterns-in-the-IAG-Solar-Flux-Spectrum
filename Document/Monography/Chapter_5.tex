This project has characterized the convective blueshift in solar absorption lines and the anomalous chromodependence present across the three signatures of solar granulation: Line broadening, line asymmetry, and the line-depth dependence of convective blueshift following David Gray fundamental research (see~\cite{Gray_2009} and references therein). 
A foundational outcome of this work is the production of a refined, blend-free list of Fe I lines, from which solar granulation patterns were derived with minimal scatter. 
This list, complete with measured parameters such as convective blueshift, line core curvature, line profile bisector slope, excitation potential, probability of transition, damping coefficient and line depth for each line, should be published to serve as an astronomy community resource. 
Furthermore, our analysis statistically justified a window of $\SI{0.1}{\angstrom}$ around the line core as the optimal width for analysis.
Our statistical analysis also demonstrates the precision of the spectral line selection and the optimal width parameter established by Allende and Garcia~\cite{Allende_Garcia_1998}, whose methodology we followed.

A key contribution of this work is the detailed description of how granulation patterns vary with wavelength, leading to the establishment of color-dependent standard granulation curves for the entire spectrum. 
We confirmed the expected trend where deeper lines are generally located at shorter wavelengths. 
More significantly, we identified and characterized several anomalous signals of chromodependence: For a given line depth, short-wavelength lines exhibit greater convective blueshifts, higher (more positive or less negative) core bisector slopes, and narrower line profiles. 
Another relevant finding is that solar rotation is not the cause of this anomalous chromodependence observed in the third signature, as the velocity shift was present and even greater in the disk-center spectrum where rotational broadening is negligible. 
For the near infrared range, the chromodependence closely follows a general trend line, with the line bisector slope shifting linearly with depth. 
In the more complex visible range, we identified individual granulation curves and described their governing coefficients.

Focusing now on the specific signatures, we find the analysis of the line core curvature for shallow lines revealed a clear linear trend in the visible range. 
The slope of this relationship was found to be $(1.83 \pm 0.08)\times 10^{10}$ $\text{m}^{-1}$ for the disk-integrated flux and a significantly higher $(2.94\pm 0.16)\times 10^{10}$ $\text{m}^{-1}$ for the disk-center spectrum. 
The absence of wavelength dependence in this relationship indicates that line core curvatures are primarily governed by velocities (thermal, convective, and rotational), with negligible influence from atomic effects. 
The steeper slope at the disk-center confirms that rotation is a major contributor to line broadening in the integrated spectrum. 
By applying the theoretical model, we derived values for $\langle v_{\text{conv}}^2 \rangle$ of $3.15$ $(\text{km/s})^2$ for the disk-integrated flux and $2.20$ $(\text{km/s})^2$ for the disk-center spectrum. 
The fact that the convective velocity appears lower at the disk center, where it should be most fully observable, points to an unresolved effect that needs further investigation.

Regarding line asymmetry, we focused on the middle range of line depths ($0.3-0.6$), where the convection effect is less pronounced. 
A clear wavelength dependence was observed in the line core bisector slopes for the visible range, in contrast to the poor dependence and significant scatter found in the near infrared. 
For a fixed line depth, the bisector slopes in the integrated flux spectrum decrease with wavelength. 
While the same general behavior was found in the disk-center spectrum, the magnitude of the displacement in bisector slope values was consistently smaller. 
This difference highlights the effect of limb darkening in producing non-uniform shifts in the integrated spectrum. 
The analysis of wavelength shifts at fixed velocities revealed that for the disk-integrated spectrum, the shift is not uniform but fluctuates, while the disk-center spectrum exhibits a uniform, quadratically increasing behavior.

Finally, the study into the line depth-dependence on excitation potential revealed that the phenomenon is most pronounced for high-excitation potentials ($2.5$ to $5.0$ eV) and is best modeled in the visible range due to non-uniform behavior in the near infrared. 
The resulting coefficients depend strongly on wavelength and weakly on excitation potential. 
A notable finding is the difference in the velocity of maximum displacement cause it occurs at $-200$ m/s for the disk-integrated flux but shifts to $0$ m/s for the spatially resolved quiet Sun atlas.

For future work, it is imperative to study atomic broadening effects and develop more sophisticated models of convection to investigate the roles of temperature and density fluctuations in driving the anomalous chromodependencies uncovered in this study. 
The standard granulation curves and the comprehensive line list provided here offer a robust foundation for such endeavors.
Finally, the statistical analysis performed suggests that an algorithm could be developed to select customized window widths based on specific spectral line conditions.

