%Por cada objetivo va una conclusion

In conclusion, this project characterized the convective blueshift in solar absorption lines and the anomalous chromodependence present across the three signatures of convection. 
Specifically, we produced a refined, blend-free list of Fe I lines and derived solar granulation patterns with minimal scatter. 
The analysis statistically justified a window of $\SI{1}{\miliangstrom}$ around the line core as the optimal width for examination, underscoring the accuracy and unprecedented precision of the IAG solar flux atlas. 

Furthermore, we provided a detailed description of how granulation patterns vary with wavelength, establishing color-dependent standard granulation curves for the entire spectrum. 
Our results determine that rotation is not the cause of the anomalous chromodependence observed in the third signature plot, and we established a general characterization for the full wavelength range of the atlas. 
For the near-infrared, a general trend line describes the chromodependence, while the line bisector slope shift along the line depth is linear. 
In the visible range, we identified individual granulation curves and a description for the coefficients. 


For future work, it is necessary to study atomic broadening effects and model convection to investigate the role of temperature and density fluctuations.
