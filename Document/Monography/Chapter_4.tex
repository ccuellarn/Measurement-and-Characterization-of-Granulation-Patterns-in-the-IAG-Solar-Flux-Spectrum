Our results were separated into the three signatures of convection around the main phenomenon of chromodependence.

\section{The first signature}

\subsection{Detailed view of line broadening}
The core curvature was calculated following the equation \eqref{core curvature} and plotted against line depth as shown in figure \ref{Sharpness ranges}.

\begin{figure}[H]
     \centering
     \begin{subfigure}{1.1\textwidth}
         \includegraphics[width=\textwidth]{Images/Results/Curvature_VIS.pdf}
         \caption{Visible range.}

         \label{curvature VIS}
     \end{subfigure}
\hfill
     \begin{subfigure}{1.1\textwidth}
         \includegraphics[width=\textwidth]{Images/Results/Curvature_NIR.pdf}
         \caption{Near infrared range.}
         \label{curvature NIR}
     \end{subfigure}

        \caption{Line core curvature in the Solar Flux Atlas. Is visible a characteristic curve with line depth-dependence along wavelength.}
        \label{Sharpness ranges}
\end{figure}

As shown in the figure \ref{curvature NIR}, the plot for the near infrared range is visible as a natural division for wavelengths in $11400 \mathring{A}$, which corresponds to telluric lines of absorption in the atmosphere. Plotting all ranges of the Solar Flux Atlas is visible a line depth-dependence along wavelength.

\begin{figure}[H]
    \centering
    \includegraphics[width=1.1\linewidth]{Images/Results/Sharpness_ALL.pdf}
    \caption{Line core curvature in the Solar Flux Atlas. The near infrared range presents a natural division due telluric lines in the atmosphere.}
    \label{Sharpness}
\end{figure}

As shown in figure \ref{Sharpness} the chromodepence presented on the third signature plot is characteristic of weaker lines. This type of lines are known to represent the lower potential excitations. In consequence, the dependence of line depth with the lower energy potential was studied.

\subsection{Line depth dependence on lower energy potential}

The Nave list for Fe I lines included the values for the highest and the lowest potential energy potential. As the chromodepence is shown on weaker lines we study only the lowest potential energy.

As shown in the figure () the relative velocity was plotted against lower potential energy, where it is not clear a strong shift on the velocity. However, if it creates velocity bins and plotted line depth against lower energy potential, it sees a dependency on the highest values for the lowest energy potential energy.

With this we can give a sight that the chromodepence on the curvature and more characteristics is due to a shift on the lower potential energy dependending on the line depth.

\section{The second signature}

\subsection{Detailed view of line profile bisector asymmetry}

The figure \ref{C bisector graph} shows the behavior of the C-curved line profile bisector.

\begin{figure}[H]
    \centering
    \includegraphics[width=1.1\linewidth]{Images/Results/Bisector_VIS.pdf}
    \caption{Line profile bisector slope for the Solar Flux Atlas separated by wavelength range. The behavior of the plot is according to the C-curved shape of the line bisector affected by convection movement.}
    \label{C bisector graph}
\end{figure}

The behavior of the plot is according to the C-curved shape of the line bisector affected by convection movement, and calculated following the equation \eqref{third derivative relation}.

Hamilton and Lester () gives the sight of the mean bisectors following the granulation pattern behavior. Later, Gray and Oostra show that the bisectors need to follow the granulation pattern as the form of we calculated velocities and bisectors.

However there is an interesting phenomenon called Flux déficit. This can be interpreted as the radiation of black body, and the redshift of the line presents a displacement on the flux. This made a broadening on the line and a displacement of the bisector. 

Following the Gray and Oostra method for the Planck distribution obtention, where the line analyzed is 6253 and the line altered is the line on the granulation pattern, was found a flux imbalance for and temperatures of the mean, the maximum and the HWM point of the bisector.

Taking the model on gray 2017, the respective temperatures are . The difference in temperature indicates a larger temperature difference.

The motivation is, naturally, that stellar observations are made with lower-resolution spectrographs and often lower signal-to-noise ratios than we have with the solar atlases (Gray, 2018).

\section{The third signature}

\subsection{Chromodependence on the granulation pattern}

The granulation patterns for the IAG Solar Flux Atlas in all the wavelength range was obtained.

\begin{figure}[H]
    \centering
 \includegraphics[width=1.1\linewidth]{Images/Results/GranulationPattern_ALL.pdf}
    \caption{Granulation pattern obtained for the Solar Flux Atlas. The wavelength shift dependence is along the line depth.}
    \label{Granulation solar flux}
\end{figure}

As shown in the figure \ref{Granulation solar flux} the behavior along the line depth is according to literature, in which is clear the wavelength shift dependence along the line depth or chromodependence.

\subsection{Characterization of chromodepence on granulation pattern}

As mentioned before, the characterization given to this plot by Gray and Oostra \cite{Gray_Oostra_2018} don't take into account all the wavelengths.

For a different perspective into the characterization of this behavior, different second order polynomials fit for each color range were generated.

\begin{figure}[H]
    \centering
    \includegraphics[width=1.1\linewidth]{Images/Results/ColorCurves_VIS.pdf}
    \caption{Granulation pattern for the Solar Flux Atlas with color curves. The tendency on the curves is more pronounced in the violet and red range.}
    \label{Curves Solar Flux}
\end{figure}

The tendency on the curves is more pronounced in the violet and red range. This can be interpreted as the standard curve from Gray and Oostra is shifted and flattered along decreasing wavelengths.
The coefficients of each color curve were plotted in figure \ref{Curves Solar Flux coef}.

\begin{figure}[H]
    \centering
    \includegraphics[width=0.9\linewidth]{Images/Results/Relation_coef.pdf}
    \caption{Coefficient tendency for the second order polynomial fit in each color curve. All of the coefficients show a rational increment with decreasing wavelength.}
    \label{Curves Solar Flux coef}
\end{figure}

All of the coefficients show a rational increment with decreasing wavelength, the values are shown in the table \ref{values coefficients}.

\begin{table}[H]
    \centering
\begin{tabular}{|l|l|l|l|}
    \hline
\textbf{Wavelength range $(\mathring{A})$} & \textbf{Second order} & \textbf{First order} & \textbf{Shift} \\ \hline
3800-4270         &    4924.2052         &     -14022.7063                 &     10434.9782           \\\hline
4270-4760           &     757.0966         &     -2590.7356                 &   2719.8840             \\\hline
4760-4970         &      263.3435        &     -974.5109                 &    1574.3597            \\ \hline
4970-5700         &     204.8798         &       -709.6070               &    1413.5460            \\ \hline
5700-5810         &       155.9806       &      -432.9495                &     1256.0760           \\\hline
5810-6180           &        190.1927      &      -739.9248                 &    1818.1626         \\ \hline
6180-7800            &        142.9648      &       -166.1503               &     1039.8561        \\ \hline
7800-11000            &        160.4951      &       -29.5577               &     1067.7370        \\  \hline
\end{tabular}
\caption{Values for the coefficients shown in figure \ref{Curves Solar Flux coef}}
\label{values coefficients}
\end{table}

\subsection{Frequency shifts for rotation hypothesis}

For a description of this tendency was performed an analysis of line depth against wavelength. The hypothesis was: “If there exists a frequency shift only in the Solar Flux Atlas, then rotation could be the cause of this phenomenon.”
However, the frequency shift was observed  in both spectral datasets. This was initially unexpected because the rotation is negligible at the disc center.

For the measurement of its frequency, a range from $4300 \mathring{A}$ to $5600 \mathring{A}$ was taken. Then was sorted all wavelengths from both atlases into 50 m/s velocity bins (see figure \ref{velocity bins}).

\begin{figure}[H]
     \centering
     \begin{subfigure}{1.0\textwidth}
         \includegraphics[width=\textwidth]{Images/Results/Velocity bins VIS_plot.pdf}
         \caption{Visible range for the Solar Flux Atlas.}

         \label{vel bins plot VIS}
     \end{subfigure}
\hfill
     \begin{subfigure}{1.0\textwidth}
         \includegraphics[width=\textwidth]{Images/Results/Velocity bins SPA_plot.pdf}
         \caption{Visible range for the Spatially Resolved Quiet Sun Atlas.}
         \label{vel bins plot SPA}
     \end{subfigure}

        \caption{Comparision between atlases with velocity bins for the relation between wavelength and line depth. For each velocity bin was plotted wavelength against line depth to measure a frequency shift with a first order polynomial fit.}
        \label{Velocity bins}
\end{figure}

For each velocity bin was plotted wavelength against line depth to measure a frequency shift with a first order polynomial fit (see figure \ref{Velocity bins slopes})

\begin{figure}[H]
     \centering
     \begin{subfigure}{1.0\textwidth}
         \includegraphics[width=\textwidth]{Images/Results/Velocity bins VIS.pdf}
         \caption{Velocity bins for the Solar Flux Atlas plot (see figure \ref{vel bins plot VIS}).}

         \label{vel bins SPA}
     \end{subfigure}
\hfill
     \begin{subfigure}{1.0\textwidth}
         \includegraphics[width=\textwidth]{Images/Results/Velocity bins SPA.pdf}
         \caption{Velocity bins for the Spatially Resolved Quiet Sun Atlas (see figure \ref{vel bins plot SPA}).}
         \label{vel bins SPA}
     \end{subfigure}

        \caption{Comparision between atlases with velocity bins for the relation between wavelength and line depth. The first order fit is showed for each velocity bin.}
        \label{Velocity bins slopes}
\end{figure}

The table \ref{velocity bins slopes values} shows the obtained slopes in both atlases. 
The value of slopes for the Spatially Resolved Quiet Sun Atlas are greater than the Solar Flux Atlas, which contradicts the initial hypothesis.
There is no existence of frequency shifts in the granulation patterns.

\section{Higher quality graphs}

To show the improvement on the quality in the third signature plots, the figure \ref{Scattered points} compare the plot obtained by Ellwarth \cite{Ellwarth_2023} and the graphic.

\begin{figure}[H]
     \centering
     \begin{subfigure}{0.9\textwidth}
         \includegraphics[width=\textwidth]{Images/Results/Standard curve_Ellwarth.pdf}

     \end{subfigure}
\hfill
     \begin{subfigure}{0.8\textwidth}
         \includegraphics[width=\textwidth]{Images/Granulation pattern Ellwarth.jpg}
     \end{subfigure}

        \caption{We realized the same graphic for the comparision with the Ellwarth article to show the less scattered points.}
        \label{Scattered points}
\end{figure}

The less scattered points and the improvement on the chromodepence identification is evident.
