Our results can be summarized in three principal aspects: Chromodependence on the granulation pattern along the line depth; a detailed view of the characteristic curvature, asymmetries, and sharpness of spectral lines and higher-quality graphs with reduced scatter.

\section{The third signature: Chromodependence on the granulation pattern}
We obtain the granulation patterns for the IAG Solar Flux Atlas for all the wavelength.
\begin{figure}[H]
    \centering
    \includegraphics[width=0.6\linewidth]{GranulationPattern_VIS.png}
    \includegraphics[width=0.6\linewidth]{GranulationPattern_NIR.png}
    \caption{Granulation pattern for the Solar Flux Atlas.}
    \label{GP VIS and NIR}
\end{figure}
The behavior along the line depth is according to literature, in which is clear the chromodependence. 
For a different try into the characterization of this behavior we try to genearte different second order polynomial fits for each color range.
\begin{figure}[H]
    \centering
    \includegraphics[width=0.6\linewidth]{Chromocharacterization_VIS.png}
    \caption{Granulation pattern for the Solar Flux Atlas with color curves.}
    \label{Curves Solar Flux}
\end{figure}
But this is not standard. 
So, we performed an analysis of line depth against wavelength. Our hypothesis was: “If the chromodependence is present only in the Solar Flux Atlas, then rotation could be the cause of this phenomenon”
Surprisly, we observed a chromodependence in both spectral datasets. 
We start taking the granulation pattern of the center disk
\begin{figure}[H]
    \centering
    \includegraphics[width=0.6\linewidth]{GranulationPattern_SPA.png}
    \caption{Granulation pattern for the Spatially Resolved Quiet Sun Atlas at $\mu=1$.}
    \label{GP SPA}
\end{figure}
This was initially unexpected because the rotation is negligible at the disc center.
For the measurement of its velocity we took a 4300-5600 A range and sorted all lines from both atlases into 50 m/s velocity bins.
\begin{figure}[H]
    \centering
    \includegraphics[width=0.6\linewidth]{V_bins_VIS.png}
    \includegraphics[width=0.6\linewidth]{V_bins_SPA.png}
    \caption{Comparision between atlases with velocity bins for the relation between wavelength and line depth.}
    \label{Velocity analysis}
\end{figure}
For the measurement of rotation in the lines we made a first order polynomial fit for each one.
\begin{figure}[H]
    \centering
    \includegraphics[width=0.6\linewidth]{Velocity bins SPA.png}
    \includegraphics[width=0.6\linewidth]{Velocity bins VIS.png}
    \caption{Comparision between atlases with velocity bins for the relation between wavelength and line depth. The first order fit is showed for each velocity bin.}
    \label{Velocity analysis slopes}
\end{figure}
The clue here is the value of slopes in the Spatially Resolved Atlas is greater than the Solar Flux, which contrindicates for complete the initial hypothesis.
The rotation is not the cause.
 
\section{The first and second signature: Detailed view of line broadening and asymmetry}
We made two graphics corresponding to each signature related to the line bisector and core asymmetries.
The slope of line bisector shows the behavior of the c curved line profile bisector.
\begin{figure}[H]
    \centering
    \includegraphics[width=0.6\linewidth]{CBisector_VIS.png}
    \includegraphics[width=0.6\linewidth]{Cbisector_slopes.png}
    \caption{C bisector graph for slopes in ehich is clear the convection movement and how its affected in the profile.}
    \label{C bisector graph}
\end{figure}
The interesting one is the core curvature but multiplied by the squared wavelength, cause it affirms the chromodependence in the weaker lines as the literature says.
\begin{figure}[H]
    \centering
    \includegraphics[width=0.6\linewidth]{Sharpness_VIS.png}
    \includegraphics[width=0.6\linewidth]{Sharpness_NIR.png}
    \caption{Sharpness of the core bisector in the solar flux atlas separated by range.}
    \label{Sharpness}
\end{figure}
We can see a natural division for lines in the infrared range, which is interesting due to corresponding Teluric lines of absorption in the atmosphere.
Theres is a little division in $11400 A$ who separates the lines i one that follow the mean curve and others than doesnt do that.
\begin{figure}[H]
    \centering
    \includegraphics[width=0.6\linewidth]{Sharpness_ALL.png}
    \caption{Sharpness of the core bisector in the solar flux atlas. Is clear that the infrared follow two different behavior due teluric lines.}
    \label{Sharpness}
\end{figure}
This is the graphic which is clear that we have chromodependence in the weaker lines.

\section{Higher quality graphs}
In comparation of Ellwarth graphics, which is the study most recent, there is a better resolution in the different graphs without scattered points.
\begin{figure}[H]
    \centering
    \includegraphics[width=0.6\linewidth]{ConvectiveBlueshift_VIS_Gray.png}
    \caption{We realized the same graphic for the comparision with the Ellwarth article to show the less scattered points.}
    \label{Scattered points}
\end{figure}