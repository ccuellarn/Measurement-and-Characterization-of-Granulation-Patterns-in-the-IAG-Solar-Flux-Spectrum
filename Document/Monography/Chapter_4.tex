Our results were separated into three sections following the signatures of convection.
All of them analyses these signatures around the main phenomenon of wavelength-dependence.

\section{The first signature: Line broadening}

\subsection{Line depth-dependence on line core curvature}
The line core curvatures were calculated following Equation~\eqref{eq:core curvature}, and plotted against line depth as shown Figure~\ref{fig:curvatures solar flux}.
In general, the behavior described in previous research was observed.
For shallow lines, the line core curvature is proportional to depth as expected. 
Furthermore, for deep lines the line core curvature is reduced by saturation with strongly wavelength-dependence.

\begin{figure}[H]
     \centering
     \begin{subfigure}{0.9\textwidth}
         \includegraphics[width=\textwidth]{Images/Results/First signature/Curvature_VIS.pdf}
         \caption{Line core curvatures for the visible range. While shallow lines curvature is proportinal to depth, deep lines curvature is reduced by saturation with is strongly wavelength-dependence.}\label{fig:curvature VIS}
     \end{subfigure}
\hfill
     \begin{subfigure}{0.9\textwidth}
         \includegraphics[width=\textwidth]{Images/Results/First signature/Curvature_NIR.pdf}
         \caption{Line core curvatures for the near infrared range. The plot shows a conspicuous partition around $\SI{14000}{\angstrom}$}\label{fig:curvature NIR}%
     \end{subfigure}
     \caption{Line core curvatures for the IAG solar flux atlas separated in wavelength ranges.}\label{fig:curvatures solar flux}%
\end{figure}

As shown in Figure~\ref{fig:curvature NIR} the near infrared range exhibits a natural division at approximately $\SI{14000}{\angstrom}$, a region dominated by telluric absorption lines from Earth's atmosphere.
The division separates the J band, which refers to an atmospheric transmission window of $\SI{3000}{\angstrom}$ centered on $\SI{12500}{\angstrom}$; and the H band, which refers to an window of $\SI{3500}{\angstrom}$ centered on $\SI{16500}{\angstrom}$~\cite{McLean_2008}.
Plotting the entire spectral range of the IAG solar flux atlas, a clear line depth-dependence of the line core curvature as a function of wavelength is observed.

\begin{figure}[H]
    \centering
    \includegraphics[width=1.0\linewidth]{Images/Results/First signature/Curvature_ALL.pdf}
    \caption{Line core curvatures in the IAG solar flux atlas. While the J band for the near infrared range adjust to the general shallow lines trend, the H band exhibit wavelength-dependence.}\label{fig:all curvature solar flux}%
\end{figure}

While the J band for the near infrared range adjust to the general shallow lines trend in Figure~\ref{fig:all curvature solar flux}, the H band exhibit wavelength-dependence.
Despite the line depth-dependent shifts along wavelength, the visible range of the IAG solar flux atlas exhibits a clear linear tendency for shallow lines (see Figure~\ref{fig:all curvature solar flux}).
Consequently, a linear fit was applied to the visible range for the IAG solar flux atlas in the range $0.0 - 0.1$ of line depth. 

\begin{figure}[H]
    \centering
    \includegraphics[width=1.0\linewidth]{Images/Results/First signature/Curvature_Slope_VIS.pdf}
    \caption{Linear fit applied to the range $(0.0-0.1)$ of line depth in the visible range of the IAG solar flux atlas.}\label{fig:curvature slope VIS}
\end{figure}

As shown in Figure~\ref{fig:curvature slope VIS}, the slope of the relationship has a value of $(1.83 \pm 0.08)\times 10^{10}$.
The absence of wavelength-dependence in this spectral range indicates that line core curvatures have a net dependence on velocities (thermal, convective and rotational), with negligible influence from atomic effects.
To confirm this statement,  the line core curvature was also plotted for the visible range using the IAG spatially resolved quiet sun atlas at the solar limb $\mu=0$ as shown Figure~\ref{fig:curvature SPA}.
In other words, this analysis uses light from the center of the solar disk, which is not affected by rotational Doppler broadening, thereby isolating the local effects.

\begin{figure}[H]
    \centering
    \includegraphics[width=1.0\linewidth]{Images/Results/First signature/Curvature_SPA.pdf}
    \caption{Line core curvatures for the visible range of the disk center data from the IAG spatially resolved quiet sun atlas.}\label{fig:curvature SPA}%
\end{figure}

A linear fit was applied taking the same range $0.0 - 0.1$ of line depth, the Figure~\ref{fig:curvature slope SPA} shows a value of $(2.94\pm 0.16)\times 10^{10}$ for the slope.

\begin{figure}[H]
    \centering
    \includegraphics[width=1.0\linewidth]{Images/Results/First signature/Curvature_Slope_SPA.pdf}
    \caption{Linear fit applied to the range $(0.0-0.1)$ of line depth in the visible range of the IAG spatially resolved quiet sun atlas.}\label{fig:curvature slope SPA}%
\end{figure}

The result of found a greater slope in the center-disk confirm that rotation is a neligible effect for the line broadening.
This allows deduce the variance of convection speed, knowing the thermal velocity. 

\textcolor{Miku}{IN PROGRESS..}
As mention in chapter 3, we can induce from here the $\sigma_{conv}$ as we the other parameters described in equation () and confirm that rotation is neligible. 
Benja comment: Here you could discuss all that can be learned about line broadening. 
For example:
The slope of the disk-center spectrum doesn't depend on rotation, meaning it depends only on thermal broadening and convection. 
The thermal is known, so the variance of the convection speeds can be deduced.
The slope of the flux spectrum is lower, because it includes rotation. 
The rotation is also known (but more work to calculate); but the three contributions can be assessed. 
Deeper lines deviate from the initial straight line, and this part is wavelength-dependent. 
The cause (s) should be discussed: Saturation in the deepest lines, but maybe other broadening effects (Lorentzian) may become important, and damping coefficients are somewhat related to wavelength by atomic structure.
\textcolor{Miku}{IN PROGRESS..}


\section{The second signature: Line profile asymmetry}

\subsection{The line core bisector slope}

As mention before, convection is the only mechanism that creates asymmetric line profiles.
Consequently, the slope of the line core is a indirect line profile asymmetry measure, expressed as a velocity.

The line core bisector slope was calculated following the equation~\eqref{eq:third derivative relation} and plotted against line depth, as shown Figure~\ref{fig:bisector solar flux}.

The shallow lines, due to their small depth, do not exhibit a fully developed C-shaped bisector. 
Instead, their bisectors shows as a positive slope. 
In contrast, deep lines experiences the convective blueshift in its totality, exhibit a negative bisector slope.
The line depths in the middle range represent lines profiles which are not greatly affected by convection.

\begin{figure}[H]
     \centering
     \begin{subfigure}{0.9\textwidth}
         \includegraphics[width=\textwidth]{Images/Results/Second signature/Bisector_VIS.pdf}
         \caption{Line core bisector slopes for the visible range. }\label{fig:bisector VIS}
     \end{subfigure}
\hfill
     \begin{subfigure}{0.9\textwidth}
         \includegraphics[width=\textwidth]{Images/Results/Second signature/Bisector_NIR.pdf}
         \caption{Line core bisector slopes for the near infrared range.}\label{fig:bisector NIR}%
     \end{subfigure}
     \caption{Line core bisector slopes for the IAG solar flux atlas separated in wavelength ranges. The behavior of both plots is according to the C-curved shape of the line profile bisector affected by convection movement.}\label{fig:bisector solar flux}%
\end{figure}

Searching for a comparision with the IAG spatially resolved quiet sun atlas, only the visible range was analysed.
Then a lineal fit was applied to the line core bisector slope data in the line depth range of $(0.3-0.6)$ to quantify this transition, as shown in Figure~\ref{fig:bisector slope VIS}.

\begin{figure}[H]
    \centering
    \includegraphics[width=1.0\linewidth]{Images/Results/Second signature/Bisector slope_VIS.pdf}
    \caption{Lineal fit adjusted to the range $(0.3-0.6)$ of line core bisector slopes in the IAG solar flux atlas, where is evident an wavelength-dependence.}\label{fig:bisector slope VIS}%
\end{figure}

As there is an evident wavelength-dependence along the line core bisector slopes, a point of comparision was searched comparing with the disk center spectrum (see figure~\ref{fig:bisector SPA}).

\begin{figure}[H]
    \centering
    \includegraphics[width=1.0\linewidth]{Images/Results/Second signature/Bisector_SPA.pdf}
    \caption{Line profile bisector slopes for the visible range in the IAG spatially resolved quiet sun atlas for $\mu=0$.}\label{fig:bisector SPA}
\end{figure}

The bisector slope derived from the disk center is smaller in magnitude compared to the integrated flux, despite both exhibiting the same characteristic trend (see Figure~\ref{fig:bisector slope SPA}).

\begin{figure}[H]
    \centering
    \includegraphics[width=1.0\linewidth]{Images/Results/Second signature/Bisector slope_SPA.pdf}
    \caption{Lineal fit adjusted to the range $(0.3-0.6)$ of line core bisector slopes int disk-center spectrum, where is evident an wavelength-dependence.}\label{fig:bisector slope SPA}
\end{figure}
 
The phenomenon of chromodependence is evident of Figures~\ref{fig:bisector slope VIS} and~\ref{fig:bisector slope SPA}, where the line profile bisectors slopes are organised by decreasing wavelength.
This behavior is clearer on the disk-center spectrum than the disk-integrated flux spectra.

If line depth is separate into bins of $0.1$ and plot the line profile bisector slope against wavelength, shifts in the line bisector slope become apparent as shown in Figure~\ref{fig:velocity bins bisector plot ALL}.

\begin{figure}[H]
    \centering
    \includegraphics[width=1.0\linewidth]{Images/Results/Second signature/Velocity bins bisector plot_VIS.pdf}
    \caption{Line depth bins of $0.1$, where wavelength-dependence of the slopes becomes apparent.}\label{fig:velocity bins bisector plot ALL}%
\end{figure}

For the integrated flux spectrum was found that, for a fixed flux, the slopes decrease with wavelength (see Figure~\ref{fig:velocity bins bisector ALL}).

\begin{figure}[H]
    \centering
    \includegraphics[width=0.9\linewidth]{Images/Results/Second signature/Velocity bins bisector VIS.pdf}
    \caption{Individual plots of line core bisector slope shift across wavelength representing each bin of line depth for the IAG solar flux atlas.}\label{fig:velocity bins bisector ALL}%
\end{figure}

The same analysis was performed on the disk-center flux and the same behavior was found (see Figure~\ref{fig:velocity bins bisector SPA}).

\begin{figure}[H]
    \centering
    \includegraphics[width=0.9\linewidth]{Images/Results/Second signature/Velocity bins bisector SPA.pdf}
    \caption{Individual plots of line bisector slope shift across wavelength representing each bin of line depth for the Spatially Resolved Quiet Sun Atlas.}\label{fig:velocity bins bisector SPA}%
\end{figure}

\textcolor{Miku}{IN PROGRESS..}
So, for fixed flux we have chromodependence..
\textcolor{Miku}{IN PROGRESS..}

\subsection{Flux deficit}

Due to radiation, the redshift part of a line is displced on the flux, inducing a rotation on the c-curved profile bisector.
This phenomenon is called flux deficit.
Hamilton and Lester~\cite{Hamilton_1999} noticed that the behavior of the third signature mimics the bisectors
gives the sight of the mean bisectors following the granulation pattern behavior, this last is discussed in the next section. 
Later, Gray and Oostra~\cite{Gray_Oostra_2018} show that the bisectors need to follow the granulation pattern as the form of we calculated velocities and bisectors.

For comparision with Gray and Oostra work, was taken the line $6254.2850\mathring{A}$ as shown in figure~\ref{fig:bisector flux deficit}.

\begin{figure}[H]
    \centering
    \includegraphics[width=1.0\linewidth]{Images/Results/Second signature/Flux deficit_6253.pdf}
    \caption{C-curved line profile bisector for the $6254.2850\mathring{A}$ and the standard curve for the green range from the third signature plot.}\label{fig:bisector flux deficit}
\end{figure}

Following the same method, the figure~\ref{fig:flux deficit} shows a flux imbalance described on the distribution and temperatures of the mean, the maximum and the RMS point of the bisector.

\begin{figure}[H]
    \centering
    \includegraphics[width=1.0\linewidth]{Images/Results/Second signature/Flux deficit curve_6253.pdf}
    \caption{Flux deficit curve for the $6254.2850\mathring{A}$ and the temperatures of the mean, the maximum and the RMS point of the bisector.}\label{fig:flux deficit}
\end{figure}

The idea is add the flux deficit using the color curves and just report the velocities with the percent of the line corresponds that velocity.
As we found the standard curves for all the ranges, the same analysis was realized for a random line in the range of the respective standard curve (im working on that).

\section{The third signature: Line depth-dependence on wavelength shifts}

\subsection{The granulation pattern}

The granulation patterns for the IAG Solar Flux Atlas in all the wavelength range was obtained.

\begin{figure}[H]
    \centering
 \includegraphics[width=1.0\linewidth]{Images/Results/Third signature/GranulationPattern_ALL.pdf}
    \caption{Granulation pattern obtained for the Solar Flux Atlas. The wavelength shift dependence is along the line depth.}\label{fig:Granulation solar flux}
\end{figure}

As shown in Figure~\ref{fig:Granulation solar flux}, the behavior along line depth is according to literature, clearly demonstrating the line depth-dependent wavelength shift, also known as chromodependence.
To further characterize this trend, an analysis of line depth versus wavelength was performed. The guiding hypothesis was: “If the velocity shift is present only in the disk-integrated Solar Flux Atlas and absent at the disk center, then solar rotation is the cause.”
However, the velocity shift was observed in both spectral datasets. 
This was initially unexpected, as rotational doppler broadening is negligible at the disk center.

To measure these wavelength shifts, the analysis was restricted to the spectral range from $4300 \mathring{A}$ to $5600 \mathring{A}$. 
The wavelengths from both atlases were then sorted into velocity bins of $50m/s$ (see figure~\ref{fig:Velocity bins}).

\begin{figure}[H]
     \centering
     \begin{subfigure}{1.0\textwidth}
         \includegraphics[width=\textwidth]{Images/Results/Third signature/Velocity bins VIS_plot.pdf}
         \caption{Visible range for the Solar Flux Atlas.}\label{fig:vel bins plot VIS}
     \end{subfigure}
\hfill
     \begin{subfigure}{1.0\textwidth}
         \includegraphics[width=\textwidth]{Images/Results/Third signature/Velocity bins SPA_plot.pdf}
         \caption{Visible range for the Spatially Resolved Quiet Sun Atlas.}\label{fig:vel bins plot SPA}
     \end{subfigure}

        \caption{Comparision between atlases with velocity bins for the relation between wavelength and line depth. For each velocity bin was plotted wavelength against line depth to measure a frequency shift with a first order polynomial fit.}\label{fig:Velocity bins}
\end{figure}

For each velocity bin was plotted wavelength against line depth to measure a frequency shift with a first order polynomial fit (see figure~\ref{fig:Velocity bins slopes})

\begin{figure}[H]
     \centering
     \begin{subfigure}{1.0\textwidth}
         \includegraphics[width=\textwidth]{Images/Results/Third signature/Velocity bins VIS.pdf}
         \caption{Velocity bins of $50m/s$ for the Solar Flux Atlas.}\label{fig:vel bins VIS}
     \end{subfigure}
\hfill
     \begin{subfigure}{1.0\textwidth}
         \includegraphics[width=\textwidth]{Images/Results/Third signature/Velocity bins SPA.pdf}
         \caption{Velocity bins of $50m/s$ for the Spatially Resolved Quiet Sun Atlas.}\label{fig:vel bins SPA}
     \end{subfigure}

        \caption{Individual plots of lower excitation energy across line depth representing each bin of velocity.}\label{fig:Velocity bins slopes}
\end{figure}

The tables~\ref{tab:velocity bins SPA} and~\ref{tab:velocity bins VIS} shows the obtained slopes in both atlases. 

\begin{table}[H]
    \centering
\begin{tabular}{||c|c|c|c|c||}
\hline
\multicolumn{1}{|c|}{\textbf{Velocity bin}} & \multicolumn{1}{c|}{\textbf{Shift}} & \multicolumn{1}{c|}{\textbf{Slope}} & \multicolumn{1}{c|}{\textbf{Error slope}} & \multicolumn{1}{c|}{\textbf{Error shift}} \\ \hline
50 & 1.1937 & -0.000063 & 0.000022 & 0.1166 \\ \hline
0 & 1.1720 & -0.000056 & 0.000018 & 0.0923 \\ \hline
-50 & 1.2840 & -0.000087 & 0.000021 & 0.1087 \\ \hline
-100 & 1.3953 & -0.000112 & 0.000012 & 0.0584 \\ \hline
-150 & 1.4327 & -0.000124 & 0.000017 & 0.0842 \\ \hline
-200 & 1.5028 & -0.000142 & 0.000013 & 0.0658 \\ \hline
-250 & 1.5712 & -0.000165 & 0.000018 & 0.0890 \\ \hline
-300 & 1.6233 & -0.000184 & 0.000019 & 0.0971 \\ \hline
-350 & 1.4129 & -0.000154 & 0.000055 & 0.2583 \\ \hline
-400 & 1.6455 & -0.000208 & 0.000033 & 0.1617 \\ \hline
-450 & 1.1952 & -0.000144 & 0.000060 & 0.3094 \\ \hline
\end{tabular}
\caption{Slopes for the first order polynomial fit in each velocity bin for the Spatially Resolved Quiet Sun Atlas.}\label{tab:velocity bins SPA}
\end{table}

\begin{table}[H]
    \centering
\begin{tabular}{||c|c|c|c|c||}
\hline
\multicolumn{1}{|c|}{\textbf{Velocity bin}} & \multicolumn{1}{c|}{\textbf{Shift}} & \multicolumn{1}{c|}{\textbf{Slope}} & \multicolumn{1}{c|}{\textbf{Error slope}} & \multicolumn{1}{c|}{\textbf{Error shift}} \\ \hline
-100 & 1.2704 & -0.000091 & 0.000015 & 0.0797 \\ \hline
-150 & 1.2601 & -0.000094 & 0.000013 & 0.0673 \\ \hline
-200 & 1.3340 & -0.000115 & 0.000019 & 0.0985 \\ \hline
-250 & 1.3920 & -0.000133 & 0.000023 & 0.1164 \\ \hline
-300 & 1.3724 & -0.000139 & 0.000017 & 0.0874 \\ \hline
-350 & 1.3106 & -0.000137 & 0.000015 & 0.0744 \\ \hline
-400 & 1.2119 & -0.000128 & 0.000018 & 0.0866 \\ \hline
-450 & 1.4212 & -0.000181 & 0.000039 & 0.1947 \\ \hline
-500 & 1.3787 & -0.000189 & 0.000052 & 0.2609 \\ \hline
\end{tabular}
\caption{Slopes for the first order polynomial fit in each velocity bin for the Solar Flux Atlas.}\label{tab:velocity bins VIS}
\end{table}

The value of slopes for the Spatially Resolved Quiet Sun Atlas are greater than the Solar Flux Atlas, which contradicts the initial hypothesis.

\subsection{Line depth-dependence on excitation potential for the lowest energy level}  

As shown Figure~\ref{fig:lower energy VIS} the relative velocity was plotted against excitation potential for lowest energy level.
The distribution of wavelength along excitation potential is according to literature, showing that lower wavelength have less excitation potential on the lowest energy level than high wavelength.

\begin{figure}[H]
    \centering
    \includegraphics[width=1.0\linewidth]{Images/Results/Third signature/Velocity lower potential_VIS.pdf}
    \caption{Relative velocity against excitation potential for lowest energy level in the visible range of the IAG solar flux atlas.}\label{fig:lower energy VIS}%
\end{figure}

However, if relative velocity is separate on bins of $100$ m/s and plot the excitation potential for lowest energy level against line depth, shifts in the range $(2.5$ to $5.0)$ eV of excitation potential become apparent as shown Figure~\ref{fig:velocity bins energy plot VIS}.

\begin{figure}[H]
    \centering
    \includegraphics[width=1.0\linewidth]{Images/Results/Third signature/Velocity bins energy plot_VIS.pdf}
    \caption{Velocity bins of $100$ m/s across Figure~\ref{fig:lower energy VIS}, with this separation is visible the excitation potential shift in the range $(2.5$ to $5.0)$ eV across line depth.}\label{fig:velocity bins energy plot VIS}%
\end{figure}

Figure~\ref{fig:velocity bins energy VIS} explicitly shows the dependence on the highest values $(2.5$ to $5.0)$ eV for the excitation potential of lower energy levels across line depth, which can be modeled with a lineal fit.

\begin{figure}[H]
    \centering
    \includegraphics[width=0.9\linewidth]{Images/Results/Third signature/Velocity bins energy VIS.pdf}
    \caption{Individual plots for the range $(2.5$ to $5.0)$ eV of excitation potential of lower energy level across line depth, representing each bin of velocity for the visible range of IAG solar flux atlas. The Table~\ref{tab:coef velocity bins energy VIS} shows the coefficients for each applied lineal fit.}\label{fig:velocity bins energy VIS}%
\end{figure}

For the velocity range of $-175$ m/s to $-525$ m/s the calculated slopes are similar, as shown in Table~\ref{tab:coef velocity bins energy VIS}. 
This indicates a consistent ratio on energy shifts in the energy across the velocity range.

\begin{table}[H]
\centering
\begin{tabular}{||r|r|r|r|r||}
\hline
\multicolumn{1}{|c|}{\multirow{2}{*}{\begin{tabular}[c]{@{}c@{}}\textcolor{Miku}{Velocity bin}\\ \textcolor{Miku}{(m/s)}\end{tabular}}} & \multicolumn{1}{c|}{\multirow{2}{*}{\begin{tabular}[c]{@{}c@{}}\textcolor{Miku}{Slop}\\ \textcolor{Miku}{(eV)}\end{tabular}}} & \multicolumn{1}{c|}{\multirow{2}{*}{\textcolor{Miku}{$\sigma_{\text{Slope}}$}}} & \multicolumn{1}{c|}{\multirow{2}{*}{\begin{tabular}[c]{@{}c@{}}\textcolor{Miku}{Shift}\\ \textcolor{Miku}{(eV)}\end{tabular}}} & \multicolumn{1}{c|}{\multirow{2}{*}{\textcolor{Miku}{$\sigma_{\text{Shift}}$}}} \\
\multicolumn{1}{|c|}{} & \multicolumn{1}{c|}{} & \multicolumn{1}{c|}{} & \multicolumn{1}{c|}{} & \multicolumn{1}{c|}{} \\ \hline
0 & -4.5638 & 1.1564 & 7.1636 &  0.9651 \\ \hline
-50 & -2.1781 & 0.8069 & 5.0005 &  0.6495 \\ \hline
-100 & -3.1355 & 0.4958 & 5.7114 &  0.3697 \\ \hline
-150 & -2.3102 & 0.8208 & 5.1098 & 0.5688 \\ \hline
-200 & -2.1203 & 0.5594 & 4.7782 &  0.3747 \\ \hline
-250 & -1.9361 & 0.4121 & 4.6533 &  0.2588 \\ \hline
-300 & -1.6399 & 0.2408 & 4.3146 &  0.1314 \\ \hline
-350 & -1.4534 & 0.2694 & 4.1778 & 0.1251 \\ \hline
-400 & -1.4863 & 0.2683 & 4.2510 &  0.1030 \\ \hline
-450 & -0.9916 & 0.2973 & 3.9734 &  0.0982 \\ \hline
-500 & -1.4429 & 0.4428 & 3.9506 &  0.1097 \\ \hline
-550 & -0.9308 & 0.7750 & 3.9919 & 0.1497 \\ \hline
\end{tabular}
\caption{Values for the lineal fit coefficients on the different velocity bins for the Figure~\ref{fig:velocity bins energy VIS}.
The values $\sigma_{\text{Slope}}$ and $\sigma_{\text{Shift}}$ refers to the standard error of the respective lineal coefficient.}\label{tab:coef velocity bins energy VIS}%
\end{table}

To corroborate the only dependence on excitation potential of lowest energy level, the same analysis was performed on the IAG spatially resolved quiet sun atlas at $\mu=0$.

\begin{figure}[H]
    \centering
    \includegraphics[width=1.0\linewidth]{Images/Results/Third signature/Velocity lower potential_SPA.pdf}
    \caption{Relative velocity against excitation potential of lowest energy level in the visible range for the IAG spatially resolved quiet sun atlas.}\label{fig:lower energy SPA}%
\end{figure}

Figure~\ref{fig:lower energy SPA} shows the relative velocity against excitation potential of lowest energy level for disk center data.

Separating the relative velocity on bins of $100$ m/s and taking the range $(2.5$ to $5.0)$ eV of excitation potential, the disk-center flux exhibits the same qualitative behavior as the IAG solar flux atlas (see Figure~\ref{fig:velocity bins energy SPA}).

\begin{figure}[H]
    \centering
    \includegraphics[width=0.9\linewidth]{Images/Results/Third signature/Velocity bins energy SPA.pdf}
    \caption{Individual plots for the range $(2.5$ to $5.0)$ eV of excitation potential of lowest energy level across line depth, representing each bin of velocity for the IAG spatially resolved quiet sun atlas.}\label{fig:velocity bins energy SPA}%
\end{figure}

However, there is a quantitative difference, the lineal fit coefficients are higher to the reported for the disk-integrated atlas (see Table~\ref{tab:coef velocity bins energy SPA}). 

\begin{table}[H]
    \centering
    \begin{tabular}{||r|r|r|r|r||}
\hline
\multicolumn{1}{|c|}{\multirow{2}{*}{\begin{tabular}[c]{@{}c@{}}Velocity bin\\ (m/s)\end{tabular}}} & \multicolumn{1}{c|}{\multirow{2}{*}{\begin{tabular}[c]{@{}c@{}}Slop\\ (eV)\end{tabular}}} & \multicolumn{1}{c|}{\multirow{2}{*}{$\sigma_{\text{Slope}}$}} & \multicolumn{1}{c|}{\multirow{2}{*}{\begin{tabular}[c]{@{}c@{}}Shift\\ (eV)\end{tabular}}} & \multicolumn{1}{c|}{\multirow{2}{*}{$\sigma_{\text{Shift}}$}} \\
\multicolumn{1}{|c|}{} & \multicolumn{1}{c|}{} & \multicolumn{1}{c|}{} & \multicolumn{1}{c|}{} & \multicolumn{1}{c|}{} \\ \hline
0 & -4.5638 & 1.1564 & 7.1636 &  0.9651 \\ \hline
-50 & -1.9274 & 1.1583 & 4.7903 &  0.9496 \\ \hline
-100 & -2.9149 & 2.0695 & 5.5099 & 1.6744 \\ \hline
-150 & -2.7907 & 1.2421 & 5.4525 & 0.9195 \\ \hline
-200 & -2.8771 & 0.9840 & 5.3471 & 0.7277 \\ \hline
-250 & -2.1774 & 0.7516 & 4.8238 & 0.5241 \\ \hline
-300 & -0.9802 & 0.3714 & 3.8374 & 0.2419 \\ \hline
-350 & -0.3246 & 0.3473 & 3.4399 &  0.1931 \\ \hline
-400 & -0.9455 & 0.3184 & 3.8607 & 0.1511 \\ \hline
-450 & -0.7942 & 0.3032 & 3.7832 &  0.1174 \\ \hline
-500 & -1.1728 & 0.4614 & 3.7206 &  0.1291 \\ \hline
-550 & -1.8835 & 1.0826 & 3.9324 &  0.1770 \\ \hline
\end{tabular}
    \caption{Values for the lineal fit coefficients on the different velocity bins for the Figure~\ref{fig:velocity bins energy SPA}.
The values $\sigma_{\text{Slope}}$ and $\sigma_{\text{Shift}}$ refers to the standard error of the respective lineal coefficient.}\label{tab:coef velocity bins energy SPA}
\end{table}

\textcolor{Miku}{IN PROGRESS..}
The rotation eliminates one of the chromodependence but this dont confirm anything
\textcolor{Miku}{IN PROGRESS..}

\subsection{Characterization of chromodepence on granulation pattern}

The figure~\ref{fig:standard curve gray} show the standard curve proposed for Gray and Oostra~\cite{Gray_Oostra_2018}, stablish on the range $6020-6340\mathring{A}$

\begin{figure}[H]
    \centering
    \includegraphics[width=1.0\linewidth]{Images/Results/Third signature/Standard curve gray.pdf}
    \caption{Standard curve given by Gray and Oostra and recalculated for comparision with the current work.}\label{fig:standard curve gray}
\end{figure}

Since the standard curve does not account for the full wavelength range and the given third-order polynomial fit overestimates the data points, a different approach was adopted. 
The spectrum was separated into distinct color ranges, and a second-order polynomial was fitted to each segment.

\begin{figure}[H]
    \centering
    \includegraphics[width=1.0\linewidth]{Images/Results/Third signature/color curves.pdf}
    \caption{Different color curves fitted to a specific range on granulation pattern for the visible range in the Solar Flux Atlas. The tendency on the curves is more pronounced in the violet and red range.}\label{fig:Curves Solar Flux}
\end{figure}

The trend in the curves is more pronounced in the violet and red ranges. 
This can be interpreted as a shift and flattening of the standard curve from Gray and Oostra towards shorter wavelengths.
Using the curve for the range $4970-5700\mathring{A}$ as a reference (green standard curve), distinct scaling coefficients were calculated for each color segment, as shown in Table~\ref{tab:color coefficients}.

\begin{table}[H]
    \centering
\begin{tabular}{||c|c||}
    \hline
\textbf{Wavelength range $(\mathring{A})$} & \textbf{Color coefficient}  \\ \hline
3800-4270         &    0.5359          \\\hline
4270-4760           &     0.8195        \\\hline
4760-4970         &      0.8348    \\ \hline
4970-5700         &    1.000     \\ \hline
5700-5810         &       1.0595       \\\hline
5810-6180           &        0.9834          \\ \hline
6180-7800            &        1.1149      \\ \hline

\end{tabular}
\caption{Scaling factors $\alpha_c$ to the new standard curve shown in the equation~\eqref{eq:new standard curve}.}\label{tab:color coefficients}
\end{table}

The equation~\eqref{eq:new standard curve} reports a new, color dependent standard curve. 
Its coefficients $(\alpha_c)$ vary with the color range, generating the corresponding granulation pattern for each spectral segment listed in Table~\ref{tab:color coefficients}.

\begin{equation}
    F/F_c = \alpha_c(1.2925\times 10^{-6} v^2 - 0.0019 v +0.8671)	
\label{eq:new standard curve}
\end{equation}

However, when the color coefficients $\alpha_c$ are plotted against wavelength, they exhibit a quadratic tendency that increases with longer wavelengths.
This finding leads us to model the coefficient $\alpha_c$ as a function of wavelength, as shown in equation~\eqref{eq:coefficient trend}

\begin{equation}
    \circledast(\lambda) = -6.7598\times 10^{-8} \lambda^2 + 9.1584\times 10^{-4} \lambda - 2.0041	
\label{eq:coefficient trend}
\end{equation}

Then, the characterization for the granulation pattern it can be described by equation~\ref{eq:THE curve}

\begin{equation}
    F/F_c(\lambda,v) = \circledast(\lambda)[1.2925\times 10^{-6} v^2 - 0.0019 v +0.8671]	
\label{eq:THE curve}
\end{equation}

On the other hand, chromodependence was also identified in the near infrared range, although it is less pronounced than in the visible spectrum.
There were defined specific wavelength ranges by excluding regions dominated by telluric absorption lines, and was found linear relationships for each range (see Figure~\ref{fig:Curves nir Solar Flux}).
However, unlike the visible range these relationships do not exhibit a unified correlation.

\begin{figure}[H]
    \centering
    \includegraphics[width=1.0\linewidth]{Images/Results/Third signature/color curves nir.pdf}
    \caption{Different color curves fitted to a specific ranges on granulation pattern for the near infrared range in the Solar Flux Atlas.}\label{fig:Curves nir Solar Flux}
\end{figure}

The coefficients for each linear fit applied are shown in the table~\ref{tab:color coefficients nir}.

\begin{table}[H]
    \centering
\begin{tabular}{||c|c|c||}
    \hline
\textbf{Wavelength range} &\textbf{Slope} & \textbf{Shift}  \\ \hline
7000-8500         &    -0.0010     &   0.9279 \\\hline
8500-11000           &     -0.0009   &  0.9496   \\\hline
11000-16000         &      -0.0003  &  0.7546 \\ \hline
16000-23000         &    -0.0001  &  0.7633 \\ \hline
\end{tabular}
\caption{Coefficients for the first order polynomial fit in each wavelength range in the near infrared.}\label{tab:color coefficients nir}
\end{table}



\section{Higher quality graphs}

To show the improvement on the quality in the third signature plots, the figure~\ref{fig:Scattered points} compare the plot obtained by Ellwarth~\cite{Ellwarth_2023} and the granulation pattern obtained in this project.

\begin{figure}[H]
     \centering
     \begin{subfigure}{0.47\textwidth}
         \includegraphics[width=\textwidth]{Images/Results/Third signature/Standard curve_Ellwarth.pdf}

     \end{subfigure}
\hfill
     \begin{subfigure}{0.47\textwidth}

         \includegraphics[width=\textwidth]{Images/Granulation pattern Ellwarth.jpg}
     \end{subfigure}

        \caption{We realized the same graphic for the comparision with the Ellwarth article to show the less scattered points.}\label{fig:Scattered points}
\end{figure}

The less scattered points and the improvement on the chromodepence identification is evident.

Furthermore, the variance of the observed wavelength was analyzed by altering the number of points used in the fourth-order polynomial fit. 
Figure~\ref{fig:variance} shows that this parameter set performs well in the visible spectrum. H
However, its performance degrades in the infrared, where the number of data points defining the line core is reduced compared to a typical line in the visible range.

\begin{figure}[H]
     \centering
     \begin{subfigure}{1.0\textwidth}
         \includegraphics[width=\textwidth]{Images/Results/variance VIS.pdf}
         \caption{Variance for random lines in the visible range for the Solar Flux Atlas.}

     \end{subfigure}
\hfill
     \begin{subfigure}{1.0\textwidth}

         \includegraphics[width=\textwidth]{Images/Results/variance NIR.pdf}
         \caption{Variance for random lines in the near infrared range for the Solar Flux Atlas.}

     \end{subfigure}

        \caption{Variance of the observed wavelength altering the number of points on the fourth order polynomial fit}\label{fig:variance}
\end{figure}


