
Our results were separated into the three signatures of convection around the main phenomenon of chromodependence.

\section{The first signature: Line broadening}

\subsection{Line depth-dependence on line core curvature}
The core curvature was calculated following the equation~\eqref{eq:core curvature} and plotted against line depth as shown in figure~\ref{fig:curvatures solar flux}.

\begin{figure}[H]
     \centering
     \begin{subfigure}{0.85\textwidth}
         \includegraphics[width=\textwidth]{Images/Results/First signature/Curvature_VIS.pdf}
         \caption{Line core curvature for the visible range in the Solar Flux Atlas. Is visible a characteristic curve with line depth-dependence along wavelength.}\label{fig:curvature VIS}
     \end{subfigure}
\hfill
     \begin{subfigure}{0.85\textwidth}
         \includegraphics[width=\textwidth]{Images/Results/First signature/Curvature_NIR.pdf}
         \caption{Line core curvature for the near infrared range in the Solar Flux Atlas. Is visible a natural division for wavelengths in $11400 \mathring{A}$.}\label{fig:curvature NIR}%
     \end{subfigure}
     \caption{Line core curvature for the Solar Flux Atlas separated in wavelength ranges.}\label{fig:curvatures solar flux}%
\end{figure}

As shown in Figure~\ref{fig:curvature NIR}, the near infrared range exhibits a natural division at approximately $11400 \mathring{A}$, a region dominated by telluric absorption lines from Earth's atmosphere. 
When plotting the entire spectral range of the Solar Flux Atlas, a clear line depth-dependence of the convective blueshift as a function of wavelength is observed.

\begin{figure}[H]
    \centering
    \includegraphics[width=1.0\linewidth]{Images/Results/First signature/Curvature_ALL.pdf}
    \caption{Line core curvature in the Solar Flux Atlas. The near infrared range presents a natural division due telluric lines in the atmosphere.}\label{fig:all curvature solar flux}%
\end{figure}

Despite the line depth-dependent shifts along wavelength, the visible range of the atlas exhibits a clear linear tendency (see figure~\ref{fig:all curvature solar flux}).
Consequently, a first order polynomial fit was applied to the data in the range $(0.0-0.1)F/F_c$ of line depth of the Solar Flux Atlas, where $F/F_c$ represents the normalized flux. 

\begin{figure}[H]
    \centering
    \includegraphics[width=1.0\linewidth]{Images/Results/First signature/Curvature_Slope_VIS.pdf}
    \caption{Linear polynomial fit applied to the range $(0.0-0.1)F/F_c$ of line depth in the visible range of the Solar Flux Atlas.}\label{fig:curvature slope VIS}
\end{figure}

As shown in Figure~\ref{fig:curvature slope VIS}, the slope of the relationship has a value of $1.8739\times 10^{10}$.
The absence of chromodependence in this spectral range indicates that line core curvatures have a net dependence on hydrodynamic velocities, with negligible influence from atomic effects.
To confirm this statement,  the line core curvature was also plotted for the visible range using the Spatially Resolved Quiet Sun Atlas at the solar limb $\mu=0$,as shown figure~\ref{fig:curvature SPA}.
In other words, this analysis uses disk-integrated flux from the center of the solar disk, which is not affected by rotational Doppler broadening, thereby isolating the external effects.

\begin{figure}[H]
    \centering
    \includegraphics[width=1.0\linewidth]{Images/Results/First signature/Curvature_SPA.pdf}
    \caption{Line core curvature for the visible range for disk center data from the Spatially Resolved Quiet Sun Atlas.}\label{fig:curvature SPA}%
\end{figure}

A first order polynomial fit in figure~\ref{fig:curvature slope SPA} shows a value of $2.9360\times 10^{10}$ for the slope.

\begin{figure}[H]
    \centering
    \includegraphics[width=1.0\linewidth]{Images/Results/First signature/Curvature_Slope_SPA.pdf}
    \caption{First order polynomial fit applied to the range $(0.0-0.1)F/F_c$ of line depth in the visible range of the Spatially Resolved Quiet Sun Atlas.}\label{fig:curvature slope SPA}%
\end{figure}

This result confirms the hypothesis that weaker lines, which have smaller line core curvature, are systematically displaced across the line depth due to rotational Doppler effects. 
These weaker lines correspond to transitions with lower excitation energies. 
Consequently, the relationship between line depth and lower excitation potential was analysed.

\subsection{Line depth-dependence on lower excitation energy}

The Nave list for Fe I lines~\cite{Nave_1994} include the values for the highest and the lowest excitation energy.
As shown in the figure~\ref{fig:lower energy ALL} the relative velocity was plotted against lower excitation energy, where it is not clear a strong shift on the velocity. 

\begin{figure}[H]
    \centering
    \includegraphics[width=1.0\linewidth]{Images/Results/First signature/Velocity lower potential_ALL.pdf}
    \caption{Relative velocity against lower excitation energy for all the range in the Solar Flux Atlas}\label{fig:lower energy ALL}%
\end{figure}

However, if there separate the relative velocity on bins of $100m/s$ around a fixed velocity and plot lower excitation energy against line depth, shifts in the excitation energy become apparent as shown figure~\ref{fig:velocity bins energy plot ALL}.

\begin{figure}[H]
    \centering
    \includegraphics[width=1.0\linewidth]{Images/Results/First signature/Velocity bins energy plot_ALL.pdf}
    \caption{Velocity bins of $100m/s$ around a fixed velocity, across the figure~\ref{fig:lower energy ALL}, with this separation is visible the energy shift across line depth.}\label{fig:velocity bins energy plot ALL}%
\end{figure}

The figure~\ref{fig:velocity bins energy ALL} explicitly shows the dependency on the highest values for the lowest excitation energy across the wavelength, which can be fitted as a first order polynomial fit.

\begin{figure}[H]
    \centering
    \includegraphics[width=1.0\linewidth]{Images/Results/First signature/Velocity bins energy ALL.pdf}
    \caption{Individual plots of lower excitation energy across line depth representing each bin of velocity for the Solar Flux Atlas.}\label{fig:velocity bins energy ALL}%
\end{figure}

For the velocity range of $-175m/s$ to $-525m/s$ the calculated slopes are similar, as shown in table~\ref{tab:coef velocity bins energy ALL}. 
This indicates a consistent ratio on energy shifts in the energy across the velocity range.

\begin{table}[H]
\centering
\begin{tabular}{||c|c|c|c|c||}
\hline
\textbf{Velocity bin} & \textbf{Shift} & \textbf{Slope} & \textbf{Error slope} & \textbf{Error shift} \\ \hline
0 & 7.1636 & -4.5638 & 1.1564 & 0.9651 \\ \hline
-50 & 5.0005 & -2.1781 & 0.8069 & 0.6495 \\ \hline
-100 & 5.7114 & -3.1355 & 0.4958 & 0.3697 \\ \hline
-150 & 5.1098 & -2.3102 & 0.8208 & 0.5688 \\ \hline
-200 & 4.7782 & -2.1203 & 0.5594 & 0.3747 \\ \hline
-250 & 4.6533 & -1.9361 & 0.4121 & 0.2588 \\ \hline
-300 & 4.3146 & -1.6399 & 0.2408 & 0.1314 \\ \hline
-350 & 4.1778 & -1.4534 & 0.2694 & 0.1251 \\ \hline
-400 & 4.2510 & -1.4863 & 0.2683 & 0.1030 \\ \hline
-450 & 3.9734 & -0.9916 & 0.2973 & 0.0982 \\ \hline
-500 & 3.9506 & -1.4429 & 0.4428 & 0.1097 \\ \hline
-550 & 3.9919 & -0.9308 & 0.7750 & 0.1497 \\ \hline
\end{tabular}
\caption{Values for the first order polynomial coefficients on the different velocity bins for the Solar Flux Atlas.}\label{tab:coef velocity bins energy ALL}%
\end{table}

To corroborate that dependence is only on lower excitation energy, the same analysis was performed on the Spatially Resolved Quiet Sun Atlas at the solar limb $\mu=0$.
The plot of relative velocity against lower excitation energy for this atlas exhibits the same qualitative behavior as the Solar Flux Atlas.
However, there is a quantitative difference, the first order polynomial coefficients are higher (see figure~\ref{fig:velocity bins energy SPA} and table~\ref{tab:coef velocity bins energy SPA}). 

\begin{figure}[H]
    \centering
    \includegraphics[width=1.0\linewidth]{Images/Results/First signature/Velocity bins energy SPA.pdf}
    \caption{Individual plots of lower excitation energy across line depth representing each bin of velocity for the Spatially Resolved Quiet Sun Atlas.}\label{fig:velocity bins energy SPA}%
\end{figure}

\begin{table}
    \centering
    \begin{tabular}{||c|c|c|c|c||}
\hline
\textbf{Velocity bin} & \textbf{Shift} & \textbf{Slope} & \textbf{Error slope} & \textbf{Error shift} \\ \hline
0 & 7.1636 & -4.5638 & 1.1564 & 0.9651 \\ \hline
-50 & 4.7903 & -1.9274 & 1.1583 & 0.9496 \\ \hline
-100 & 5.5099 & -2.9149 & 2.0695 & 1.6744 \\ \hline
-150 & 5.4525 & -2.7907 & 1.2421 & 0.9195 \\ \hline
-200 & 5.3471 & -2.8771 & 0.9840 & 0.7277 \\ \hline
-250 & 4.8238 & -2.1774 & 0.7516 & 0.5241 \\ \hline
-300 & 3.8374 & -0.9802 & 0.3714 & 0.2419 \\ \hline
-350 & 3.4399 & -0.3246 & 0.3473 & 0.1931 \\ \hline
-400 & 3.8607 & -0.9455 & 0.3184 & 0.1511 \\ \hline
-450 & 3.7832 & -0.7942 & 0.3032 & 0.1174 \\ \hline
-500 & 3.7206 & -1.1728 & 0.4614 & 0.1291 \\ \hline
-550 & 3.9324 & -1.8835 & 1.0826 & 0.1770 \\ \hline
\end{tabular}
    \caption{Values for the first order polynomial coefficients on the different velocity bins for the Spatially Resolved Quiet Sun Atlas.}\label{tab:coef velocity bins energy SPA}
\end{table}

\section{The second signature: Line profile bisector asymmetry}

\subsection{The bisector slope}

The line core bisector slope was calculated following the equation~\eqref{eq:third derivative relation} and plotted against line depth, as shown the figure~\ref{fig:bisector VIS}.

\begin{figure}[H]
    \centering
    \includegraphics[width=1.0\linewidth]{Images/Results/Second signature/Bisector_VIS.pdf}
    \caption{Line profile bisector slopes for the visible range in the Solar Flux Atlas. The behavior of the plot is according to the C-curved shape of the line bisector affected by convection movement.}\label{fig:bisector VIS}%
\end{figure}

The weakest lines, due to their small depth, do not exhibit a fully developed C-shaped bisector. 
Instead, their bisectors shows as a redshifted slope. 
In contrast, lines formed deeper in the photosphere experience the convective blueshift in its totality, resulting in a negative bisector slope.
he line depths in the middle range represent an equilibrium point where the upward motion of the convective cell is overcome by gravitational forces, leading to overshoot and subsequent descent. 
Then a first-order polynomial fit was applied to the bisector slope data in the line depth range of $(0.3-0.6)F/F_c$ to quantify this transition, as shown in Figure~\ref{fig:bisector slope VIS}.

\begin{figure}[H]
    \centering
    \includegraphics[width=1.0\linewidth]{Images/Results/Second signature/Bisector slope_VIS.pdf}
    \caption{First order polynomial fit adjusted to the range $(0.3-0.6)F/F_c$ for the line bisector slopes.}\label{fig:bisector slope VIS}%
\end{figure}

For a point of comparision, the same analysis was performed to the center limb spectra (see figure~\ref{fig:bisector SPA}).

\begin{figure}[H]
    \centering
    \includegraphics[width=1.0\linewidth]{Images/Results/Second signature/Bisector_SPA.pdf}
    \caption{Line profile bisector slope for the visible range in the Spatially Resolved Quiet Sun Atlas.}\label{fig:bisector SPA}
\end{figure}

The bisector slope derived from the disk center is smaller in magnitude compared to the integrated flux, despite both exhibiting the same characteristic trend (see Figure~\ref{fig:bisector slope SPA}).

\begin{figure}[H]
    \centering
    \includegraphics[width=1.0\linewidth]{Images/Results/Second signature/Bisector slope_SPA.pdf}
    \caption{First order polynomial fit adjusted to the range $(0.3-0.6)F/F_c$ for the line bisector slopes.}\label{fig:bisector slope SPA}
\end{figure}
 
The phenomenon of chromodependence is presented on the figures~\ref{fig:bisector slope VIS} and~\ref{fig:bisector slope SPA}, where the line profile bisectors slopes are organised by decreasing wavelength.
This behavior is clearer on the disc center spectra than the integrated flux.

However, if there separate the line depth on bins of $0.1$ around a fixed depth and plot the line profile bisector slope against wavelength, shifts in the line bisector slope become apparent as shown figure~\ref{fig:velocity bins bisector plot ALL}.

\begin{figure}[H]
    \centering
    \includegraphics[width=1.0\linewidth]{Images/Results/Second signature/Velocity bins bisector plot_ALL.pdf}
    \caption{Line depth bins of $0.1$ around a fixed depth, with this separation is visible the line bisector slope shift across wavelength.}\label{fig:velocity bins bisector plot ALL}%
\end{figure}

For the integrated flux spectra was found negative slopes with a wavelength dependence (see figure~\ref{fig:velocity bins bisector ALL}).

\begin{figure}[H]
    \centering
    \includegraphics[width=1.0\linewidth]{Images/Results/Second signature/Velocity bins bisector ALL.pdf}
    \caption{Individual plots of line bisector slope shift across wavelength representing each bin of line depth for the Solar Flux Atlas.}\label{fig:velocity bins bisector ALL}%
\end{figure}

The same analysis was performed on the disc center flux and was found the same behavior (see figure~\ref{fig:velocity bins bisector SPA})

\begin{figure}[H]
    \centering
    \includegraphics[width=1.0\linewidth]{Images/Results/Second signature/Velocity bins bisector SPA.pdf}
    \caption{Individual plots of line bisector slope shift across wavelength representing each bin of line depth for the Spatially Resolved Quiet Sun Atlas.}\label{fig:velocity bins bisector SPA}%
\end{figure}

\subsection{Flux deficit (in progress)}

Due to radiation, the redshift part of a line is displced on the flux, inducing a rotation on the c-curved profile bisector.
This phenomenon is called flux deficit.
Hamilton and Lester~\cite{Hamilton_1999} noticed that the behavior of the third signature mimics the bisectors
gives the sight of the mean bisectors following the granulation pattern behavior, this last is discussed in the next section. 
Later, Gray and Oostra~\cite{Gray_Oostra_2018} show that the bisectors need to follow the granulation pattern as the form of we calculated velocities and bisectors.

For comparision with Gray and Oostra work, was taken the line $6254.2850\mathring{A}$ as shown in figure~\ref{fig:bisector flux deficit}.

\begin{figure}[H]
    \centering
    \includegraphics[width=1.0\linewidth]{Images/Results/Second signature/Flux deficit_6253.pdf}
    \caption{C-curved line profile bisector for the $6254.2850\mathring{A}$ and the standard curve for the green range from the third signature plot.}\label{fig:bisector flux deficit}
\end{figure}

Following the same method, the figure~\ref{fig:flux deficit} shows a flux imbalance described on the distribution and temperatures of the mean, the maximum and the RMS point of the bisector.

\begin{figure}[H]
    \centering
    \includegraphics[width=1.0\linewidth]{Images/Results/Second signature/Flux deficit curve_6253.pdf}
    \caption{Flux deficit curve for the $6254.2850\mathring{A}$ and the temperatures of the mean, the maximum and the RMS point of the bisector.}\label{fig:flux deficit}
\end{figure}

The idea is add the flux deficit using the color curves and just report the velocities with the percent of the line corresponds that velocity.
As we found the standard curves for all the ranges, the same analysis was realized for a random line in the range of the respective standard curve (im working on that).

\section{The third signature: Line depth-dependence on wavelength shifts}

\subsection{The granulation pattern}

The granulation patterns for the IAG Solar Flux Atlas in all the wavelength range was obtained.

\begin{figure}[H]
    \centering
 \includegraphics[width=1.0\linewidth]{Images/Results/Third signature/GranulationPattern_ALL.pdf}
    \caption{Granulation pattern obtained for the Solar Flux Atlas. The wavelength shift dependence is along the line depth.}\label{fig:Granulation solar flux}
\end{figure}

As shown in Figure~\ref{fig:Granulation solar flux}, the behavior along line depth is according to literature, clearly demonstrating the line depth-dependent wavelength shift, also known as chromodependence.
To further characterize this trend, an analysis of line depth versus wavelength was performed. The guiding hypothesis was: “If the velocity shift is present only in the disk-integrated Solar Flux Atlas and absent at the disk center, then solar rotation is the cause.”
However, the velocity shift was observed in both spectral datasets. 
This was initially unexpected, as rotational doppler broadening is negligible at the disk center.

To measure these wavelength shifts, the analysis was restricted to the spectral range from $4300 \mathring{A}$ to $5600 \mathring{A}$. 
The wavelengths from both atlases were then sorted into velocity bins of $50m/s$ (see figure~\ref{fig:Velocity bins}).

\begin{figure}[H]
     \centering
     \begin{subfigure}{1.0\textwidth}
         \includegraphics[width=\textwidth]{Images/Results/Third signature/Velocity bins VIS_plot.pdf}
         \caption{Visible range for the Solar Flux Atlas.}\label{fig:vel bins plot VIS}
     \end{subfigure}
\hfill
     \begin{subfigure}{1.0\textwidth}
         \includegraphics[width=\textwidth]{Images/Results/Third signature/Velocity bins SPA_plot.pdf}
         \caption{Visible range for the Spatially Resolved Quiet Sun Atlas.}\label{fig:vel bins plot SPA}
     \end{subfigure}

        \caption{Comparision between atlases with velocity bins for the relation between wavelength and line depth. For each velocity bin was plotted wavelength against line depth to measure a frequency shift with a first order polynomial fit.}\label{fig:Velocity bins}
\end{figure}

For each velocity bin was plotted wavelength against line depth to measure a frequency shift with a first order polynomial fit (see figure~\ref{fig:Velocity bins slopes})

\begin{figure}[H]
     \centering
     \begin{subfigure}{1.0\textwidth}
         \includegraphics[width=\textwidth]{Images/Results/Third signature/Velocity bins VIS.pdf}
         \caption{Velocity bins of $50m/s$ for the Solar Flux Atlas.}\label{fig:vel bins VIS}
     \end{subfigure}
\hfill
     \begin{subfigure}{1.0\textwidth}
         \includegraphics[width=\textwidth]{Images/Results/Third signature/Velocity bins SPA.pdf}
         \caption{Velocity bins of $50m/s$ for the Spatially Resolved Quiet Sun Atlas.}\label{fig:vel bins SPA}
     \end{subfigure}

        \caption{Individual plots of lower excitation energy across line depth representing each bin of velocity.}\label{fig:Velocity bins slopes}
\end{figure}

The tables~\ref{tab:velocity bins SPA} and~\ref{tab:velocity bins VIS} shows the obtained slopes in both atlases. 

\begin{table}[H]
    \centering
\begin{tabular}{||c|c|c|c|c||}
\hline
\multicolumn{1}{|c|}{\textbf{Velocity bin}} & \multicolumn{1}{c|}{\textbf{Shift}} & \multicolumn{1}{c|}{\textbf{Slope}} & \multicolumn{1}{c|}{\textbf{Error slope}} & \multicolumn{1}{c|}{\textbf{Error shift}} \\ \hline
50 & 1.1937 & -0.000063 & 0.000022 & 0.1166 \\ \hline
0 & 1.1720 & -0.000056 & 0.000018 & 0.0923 \\ \hline
-50 & 1.2840 & -0.000087 & 0.000021 & 0.1087 \\ \hline
-100 & 1.3953 & -0.000112 & 0.000012 & 0.0584 \\ \hline
-150 & 1.4327 & -0.000124 & 0.000017 & 0.0842 \\ \hline
-200 & 1.5028 & -0.000142 & 0.000013 & 0.0658 \\ \hline
-250 & 1.5712 & -0.000165 & 0.000018 & 0.0890 \\ \hline
-300 & 1.6233 & -0.000184 & 0.000019 & 0.0971 \\ \hline
-350 & 1.4129 & -0.000154 & 0.000055 & 0.2583 \\ \hline
-400 & 1.6455 & -0.000208 & 0.000033 & 0.1617 \\ \hline
-450 & 1.1952 & -0.000144 & 0.000060 & 0.3094 \\ \hline
\end{tabular}
\caption{Slopes for the first order polynomial fit in each velocity bin for the Spatially Resolved Quiet Sun Atlas.}
\label{tab:velocity bins SPA}
\end{table}

\begin{table}[H]
    \centering
\begin{tabular}{||c|c|c|c|c||}
\hline
\multicolumn{1}{|c|}{\textbf{Velocity bin}} & \multicolumn{1}{c|}{\textbf{Shift}} & \multicolumn{1}{c|}{\textbf{Slope}} & \multicolumn{1}{c|}{\textbf{Error slope}} & \multicolumn{1}{c|}{\textbf{Error shift}} \\ \hline
-100 & 1.2704 & -0.000091 & 0.000015 & 0.0797 \\ \hline
-150 & 1.2601 & -0.000094 & 0.000013 & 0.0673 \\ \hline
-200 & 1.3340 & -0.000115 & 0.000019 & 0.0985 \\ \hline
-250 & 1.3920 & -0.000133 & 0.000023 & 0.1164 \\ \hline
-300 & 1.3724 & -0.000139 & 0.000017 & 0.0874 \\ \hline
-350 & 1.3106 & -0.000137 & 0.000015 & 0.0744 \\ \hline
-400 & 1.2119 & -0.000128 & 0.000018 & 0.0866 \\ \hline
-450 & 1.4212 & -0.000181 & 0.000039 & 0.1947 \\ \hline
-500 & 1.3787 & -0.000189 & 0.000052 & 0.2609 \\ \hline
\end{tabular}
\caption{Slopes for the first order polynomial fit in each velocity bin for the Solar Flux Atlas.}
\label{tab:velocity bins VIS}
\end{table}

The value of slopes for the Spatially Resolved Quiet Sun Atlas are greater than the Solar Flux Atlas, which contradicts the initial hypothesis.

\subsection{Characterization of chromodepence on granulation pattern}

The figure~\ref{fig:standard curve gray} show the standard curve proposed for Gray and Oostra~\cite{Gray_Oostra_2018}, stablish on the range $6020-6340\mathring{A}$

\begin{figure}[H]
    \centering
    \includegraphics[width=1.0\linewidth]{Images/Results/Third signature/Standard curve gray.pdf}
    \caption{Standard curve given by Gray and Oostra and recalculated for comparision with the current work.}\label{fig:standard curve gray}
\end{figure}

Since the standard curve does not account for the full wavelength range and the given third-order polynomial fit overestimates the data points, a different approach was adopted. 
The spectrum was separated into distinct color ranges, and a second-order polynomial was fitted to each segment.

\begin{figure}[H]
    \centering
    \includegraphics[width=1.0\linewidth]{Images/Results/Third signature/color curves.pdf}
    \caption{Different color curves fitted to a specific range on granulation pattern for the visible range in the Solar Flux Atlas. The tendency on the curves is more pronounced in the violet and red range.}\label{fig:Curves Solar Flux}
\end{figure}

The trend in the curves is more pronounced in the violet and red ranges. 
This can be interpreted as a shift and flattening of the standard curve from Gray and Oostra towards shorter wavelengths.
Using the curve for the range $4970-5700\mathring{A}$ as a reference (green standard curve), distinct scaling coefficients were calculated for each color segment, as shown in Table~\ref{tab:color coefficients}.

\begin{table}[H]
    \centering
\begin{tabular}{||c|c||}
    \hline
\textbf{Wavelength range $(\mathring{A})$} & \textbf{Color coefficient}  \\ \hline
3800-4270         &    0.5359          \\\hline
4270-4760           &     0.8195        \\\hline
4760-4970         &      0.8348    \\ \hline
4970-5700         &    1.000     \\ \hline
5700-5810         &       1.0595       \\\hline
5810-6180           &        0.9834          \\ \hline
6180-7800            &        1.1149      \\ \hline

\end{tabular}
\caption{Scaling factors $\alpha_c$ to the new standard curve shown in the equation~\eqref{eq:new standard curve}.}\label{tab:color coefficients}
\end{table}

The equation~\eqref{eq:new standard curve} reports a new, color dependent standard curve. 
Its coefficients $(\alpha_c)$ vary with the color range, generating the corresponding granulation pattern for each spectral segment listed in Table~\ref{tab:color coefficients}.

\begin{equation}
    F/F_c = \alpha_c(1.2925\times 10^{-6} v^2 - 0.0019 v +0.8671)	
\label{eq:new standard curve}
\end{equation}

However, when the color coefficients $\alpha_c$ are plotted against wavelength, they exhibit a quadratic tendency that increases with longer wavelengths.
This finding leads us to model the coefficient $\alpha_c$ as a function of wavelength, as shown in equation~\eqref{eq:coefficient trend}

\begin{equation}
    \circledast(\lambda) = -6.7598\times 10^{-8} \lambda^2 + 9.1584\times 10^{-4} \lambda - 2.0041	
\label{eq:coefficient trend}
\end{equation}

Then, the characterization for the granulation pattern it can be described by equation~\ref{eq:THE curve}

\begin{equation}
    F/F_c(\lambda,v) = \circledast(\lambda)[1.2925\times 10^{-6} v^2 - 0.0019 v +0.8671]	
\label{eq:THE curve}
\end{equation}

On the other hand, chromodependence was also identified in the near infrared range, although it is less pronounced than in the visible spectrum.
There were defined specific wavelength ranges by excluding regions dominated by telluric absorption lines, and was found linear relationships for each range (see Figure~\ref{fig:Curves nir Solar Flux}).
However, unlike the visible range these relationships do not exhibit a unified correlation.

\begin{figure}[H]
    \centering
    \includegraphics[width=1.0\linewidth]{Images/Results/Third signature/color curves nir.pdf}
    \caption{Different color curves fitted to a specific ranges on granulation pattern for the near infrared range in the Solar Flux Atlas.}\label{fig:Curves nir Solar Flux}
\end{figure}

The coefficients for each linear fit applied are shown in the table~\ref{tab:color coefficients nir}.

\begin{table}[H]
    \centering
\begin{tabular}{||c|c|c||}
    \hline
\textbf{Wavelength range} &\textbf{Slope} & \textbf{Shift}  \\ \hline
7000-8500         &    -0.0010     &   0.9279 \\\hline
8500-11000           &     -0.0009   &  0.9496   \\\hline
11000-16000         &      -0.0003  &  0.7546 \\ \hline
16000-23000         &    -0.0001  &  0.7633 \\ \hline
\end{tabular}
\caption{Coefficients for the first order polynomial fit in each wavelength range in the near infrared.}\label{tab:color coefficients nir}
\end{table}

\section{Higher quality graphs}

To show the improvement on the quality in the third signature plots, the figure~\ref{fig:Scattered points} compare the plot obtained by Ellwarth~\cite{Ellwarth_2023} and the granulation pattern obtained in this project.

\begin{figure}[H]
     \centering
     \begin{subfigure}{0.47\textwidth}
         \includegraphics[width=\textwidth]{Images/Results/Third signature/Standard curve_Ellwarth.pdf}

     \end{subfigure}
\hfill
     \begin{subfigure}{0.47\textwidth}

         \includegraphics[width=\textwidth]{Images/Granulation pattern Ellwarth.jpg}
     \end{subfigure}

        \caption{We realized the same graphic for the comparision with the Ellwarth article to show the less scattered points.}\label{fig:Scattered points}
\end{figure}

The less scattered points and the improvement on the chromodepence identification is evident.

Furthermore, the variance of the observed wavelength was analyzed by altering the number of points used in the fourth-order polynomial fit. 
Figure~\ref{fig:variance} shows that this parameter set performs well in the visible spectrum. H
However, its performance degrades in the infrared, where the number of data points defining the line core is reduced compared to a typical line in the visible range.

\begin{figure}[H]
     \centering
     \begin{subfigure}{1.0\textwidth}
         \includegraphics[width=\textwidth]{Images/Results/variance VIS.pdf}
         \caption{Variance for random lines in the visible range for the Solar Flux Atlas.}

     \end{subfigure}
\hfill
     \begin{subfigure}{1.0\textwidth}

         \includegraphics[width=\textwidth]{Images/Results/variance NIR.pdf}
         \caption{Variance for random lines in the near infrared range for the Solar Flux Atlas.}

     \end{subfigure}

        \caption{Variance of the observed wavelength altering the number of points on the fourth order polynomial fit}\label{fig:variance}
\end{figure}


