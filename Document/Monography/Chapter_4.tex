Our results were separated into the three signatures of convection around the main phenomenon of chromodependence.

\section{The first signature: Line broadening}

\subsection{Line depth-dependence on line core curvature}
The core curvature was calculated following the equation \eqref{core curvature} and plotted against line depth as shown in figure \ref{curvatures solar flux}.

\begin{figure}[H]
     \centering
     \begin{subfigure}{1.1\textwidth}
         \includegraphics[width=\textwidth]{Images/Results/Curvature_VIS.pdf}
         \caption{Line core curvature for the visible range in the Solar Flux Atlas. Is visible a characteristic curve with line depth-dependence along wavelength.}

         \label{curvature VIS}
     \end{subfigure}
\hfill
     \begin{subfigure}{1.1\textwidth}
         \includegraphics[width=\textwidth]{Images/Results/Curvature_NIR.pdf}
         \caption{Line core curvature for the near infrared range in the Solar Flux Atlas. Is visible a natural division for wavelengths in $11400 \mathring{A}$.}
         \label{curvature NIR}
     \end{subfigure}

        \label{curvatures solar flux}
\end{figure}

As shown in the figure \ref{curvature NIR} for the near infrared range, is visible as a natural division for wavelengths in $11400 \mathring{A}$, which corresponds to telluric lines of absorption in the atmosphere. 
Plotting all ranges of the Solar Flux Atlas is visible a line depth-dependence along wavelength.

\begin{figure}[H]
    \centering
    \includegraphics[width=1.1\linewidth]{Images/Results/Sharpness_ALL.pdf}
    \caption{Line core curvature in the Solar Flux Atlas. The near infrared range presents a natural division due telluric lines in the atmosphere.}
    \label{all curvature solar flux}
\end{figure}

Despite the line depth-dependence shifts along wavelength is clear a linear tendency on the visible part of the atlas (see figure \ref{all curvature solar flux}).
Then, a first order polynomial fit was fitted to the range $0.0-0.1F/F_c$ of line depth in the visible range of the Solar Flux Atlas where $F/F_c$ represents the normalised flux. 

\begin{figure}[H]
    \centering
    \includegraphics[width=1.1\linewidth]{Images/Results/Sharpness_ALL.pdf}
    \caption{First order polynomial fit fitted to the range $0.0-0.1F/F_c$ of line depth in the visible range of the Solar Flux Atlas.}
    \label{curvature slope VIS}
\end{figure}

As can be seen in figure \ref{curvature slope VIS}, the slope has a value of $1.8739\times 10^{10}$.
Due to the fact of non-chromodependence in this range, can be concluded that line core curvatures have a net dependence of the velocities and a non-dependence caused by atomic effects.
To confirm this statement, was plotted the line core curvature for the visible range in the Spatially Resolved Quiet Sun Atlas with $\mu=0$ (see figure \ref{curvature SPA}).
In other words, was plotted the same behavior for the disk flux of the Sun center, this type of data don't present the effect of rotation. 

\begin{figure}[H]
    \centering
    \includegraphics[width=1.1\linewidth]{Images/Results/Sharpness_ALL.pdf}
    \caption{Line core curvature for the visible range for disk center data from the Spatially Resolved Quiet Sun Atlas.}
    \label{curvature SPA}
\end{figure}

As shown in figure \ref{curvature slope SPA} calculating the first order polynomial fit was found the value of $2.9360\times 10^{10}$ for the slope.

\begin{figure}[H]
    \centering
    \includegraphics[width=1.1\linewidth]{Images/Results/Sharpness_ALL.pdf}
    \caption{First order polynomial fit fitted to the range $0.0-0.1F/F_c$ of line depth in the visible range of the Spatially Resolved Quiet Sun Atlas.}
    \label{curvature slope SPA}
\end{figure}

This confirms the hypothesis that smallest line core curvature, or weaker lines, are displaced across the line depth as consequence of rotational Doppler effects.
These weaker lines are known to represent the lower energy excitations. 
In consequence, the dependence of line depth with lower excitation energy was studied.

\subsection{Line depth-dependence on lower excitation energy}

The Nave list for Fe I lines \cite{Nave_1994} include the values for the highest and the lowest excitation energy.
As shown in the figure \ref{lower energy ALL} the relative velocity was plotted against lower excitation energy, where it is not clear a strong shift on the velocity. 

\begin{figure}[H]
    \centering
    \includegraphics[width=1.1\linewidth]{Images/Results/Sharpness_ALL.pdf}
    \caption{Relative velocity against lower excitation energy for all the range in the Solar Flux Atlas}
    \label{lower energy ALL}
\end{figure}

However, when the relative velocity are separated on bins of $50m/s$ and plot the lower excitation energy against line depth it can be seen shifts in the energy as shown in figure \ref{velocity bins energy plot ALL}.

\begin{figure}[H]
    \centering
    \includegraphics[width=1.1\linewidth]{Images/Results/Sharpness_ALL.pdf}
    \caption{Velocity bins of $50m/s$ across the figure \ref{lower energy ALL}, with this separation is visible the energy shift across line depth.}
    \label{velocity bins energy plot ALL}
\end{figure}

The figure \ref{velocity bins energy ALL} explicitly shows the dependency on the highest values for the lowest excitation energy across the wavelength, which can be fitted as a first order polynomial fit.

\begin{figure}[H]
    \centering
    \includegraphics[width=1.1\linewidth]{Images/Results/Sharpness_ALL.pdf}
    \caption{Individual plots of lower excitation energy across line depth representing each bin of velocity for the Solar Flux Atlas.}
    \label{velocity bins energy ALL}
\end{figure}

For the range of velocity $-175m/s$ to $-525m/s$ the values for slopes are similar as shown in table \ref{coef velocity bins energy ALL}, whose indicates the same ratio of shifts in the energy.

\begin{table}[H]
\begin{tabular}{ccccc}
\hline
\textbf{Velocity bin} & \textbf{Shift} & \textbf{Slope} & \textbf{Error slope} & \textbf{Error shift} \\ \hline
0 & 7.1636 & -4.5638 & 1.1564 & 0.9651 \\ \hline
-50 & 5.0005 & -2.1781 & 0.8069 & 0.6495 \\ \hline
-100 & 5.7114 & -3.1355 & 0.4958 & 0.3697 \\ \hline
-150 & 5.1098 & -2.3102 & 0.8208 & 0.5688 \\ \hline
-200 & 4.7782 & -2.1203 & 0.5594 & 0.3747 \\ \hline
-250 & 4.6533 & -1.9361 & 0.4121 & 0.2588 \\ \hline
-300 & 4.3146 & -1.6399 & 0.2408 & 0.1314 \\ \hline
-350 & 4.1778 & -1.4534 & 0.2694 & 0.1251 \\ \hline
-400 & 4.2510 & -1.4863 & 0.2683 & 0.1030 \\ \hline
-450 & 3.9734 & -0.9916 & 0.2973 & 0.0982 \\ \hline
-500 & 3.9506 & -1.4429 & 0.4428 & 0.1097 \\ \hline
-550 & 3.9919 & -0.9308 & 0.7750 & 0.1497 \\ \hline
\end{tabular}
\caption{Values for the coefficients on the first order polynomial fit adjusted on te different velocity bins.}
    \label{coef velocity bins energy ALL}
\end{table}

To corroborate the only energy-dependence the same analysis was realized on the Spatially Resolved Quiet Sun Atlas for $\mu=0$.
The plot of relative velocity against lower excitation energy for this atlas presents the same behavior as the solar flux.
However, performing the first order polynomial fits was found that the values of slopes are higher (see figure \ref{velocity bins energy SPA} and table \ref{coef velocity bins energy SPA}). 

\begin{figure}[H]
    \centering
    \includegraphics[width=1.1\linewidth]{Images/Results/Sharpness_ALL.pdf}
    \caption{Individual plots of lower excitation energy across line depth representing each bin of velocity for the Spatially Resolved Quiet Sun Atlas.}
    \label{velocity bins energy SPA}
\end{figure}

\begin{table}
    \begin{tabular}{|c|c|c|c|c|}
\hline
\textbf{Velocity bin} & \textbf{Shift} & \textbf{Slope} & \textbf{Error slope} & \textbf{Error shift} \\ \hline
0 & 7.1636 & -4.5638 & 1.1564 & 0.9651 \\ \hline
-50 & 4.7903 & -1.9274 & 1.1583 & 0.9496 \\ \hline
-100 & 5.5099 & -2.9149 & 2.0695 & 1.6744 \\ \hline
-150 & 5.4525 & -2.7907 & 1.2421 & 0.9195 \\ \hline
-200 & 5.3471 & -2.8771 & 0.9840 & 0.7277 \\ \hline
-250 & 4.8238 & -2.1774 & 0.7516 & 0.5241 \\ \hline
-300 & 3.8374 & -0.9802 & 0.3714 & 0.2419 \\ \hline
-350 & 3.4399 & -0.3246 & 0.3473 & 0.1931 \\ \hline
-400 & 3.8607 & -0.9455 & 0.3184 & 0.1511 \\ \hline
-450 & 3.7832 & -0.7942 & 0.3032 & 0.1174 \\ \hline
-500 & 3.7206 & -1.1728 & 0.4614 & 0.1291 \\ \hline
-550 & 3.9324 & -1.8835 & 1.0826 & 0.1770 \\ \hline
\end{tabular}
    \caption{Values for the coefficients on the first order polynomial fit adjusted on te different velocity bins.}
    \label{coef velocity bins energy SPA}
\end{table}

Then, the rotational doppler affected the energy...(\textit{argument on construction})

\section{The second signature: Line profile bisector asymmetry}

\subsection{The bisector slope}

The line core bisector slope was calculated following the equation \eqref{third derivative relation} and plotted against line depth as shown in figure \ref{bisector VIS}.

\begin{figure}[H]
    \centering
    \includegraphics[width=1.1\linewidth]{Images/Results/Bisector_VIS.pdf}
    \caption{Line profile bisector slope for the visible range in the Solar Flux Atlas. The behavior of the plot is according to the C-curved shape of the line bisector affected by convection movement.}
    \label{bisector VIS}
\end{figure}

The weakest lines have a non c-curved profile bisector due to the small depth, so the bisector just shows an redshift or a slope. 
On the other hand, the lines which are more deeper in the photosphere experimenting the convective blueshift in its totality, so the slope had to be negative.  
However, the values for line depth which are in the middle represents equilibrium point where the convective cell overshoot come back due to gravitational forces.
Then, a first order polynomial fit was adjusted to the range $0.3-0.6F/F_c$ as shown figure \ref{bisector slope VIS}.

\begin{figure}[H]
    \centering
    \includegraphics[width=1.1\linewidth]{Images/Results/Bisector_VIS.pdf}
    \caption{First order polynomial fit adjusted to the range $0.3-0.6F/F_c$ for the line bisector slopes.}
    \label{bisector slope VIS}
\end{figure}

For a point of comparision, the same analysis was realized to the disk center spectra (see figure \ref{}).

\begin{figure}[H]
    \centering
    \includegraphics[width=1.1\linewidth]{Images/Results/Bisector_VIS.pdf}
    \caption{Line profile bisector slope for the visible range in the Spatially Resolved Quiet Sun Atlas.}
    \label{bisector SPA}
\end{figure}

Where the center disk shows a small slope than all integrated flux, but with the same behavior (see figure \ref{bisector slope SPA}).

\begin{figure}[H]
    \centering
    \includegraphics[width=1.1\linewidth]{Images/Results/Bisector_VIS.pdf}
    \caption{First order polynomial fit adjusted to the range $0.3-0.6F/F_c$ for the line bisector slopes.}
    \label{bisector slope SPA}
\end{figure}
 
This leads us to...(argument in process)

\subsection{Flux deficit (in progress)}

Due to radiation, the redshift part of a line is displced on the flux, inducing a rotation on the c-curved profile bisector.
This phenomenon is called flux deficit.
Hamilton and Lester \cite{Hamilton_1999} noticed that the behavior of the third signature mimics the bisectors
gives the sight of the mean bisectors following the granulation pattern behavior, this last is discussed in the next section. 
Later, Gray and Oostra \cite{Gray_Oostra_2018} show that the bisectors need to follow the granulation pattern as the form of we calculated velocities and bisectors.

For comparision with Gray and Oostra work, was taken the line $6254.2850\mathring{A}$ as shown in figure \ref{bisector flux deficit}.

\begin{figure}[H]
    \centering
    \includegraphics[width=1.1\linewidth]{Images/Results/Bisector_VIS.pdf}
    \caption{C-curved line profile bisector for the $6254.2850\mathring{A}$ and the standard curve for the green range from the third signature plot.}
    \label{bisector flux deficit}
\end{figure}

Following the same method, the figure \ref{flux deficit} shows a flux imbalance described on the distribution and temperatures of the mean, the maximum and the RMS point of the bisector.

\begin{figure}[H]
    \centering
    \includegraphics[width=1.1\linewidth]{Images/Results/Bisector_VIS.pdf}
    \caption{Flux deficit curve for the $6254.2850\mathring{A}$ and the temperatures of the mean, the maximum and the RMS point of the bisector.}
    \label{flux deficit}
\end{figure}

Taking the model on Gray \cite{Gray_2005}, the respective temperatures are (). 

As we found the standard curves for all the ranges, the same analysis was realized for a random line in the range of the respective standard curve.

\section{The third signature: Line depth-dependence on wavelength shifts}

\subsection{The granulation pattern}

The granulation patterns for the IAG Solar Flux Atlas in all the wavelength range was obtained.

\begin{figure}[H]
    \centering
 \includegraphics[width=1.1\linewidth]{Images/Results/GranulationPattern_ALL.pdf}
    \caption{Granulation pattern obtained for the Solar Flux Atlas. The wavelength shift dependence is along the line depth.}
    \label{Granulation solar flux}
\end{figure}

As shown in the figure \ref{Granulation solar flux} the behavior along the line depth is according to literature, in which is clear the wavelength shift dependence along the line depth or chromodependence.
For a description of this tendency was performed an analysis of line depth against wavelength. The hypothesis was: “If there exists a velocity shift only in the Solar Flux Atlas, then rotation could be the cause of this phenomenon.”
However, the velocity shift was observed in both spectral datasets. This was initially unexpected because the rotation is negligible at the disk center.

For the measurement of wavelength shifts a range from $4300 \mathring{A}$ to $5600 \mathring{A}$ was taken. 
Then was sorted all wavelengths from both atlases into $50 m/s$ velocity bins (see figure \ref{velocity bins}).

\begin{figure}[H]
     \centering
     \begin{subfigure}{1.0\textwidth}
         \includegraphics[width=\textwidth]{Images/Results/Velocity bins VIS_plot.pdf}
         \caption{Visible range for the Solar Flux Atlas.}

         \label{vel bins plot VIS}
     \end{subfigure}
\hfill
     \begin{subfigure}{1.0\textwidth}
         \includegraphics[width=\textwidth]{Images/Results/Velocity bins SPA_plot.pdf}
         \caption{Visible range for the Spatially Resolved Quiet Sun Atlas.}
         \label{vel bins plot SPA}
     \end{subfigure}

        \caption{Comparision between atlases with velocity bins for the relation between wavelength and line depth. For each velocity bin was plotted wavelength against line depth to measure a frequency shift with a first order polynomial fit.}
        \label{Velocity bins}
\end{figure}

For each velocity bin was plotted wavelength against line depth to measure a frequency shift with a first order polynomial fit (see figure \ref{Velocity bins slopes})

\begin{figure}[H]
     \centering
     \begin{subfigure}{1.0\textwidth}
         \includegraphics[width=\textwidth]{Images/Results/Velocity bins VIS.pdf}
         \caption{Velocity bins of $50m/s$ for the Solar Flux Atlas.}

         \label{vel bins SPA}
     \end{subfigure}
\hfill
     \begin{subfigure}{1.0\textwidth}
         \includegraphics[width=\textwidth]{Images/Results/Velocity bins SPA.pdf}
         \caption{Velocity bins of $50m/s$ for the Spatially Resolved Quiet Sun Atlas.}
         \label{vel bins SPA}
     \end{subfigure}

        \caption{Individual plots of lower excitation energy across line depth representing each bin of velocity.}
        \label{Velocity bins slopes}
\end{figure}

The tables \ref{velocity bins SPA} and \ref{velocity bins VIS} shows the obtained slopes in both atlases. 

\begin{table}[H]
\begin{tabular}{|r|r|r|r|r|}
\hline
\multicolumn{1}{|c|}{\textbf{Velocity bin}} & \multicolumn{1}{c|}{\textbf{Shift}} & \multicolumn{1}{c|}{\textbf{Slope}} & \multicolumn{1}{c|}{\textbf{Error slope}} & \multicolumn{1}{c|}{\textbf{Error shift}} \\ \hline
50 & 1.1937 & -0.000063 & 0.000022 & 0.1166 \\ \hline
0 & 1.1720 & -0.000056 & 0.000018 & 0.0923 \\ \hline
-50 & 1.2840 & -0.000087 & 0.000021 & 0.1087 \\ \hline
-100 & 1.3953 & -0.000112 & 0.000012 & 0.0584 \\ \hline
-150 & 1.4327 & -0.000124 & 0.000017 & 0.0842 \\ \hline
-200 & 1.5028 & -0.000142 & 0.000013 & 0.0658 \\ \hline
-250 & 1.5712 & -0.000165 & 0.000018 & 0.0890 \\ \hline
-300 & 1.6233 & -0.000184 & 0.000019 & 0.0971 \\ \hline
-350 & 1.4129 & -0.000154 & 0.000055 & 0.2583 \\ \hline
-400 & 1.6455 & -0.000208 & 0.000033 & 0.1617 \\ \hline
-450 & 1.1952 & -0.000144 & 0.000060 & 0.3094 \\ \hline
\end{tabular}
\caption{Slopes for the first order polynomial fit in each velocity bin for the Spatially Resolved Quiet Sun Atlas.}
\label{velocity bins SPA}
\end{table}

\begin{table}[H]
\begin{tabular}{|r|r|r|r|r|}
\hline
\multicolumn{1}{|c|}{\textbf{Velocity bin}} & \multicolumn{1}{c|}{\textbf{Shift}} & \multicolumn{1}{c|}{\textbf{Slope}} & \multicolumn{1}{c|}{\textbf{Error slope}} & \multicolumn{1}{c|}{\textbf{Error shift}} \\ \hline
-100 & 1.2704 & -0.000091 & 0.000015 & 0.0797 \\ \hline
-150 & 1.2601 & -0.000094 & 0.000013 & 0.0673 \\ \hline
-200 & 1.3340 & -0.000115 & 0.000019 & 0.0985 \\ \hline
-250 & 1.3920 & -0.000133 & 0.000023 & 0.1164 \\ \hline
-300 & 1.3724 & -0.000139 & 0.000017 & 0.0874 \\ \hline
-350 & 1.3106 & -0.000137 & 0.000015 & 0.0744 \\ \hline
-400 & 1.2119 & -0.000128 & 0.000018 & 0.0866 \\ \hline
-450 & 1.4212 & -0.000181 & 0.000039 & 0.1947 \\ \hline
-500 & 1.3787 & -0.000189 & 0.000052 & 0.2609 \\ \hline
\end{tabular}
\caption{Slopes for the first order polynomial fit in each velocity bin for the Solar Flux Atlas.}
\label{velocity bins VIS}
\end{table}

The value of slopes for the Spatially Resolved Quiet Sun Atlas are greater than the Solar Flux Atlas, which contradicts the initial hypothesis.

\subsection{Characterization of chromodepence on granulation pattern}

The figure \ref{standard curve gray} show the standard curve proposed for Gray and Oostra \cite{Gray_Oostra_2018}, stablish on the range $6020-6340\mathring{A}$

\begin{figure}[H]
    \centering
    \includegraphics[width=1.1\linewidth]{Images/Results/ColorCurves_VIS.pdf}
    \caption{Standard curve given by Gray and Oostra and recalculated for comparision with the current work.}
    \label{standard curve gray}
\end{figure}

As the standard curve don't take into account all the wavelengths and the given third order polynomial fit overstimate the points, a different approach was proposed.
Separating into color ranges, a second order polynomial fit was adjusted.

\begin{figure}[H]
    \centering
    \includegraphics[width=1.1\linewidth]{Images/Results/ColorCurves_VIS.pdf}
    \caption{Granulation pattern for the Solar Flux Atlas with color curves. The tendency on the curves is more pronounced in the violet and red range.}
    \label{Curves Solar Flux}
\end{figure}

The tendency on the curves is more pronounced in the violet and near infrared range (see table \ref{coeficients color curves}). 
This can be interpreted as the standard curve from Gray and Oostra is shifted and flattered along decreasing wavelengths.

\begin{table}[H]
    \centering
\begin{tabular}{|l|l|}
    \hline
\textbf{Wavelength range $(\mathring{A})$} & \textbf{Color coefficient}  \\ \hline
3800-4270         &    4924.2052          \\\hline
4270-4760           &     757.0966        \\\hline
4760-4970         &      263.3435     \\ \hline
4970-5700         &     204.8798      \\ \hline
5700-5810         &       155.9806        \\\hline
5810-6180           &        190.1927           \\ \hline
6180-7800            &        142.9648       \\ \hline
7800-11000            &        160.4951     \\  \hline
\end{tabular}
\caption{Values for the second order polynomial fit coefficients shown in figure \ref{Curves Solar Flux coef}.}
\label{coeficients color curves}
\end{table}

Taking the curve for the range $4970-5700\matring{A}$ for the starting point or "standard curve" different scaling coefficients wer calculated as shown the table \ref{color coefficients}.

4000     0.535967
4500     0.819507
4850     0.834863
5500     1.000000
5800     1.059542
6000     0.983457
7000     1.014930
9000     0.285721
14000   -0.011557

\begin{table}[H]
    \centering
\begin{tabular}{|l|l|}
    \hline
\textbf{Wavelength range $(\mathring{A})$} & \textbf{Color coefficient}  \\ \hline
3800-4270         &    0.5359          \\\hline
4270-4760           &     0.8195        \\\hline
4760-4970         &      0.8348    \\ \hline
4970-5700         &    1.000     \\ \hline
5700-5810         &       1.0595       \\\hline
5810-6180           &        0.9834          \\ \hline
6180-7800            &        1.0149      \\ \hline
7800-11000            &        0.2857    \\  \hline
11000-23000            &        -0.0115    \\  \hline
\end{tabular}
\caption{Scaling factors to the new standard curve shown in the equation \eqref{new standard curve}.}
\label{color coefficients}
\end{table}

The equation \eqref{new standard curve} reports a new standard curve which coefficients depends on the color, and give the corresponding granulation pattern curve for each color range exposed on the table \ref{color coefficients}.

\begin{equation}
    F/F_c = 1.2925\times 10^{-6}\alpha_c v^2 - 0.0019\alpha_c v +0.8671	\alpha_c
\label{new standard curve}
\end{equation}

The near infrared range dont show the same tendency...

\section{Higher quality graphs}

To show the improvement on the quality in the third signature plots, the figure \ref{Scattered points} compare the plot obtained by Ellwarth \cite{Ellwarth_2023} and the graphic.

\begin{figure}[H]
     \centering
     \begin{subfigure}{0.9\textwidth}
         \includegraphics[width=\textwidth]{Images/Results/Standard curve_Ellwarth.pdf}

     \end{subfigure}
\hfill
     \begin{subfigure}{0.8\textwidth}
         \includegraphics[width=\textwidth]{Images/Granulation pattern Ellwarth.jpg}
     \end{subfigure}

        \caption{We realized the same graphic for the comparision with the Ellwarth article to show the less scattered points.}
        \label{Scattered points}
\end{figure}

The less scattered points and the improvement on the chromodepence identification is evident.

Furthermore, was analysed the variance of observed wavelengths
