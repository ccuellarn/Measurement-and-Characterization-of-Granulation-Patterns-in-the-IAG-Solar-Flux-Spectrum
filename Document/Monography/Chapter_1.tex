For decades, the solar spectrum has served as the fundamental reference point for spectroscopic analysis and characterization. 
As our nearest star, the Sun has enabled comprehensive studies of stellar composition. 
However, advances in optical instrumentation have recently revealed previously undetected spectral details, providing new insights into even the most basic solar properties from dynamics to surface geometry. 
One of this advances is the granulation pattern due to convection motion, which is revealed by three fundamental signatures of its hydrodynamics:
Line broadening, line profile bisector asymmetry, and line-depth dependence of the convective blueshift.

Inspired by Gray’s foundational research \cite{Gray_2009}, this project seeks to extract the granulation pattern from the IAG Solar Flux Atlas to calculate relative velocities and perform detailed analysis. 
A parallel focus will investigate the chromatic dependence detected in absorption line profiles (the third signature of convection), which currently obscures the universality of the pattern. 

\section{Convective motion}

The stellar spectrum serves as astronomy's primary source of information, particularly regarding a star's composition and relative velocity. 
However, convective motions in the stellar photosphere complicate spectral interpretation by inducing differential velocities in individual spectral lines. 
Specifically, fluid movements caused by density variations from temperature fluctuations in the Sun's outermost layer modify the spectrum, causing each spectral line to display distinct relative velocities as we can relate to different research \cite{Dravins_1981,Hamilton_1999,Gray_2009,Reiners_2016}. 
Persistent convective motion generates a granular structure in solar photospheric images. 
One of the first person to discover the pattern was Janssen in 1885, who detected a granulation movement in the photosphere. Lately, in 1901 Plasketts associated this pattern with te same of convective cells in Bernard's experiment \cite{Foukal_1990}.
Where fluids heated from bellow representing hot rising gas elements convecting heat to surface.
This characteristic configuration gives the name \textit{granulation} to all observable signs of the convection. 

\section{Solar granulation pattern}
A plot of Doppler shift against line depth is called \textit{Granulation Pattern}; it shows that weaker lines are more blueshifted. 
When the Sun pushes material up through its outer layer, the spectrum exhibits a blueshift. As this material subsequently cools and falls back through the atmosphere, it produces a redshift, but emits less light, making the blueshift dominant.
This phenomenon has been particularly documented by David Gray, whose work has significantly improved the precision of stellar radial velocity measurements. 
This improvement stems primarily from Gray's observation that granulation patterns in stars resemble solar patterns, differing only by a scaling factor. 
These findings are particularly significant given the considerable challenges in obtaining such parameters for other stars, which are affected by spectral noise, stellar proper motions, and relative velocity uncertainties \cite{Gray_2009}.
Furthermore, analysis of solar granulation patterns contributes to the understanding and validation of photospheric hydrodynamic models \cite{Dravins_1981}. 
Such analysis enables improved calibration when testing dynamic atmospheric models.

\subsection{The three signatures of convection}
Due to convection motion, there's three distinct granulation signatures can be identified in stellar spectra: Line broadening, line profile bisector asymmetry, and line-depth dependent shifts in absorption lines \cite{Gray_Pugh_2012}. 
The third signature exhibits a correlation between line depth and wavelength shift\cite{Gray_2009}, for which neutral iron lines (Fe I) serve as optimal calibration references due to their high abundance, minimal thermal broadening and limited isotopic variation. 
This approach offers the additional advantage of deriving natural wavelengths from a single source, thereby eliminating potential discrepancies.

On the way of trating th Sun as any other star, which does not have the proximity for spectra study, we search for obtain the most accurate and precise granulation pattern.
Which is posible if we take the most accurate and precise solar atlas.

\section{IAG Solar Flux Atlas}

In 2016, Reiners et al.\cite{Reiners_2016} published the unprecedented precision \textit{Institut für Astrophysik Göttingen (IAG)} Solar Flux Atlas obtained with the FTS \textit{Fourier Transform Spectrograph}, simultaneously reporting convective blueshifts for a sample of neutral iron lines. 
This atlas provides highly precise and accurate data, with radial velocity uncertainties on the order of $\pm 10 m/s$ across the wavelength range of $4050$ to $10650$ $\mathring{A}$. 
In contrast to other FTS atlases, the entire visible wavelength range was observed simultaneously using only one spectrograph setting\cite{Reiners_2016}.

In 2016, the resulting granulation pattern appeared notably scattered and noisy, attributable to the rudimentary line position measurement methodology employed, and the poorly curated line selection, which included numerous blended features, outdated wavelength references and incomplete spectral coverage.
The exceptional quality of the IAG spectrum enables more accurate determination of convective blue-shifts when analyzed through refined methods. 

\section{IAG Spatially Resolved Quiet Sun Atlas}

In 2023, Ellwarth et al.\cite{Ellwarth_2023} observed and published the \textit{Institut für Astrophysik Göttingen (IAG)} Spatially Resolved Quiet Sun Atlas obtained with the FTS \textit{Fourier Transform Spectrograph}.
This atlas has the advantage of use observations from the disc center ($\mu = 1.0$) towards the solar limb ($\mu = 0$), where $\mu =\cos(\theta)$. The research had the objective of study the blueshift exhibits variations from the disc centre to the solar limb due to differing projection angles onto the solar atmosphere.

Our principal objective is to establish a characterization of the granulation pattern by treating the Sun as any other star. 
This approach allows us to develop methods that can be directly scaled and applied to other solar-type stars.
However, to accurately analyze relative velocities and dynamics, we must account for insights that can only be corrected using the Spatially Resolved Quiet Sun Atlas.

\section{Motivation}
During the first semester of 2025, physics student on the Universidad de los Andes Manuel Fuentes implemented these improvements in his computational project, focusing specifically on the visible spectral range $4050-10650$ $\mathring{A}$. 
By developing enhanced measurement techniques and employing a carefully curated line list that fully covers the IAG-VIS range with updated wavelength references \cite{Nave_1994}, Fuentes achieved significantly sharper granulation patterns compared to previous analyses. 
This optimized approach demonstrates how proper line selection and modern wavelength standards can extract more reliable convective signatures from high-quality solar spectra.

With professor Benjamin, we continue with a measurement of granulation pattern taking into account the near infrared range.
This region contains spectral lines originating from deeper layers of the solar photosphere, which are consequently weaker. 
These lines represent a rich source of valuable information \cite{Cacciani_2006}, though they require adapting the measurement methodology. 
Furthermore, we start a characterization of solar dynamics and line asymmetries based on granulation pattern observations.

By addressing these challenges, we aim to achieve two key outcomes: A robust characterization of solar granulation pattern and insights into the photosphere’s geometry. 
These efforts are guided by the central research question: What are the direct consequences of solar dynamics on its spectrum?





