For decades, the solar spectrum has served as the fundamental reference point for spectroscopic analysis. 
As our nearest star, the Sun enables detailed studies of stellar composition. 
However, advances in optical instrumentation have revealed previously undetected spectral details, providing new insights into fundamental solar properties.
One key insight is the granulation patterns caused by convective motion, which are characterized by three distinct hydrodynamic signatures: Line broadening, line profile asymmetry, and the line-depth dependence of convective blueshift.

Inspired by David Gray's foundational research, this project aims to extract the granulation patterns from the IAG Solar Flux Atlas to calculate relative velocities and perform a detailed analysis based on the three signatures of convection. 

\section{The three signatures of convection}

The stellar spectrum serves as astronomy's primary source of information, encoding details about a star's composition and velocity. 
However, in some stars this spectrum is modified due to fluid motions caused by density and temperature fluctuations in the outermost layer, where each spectral line exhibits a distinct velocity shift. 
These persistent convective motions generate a granular structure in the solar photosphere, a phenomenon supported by extensive research (see~\cite{Gray_2009,Hamilton_1999, Dravins_1981}).

The granulation in the solar photosphere was first observed as a moving granular structure by Janssen in 1885. 
Later, in 1901, Plaskett associated this pattern with the convective cells observed in Bénard's experiment~\cite{Foukal_1990}, where fluids heated from below produce rising elements of hot gas convecting heat to the surface.
This characteristic configuration produces three signatures of granulation in stellar spectra: Line broadening, line profile asymmetry, and line-depth dependence of convective blueshift~\cite{Gray_Pugh_2012}, all three related to line depth.
The method for quantifying these relations are known as the \textit{granulation patterns}; which characterize the signatures mentioned above~\cite{Gray_2009}.
The physical origin of these patterns lies in the convective process on the solar photosphere itself. 

When the Sun pushes material up through its outer layer, the spectrum exhibits a blueshift. 
As this material subsequently cools and falls back through the atmosphere, it produces a redshift but emits less light, making the blueshift dominant.
This characteristic is called \textit{convective blueshift}.

Neutral iron (Fe I) lines are optimal for calibrating this phenomenon due to their high abundance, minimal thermal broadening, limited isotopic variation and availability of accurate laboratory measurements of natural wavelengths. 
Furthermore, the even number of nucleons in the most abundant isotope resulting in no hyperfine structure shown.
The use of a single chemical species offers the additional advantage of retrieving natural wavelengths from a single source, thereby eliminating potential discrepancies~\cite{Oostra_Vargas_2022}.
The project used the Nave list of Fe I lines, which presented the values of laboratory measurements~\cite{Nave_1994}.

The study of granulation patterns has been significantly documented by David Gray, whose work has improved the precision of stellar radial velocity measurements (see~\cite{Gray_2009} and references therein).
This improvement stems primarily from his observation that granulation patterns in solar-type stars closely resemble the granulation pattern from the third signature of convection.
The line depth-dependence of convective blueshift for different solar-type stars differing primarily by a scaling factor from the solar one. 
This is particularly significant given the considerable challenges of obtaining such precise measurements for other stars, which are often affected by spectral noise, stellar proper motions, and velocity uncertainties~\cite{Gray_2009}.
Furthermore, analyzing solar granulation patterns is crucial for validating photospheric hydrodynamic models~\cite{Dravins_1981}, enabling improved calibration and testing of dynamic atmospheric models.

Following the motivation of treating the Sun as any other star, which does not have the angular resolution necessary to study the spatial granulation structure, the most accurate solar flux atlas is necessary.

\section{The IAG solar flux atlas}

In 2016, Reiners and colaborators published the unprecedented precision \textit{Institut für Astrophysik und Geophysik} (IAG) solar flux atlas obtained with the FTS \textit{Fourier Transform Spectrograph} at Göttingen, simultaneously reporting convective blueshifts for a sample of Fe I lines. 
This atlas provides highly precise and accurate data, with radial velocity uncertainties on the order of $\pm 10$  m/s across the wavelength range from $\SI{4050}{\angstrom}$ to $\SI{10650}{\angstrom}$. 
In contrast to other FTS atlases, the entire visible wavelength range was observed simultaneously using only one spectrograph setting~\cite{Reiners_2016}.

Despite the exceptional quality of the spectrum, the first derived granulation pattern for the third signature of convection appeared notably scattered and noisy. 
This was attributable to a rudimentary line position measurement methodology and a poor line selection which included numerous blended features, outdated wavelength references, and incomplete spectral coverage.
Even if the atlas contains the dataset for the near infrared range, no analysis was implemented on this part of the spectrum. 
Consequently, the full potential of the IAG solar flux atlas for determining precise convective blueshifts remains unrealized, highlighting the need for the refined analytical methods applied in this project.

\section{The IAG spatially resolved quiet Sun atlas}
In 2023, Ellwarth and colaborators~\cite{Ellwarth_2023} published the \textit{Institut für Astrophysik und Geophysik} (IAG) spatially resolved quiet sun atlas, also obtained with the FTS \textit{Fourier Transform Spectrograph} at Göttingen. 
This atlas has the advantage of its coverage from the disk center ($\mu = 1.0$) towards the solar limb ($\mu = 0$), where $\mu =\cos(\theta)$ with $\theta$ being the angle between the surface normal and the observer's position.
This spatial resolution enables the study of how convective blueshifts vary across the solar disk due to changing projection angles, a dependence that is not observed in other stars.

The principal goal of this project is to characterize the solar granulation patterns by treating the Sun as any other star. 
To achieve this, the disk-integrated IAG solar flux atlas serves as our primary data source. 
However, a precise analysis of the relative velocities in this integrated light requires correcting for the center-to-limb variations that are uniquely quantified by the IAG spatially resolved quiet Sun atlas.

\section{Motivation and goals}

In the first semester of 2025, Manuel Fuentes, a physics student at the Universidad de los Andes, implemented these improvements in a computational project focused on the visible spectral range. 
By developing enhanced measurement techniques and employing a carefully curated line list that fully covers the visible range with updated wavelength references, Fuentes achieved significantly sharper granulation patterns than previous analyses. 
This work demonstrated that proper line selection and modern wavelength standards can reliably extract convective signatures from high-quality solar spectra.

Under the guidance of Professor Benjamin, the present project studies the visible range and extends the analysis into the near infrared range.
This region contains spectral lines that originate from deeper photospheric layers.
Although these lines are consequently weaker, they represent a rich source of information. 
This project also extends previous efforts by measuring the line profile asymmetry and line broadening, other spectral signs of granulation beyond convective blueshift; and exploring how these signatures depend on wavelength. 
This last is our principal target to characterize: The observational fact of granulation patterns showing individual behaviors with wavelength dependence.

By addressing these challenges, this project intended to produce a robust characterization of the solar granulation patterns emphasizing in the anomalous chromodependence.
These efforts were guided by the central research question: What are the direct spectroscopic consequences of solar convection?
