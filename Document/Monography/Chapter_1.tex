% For decades, the solar spectrum has served as the fundamental reference point for spectroscopic analysis and characterization. As our nearest star, the Sun has enabled comprehensive studies of stellar composition. However, advances in optical instrumentation have recently revealed previously undetected spectral details, providing new insights into even the most basic solar properties from dynamics to surface geometry. 

% \section{Convective movement}

% The stellar spectrum serves as astronomy's primary source of information, particularly regarding a star's composition and relative velocity. However, convective motions in the stellar photosphere complicate spectral interpretation by inducing differential velocities in individual spectral lines. Specifically, fluid movements caused by density variations from temperature fluctuations in the Sun's outermost layer modify the spectrum, causing each spectral line to display distinct relative velocities \cite{Graham_1976}. Consequently, when the Sun pushes material up through its outer layer, the spectrum exhibits a blueshift. As this material subsequently cools and falls back through the atmosphere, it produces a redshift, but emits less light, making the blueshift dominant.

% \section{Granulation pattern}

% Persistent convective motion generates a granular structure in solar photospheric images. This characteristic configuration gives the name \textit{granulation} to all observable signs of the convection. For example, a plot of Doppler shift against line depth is called \textit{Granulation Pattern}; it shows that weaker lines are more blueshifted. This phenomenon has been particularly documented by David Gray, whose work has significantly improved the precision of stellar radial velocity measurements. This improvement stems primarily from Gray's observation that granulation patterns in stars resemble solar patterns, differing only by a scaling factor. These findings are particularly significant given the considerable challenges in obtaining such parameters for other stars, which are affected by spectral noise, stellar proper motions, and relative velocity uncertainties \cite{Gray_2009}. Furthermore, analysis of solar granulation patterns contributes to the understanding and validation of photospheric hydrodynamic models \cite{Dravins_1981}. Such analysis enables improved calibration when testing dynamic atmospheric models.

% Three distinct granulation signatures can be identified in stellar spectra: Line broadening, profile asymmetry, and depth-dependent wavelength shifts in absorption lines \cite{Gray_Pugh_2012}. The third signature exhibits a correlation between line depth and wavelength shift, for which neutral iron lines (Fe I) serve as optimal calibration references due to their high abundance, minimal thermal broadening and limited isotopic variation. This approach offers the additional advantage of deriving natural wavelengths from a single source, thereby eliminating potential discrepancies.

% \section{IAG Solar Flux Spectrum}

% In 2016, Reiners \cite{Reiners_2016} published the unprecedented precision \textit{Institut für Astrophysik Göttingen (IAG)} solar flux spectrum obtained with the FTS \textit{Fourier Transform Spectrograph}, simultaneously reporting convective blueshifts for a sample of neutral iron lines. The resulting granulation pattern appeared notably scattered and noisy, attributable to the rudimentary line position measurement methodology employed, and the poorly curated line selection, which included numerous blended features, outdated wavelength references and incomplete spectral coverage.

% The exceptional quality of the IAG spectrum enables more accurate determination of convective blue-shifts when analyzed through refined methods. During the 2025-1 semester, physics student on the Universidad de los Andes Manuel Fuentes implemented these improvements in his computational project, focusing specifically on the visible spectral range 405-1065 nm. By developing enhanced measurement techniques and employing a carefully curated line list that fully covers the IAG-VIS range with updated wavelength references \cite{Nave_1994}, Fuentes achieved significantly sharper granulation patterns compared to previous analyses. This optimized approach demonstrates how proper line selection and modern wavelength standards can extract more reliable convective signatures from high-quality solar spectra.

% Before publishing these results, it would be valuable to apply the same procedure to the infrared portion of the spectrum 1000-2300 nm. This region contains spectral lines originating from deeper layers of the solar photosphere, which are consequently weaker. These lines represent a rich source of valuable information \cite{Cacciani_2006}, though they require adapting the measurement methodology. Furthermore, the results will need thorough examination.

% The most evident discovery is the apparent chromatic dependence in Manuel Fuentes results, where the granulation pattern shows wavelength-range variations. Two potential solutions emerge: Establishing distinct "solar reference curves" for different chromatic ranges, or developing an algorithm to remove chromatic dependence and consolidate all data into a single unified curve. The pattern's sharpness also varies spectrally, with preliminary evidence suggesting additional fine structures requiring verification.


% Inspired by Gray’s foundational research \cite{Gray_2009}, this project seeks to extract the granulation pattern from the IAG infrared spectrum (1000–2300 nm) to calculate relative velocities and perform detailed analysis. A parallel focus will investigate the chromatic dependence detected in absorption line profiles, which currently obscures the universality of the pattern. By addressing these challenges, we aim to achieve two key outcomes: A robust characterization of solar granulation pattern and insights into the photosphere’s geometry. These efforts are guided by the central research question: What are the direct consequences of solar dynamics on its spectrum?


\section{The convective motion in stars}
\subsection{The three signatures}
\section{Suns granulation pattern}
\section{The third signature of convection in the sun}
\subsection{IAG Solar flux atlas}
\subsubsection{charcaterization}
\subsection{IAG Spatially Resoved quiet sun atlas}
\subsubsection{Velocity analysis}


