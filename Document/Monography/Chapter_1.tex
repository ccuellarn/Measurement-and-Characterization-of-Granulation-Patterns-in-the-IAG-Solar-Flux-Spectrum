For decades, the solar spectrum has served as the fundamental reference point for spectroscopic analysis. 
As our nearest star, the Sun enables detailed studies of stellar composition. 
However, advances in optical instrumentation have revealed previously undetected spectral details, providing new insights into fundamental solar properties
One key insight is the granulation pattern caused by convective motion, which is characterized by three distinct hydrodynamic signatures: Line broadening, line profile asymmetry, and the line-depth dependence of convective blueshift.

Inspired by David Gray's foundational research (\cite{Gray_2009} and references therein), this project aims to extract the granulation pattern from the IAG Solar Flux Atlas to calculate relative velocities and perform a detailed analysis based on the three signatures of convection. 

\section{Convective motion}

The stellar spectrum serves as astronomy's primary source of information, encoding details about a star's composition and velocity. 
However, in some stars this spectrum is modified by fluid motions caused by density variations from temperature fluctuations in the outermost layer, where each spectral line exhibit a distinct velocity shift. 
These persistent convective motions generate a granular structure in the solar photosphere, a phenomenon supported by extensive research (see~\cite{Hamilton_1999, Dravins_1981, Gray_2009}).

The granulation in the Sun photosphere was first observed as a moving granular structure by Janssen in 1885. 
Later, in 1901, Plaskett associated this pattern with the convective cells observed in Bénard's experiment~\cite{Foukal_1990}, where fluids heated from below produces rising elements of hot gas convecting heat to the surface.
This characteristic configuration gives the name \textit{granulation} to all observable signs of solar convection.

\subsection{The three signatures of convection}
Convective motion produces three signatures of granulation in stellar spectra: Line broadening, line profile asymmetry, and line-depth dependence of convective blueshift~\cite{Gray_Pugh_2012}.
This project focuses on the third signature, which exhibits a correlation between line depth and relative velocity. 
The method for quantifying this relationship is known as the \textit{granulation pattern}, which means the plot of relative velocities against line depth~\cite{Gray_2009}.

The physical origin of this pattern lies in the convective process itself. 
When the Sun pushes material up through its outer layer, the spectrum exhibits a blueshift. 
As this material subsequently cools and falls back through the atmosphere, it produces a redshift but emits less light, making the blueshift dominant.
Neutral iron lines (Fe I) are optimal for calibrating this phenomenon due to their high abundance, minimal thermal broadening, limited isotopic variation and availability of accurate laboratory measurements of natural wavelengths. 
Furthermore, the even number of nucleons in the most abundant isotope resulting in no hyperfine structure shown.
The use of a single chemical species offers the additional advantage of retrieving natural wavelengths from a single source, thereby eliminating potential discrepancies~\cite{Oostra_Vargas_2022}.

The study of granulation patterns has been significantly documented by David Gray, whose work has improved the precision of stellar radial velocity measurements (\cite{Gray_2009} and references therein).
This improvement stems primarily from his observation that granulation patterns in solar-type stars closely resemble the solar pattern, differing primarily by a scaling factor. 
This is particularly significant given the considerable challenges of obtaining such precise measurements for other stars, which are often affected by spectral noise, stellar proper motions, and velocity uncertainties~\cite{Gray_2009}.
Furthermore, analyzing solar granulation patterns is crucial for validating photospheric hydrodynamic models~\cite{Dravins_1981}, enabling improved calibration and testing of dynamic atmospheric models.

Following the motivation of treating the Sun as any other star, which does not have the angular resolution necessary to study the spatial granulation structure, the most accurate solar flux atlas is necessary.

\section{The IAG solar flux atlas}

In 2016, Reiners and colaborators published the unprecedented precision \textit{Institut für Astrophysik und Geophysik} (IAG) solar flux atlas obtained with the FTS \textit{Fourier Transform Spectrograph} at Göttingen, simultaneously reporting convective blueshifts for a sample of Fe I lines. 
This atlas provides highly precise and accurate data, with radial velocity uncertainties on the order of $\pm 10$  m/s across the wavelength range of $\SI{4050}{\angstrom}$ to $\SI{10650}{\angstrom}$. 
In contrast to other FTS atlases, the entire visible wavelength range was observed simultaneously using only one spectrograph setting~\cite{Reiners_2016}.

Despite the exceptional quality of the spectrum, the first derived granulation pattern appeared notably scattered and noisy. 
This was attributable to a rudimentary line position measurement methodology and a poor line selection, which included numerous blended features, outdated wavelength references, and incomplete spectral coverage.
Consequently, the full potential of the IAG atlas for determining precise convective blueshifts remains unrealized, highlighting the need for the refined analytical methods applied in this project.

\section{The IAG spatially resolved quiet sun atlas}
In 2023, Ellwarth and colaborators~\cite{Ellwarth_2023} published the \textit{Institut für Astrophysik und Geophysik} (IAG) spatially resolved quiet sun atlas, also obtained with the FTS \textit{Fourier Transform Spectrograph} at Göttingen. 
This atlas has the advantage of its coverage from the disk center ($\mu = 1.0$) towards the solar limb ($\mu = 0$), where $\mu =\cos(\theta)$ with $\theta$ being the angle between the surface normal and the observer's position.
This spatial resolution enables the study of how convective blueshifts vary across the solar disk due to changing projection angles, a dependency that cannot be observed in other stars.

The principal objective of this project is to characterize the solar granulation pattern by treating the Sun as any other star. 
To achieve this, the disk-integrated IAG solar flux atlas serves as our primary data source. 
However, a precise analysis of the relative velocities in this integrated light requires correcting for the center-to-limb variations that are uniquely quantified by the IAG spatially resolved quiet Sun atlas.

\section{Motivation}

In the first semester of 2025, Manuel Fuentes, a physics student at the Universidad de los Andes, implemented these improvements in a computational project focused on the visible spectral range. 
By developing enhanced measurement techniques and employing a carefully curated line list that fully covers the visible range with updated wavelength references, Fuentes achieved significantly sharper granulation patterns than previous analyses. 
This work demonstrated that proper line selection and modern wavelength standards can reliably extract convective signatures from high-quality solar spectra.

Under the guidance of Professor Benjamin, the present project extends the analysis into the near infrared range.
This region contains spectral lines that originate from deeper photospheric layers.
Although these lines are consequently weaker, they represent a rich source of information. 
However their analysis requires an adaptation of the existing measurement methodology. 
A key objective was to adapt these methods to enable the characterization of solar dynamics through granulation patterns in the near infrared range.
This project also extends previous efforts by measuring the line asymmetry and broadening, other spectral signs of granulation beyond conevtive blueshift; and exploring how these phenomenona depend on the wavelength range.


By addressing these challenges, this project aims to produce a robust characterization of the solar granulation pattern and new insights into the photosphere’s dynamics. 
These efforts are guided by the central research question: What are the direct spectroscopic consequences of solar convection?

\textcolor{Miku}{ADD VALT AS REFERENCE FOR OTHER COEFFICIENTS AND EXPLAIN THAT NAVE HAVE VALUES OF EXCITATION POTENTIAL}