%------------------------------------------------------------------ DOCUMENT TYPE
\documentclass[12pt]{report}
%--------------------------------------------------------------------- LOAD PACKAGES AND COMMANDS
\usepackage{monostyle} 
\nocite{*}
%---------------------------------------------------------------------- BEGIN DOCUMENT
\begin{document}
%---------------------------------------------------------------------- TITTLE PAGE
\begin{titlepage}
\centering
{\includegraphics[width=0.15\textwidth]{Universidad-de-los-Andes_logo.png}\par}
\vspace{1cm}
{\bfseries\LARGE Universidad de los Andes \par}
\vspace{1cm}
{\scshape\Large Physics department\par}
\vspace{1cm}
{\scshape\LARGE Measurement and Characterization of Granulation Patterns in the IAG Solar Flux Spectrum \par}

\vspace{1cm}
{\large BSc Physics Final Project \par}
\vfill
{\Large \textcolor{Miku}{\textbf{Author:}} \par}
{\Large Claudia Alejandra Cuellar Nieto \par}
{\Large \textcolor{Miku}{\textbf{Advisor:}} \par}
{\Large Benjamin Oostra Vannoppen \par}
\vfill
{\Large Nov 2025 \par}
\end{titlepage}
%------------------------------------------------------------------------ ABSTRACT AND ACKNOWLOGMENTS
\begin{abstract}

This project presents a detailed characterization of convective blueshift and the anomalous chromodependence across the three signatures of convection: Line broadening, line asymmetry, and the line-depth dependence of convective blueshift.
This following the foundational work of David Gray (see~\cite{Gray_2009} and references therein). 
A primary outcome is a refined, blend-free list of Fe I lines, providing a community resource with measured convective blueshifts, line core curvatures, bisector slopes, excitation potentials, and damping coefficients. 
Our analysis statistically validates a $\SI{0.1}{\angstrom}$ window around the line core as optimal for analysis, confirming the methodology of Allende and Garcia~\cite{Allende_Garcia_1998}.

Furthermore, we establish color dependent standard granulation curves across the solar spectrum, confirming that deeper lines are found at shorter wavelengths. 
Specifically, we identify an anomalous chromodependence where, for a fixed line depth, shorter-wavelength lines exhibit greater convective blueshifts, higher core bisector slopes, and narrower profiles. 
We discarded solar rotation as the cause for this phenomenon, as this anomaly is more pronounced in the disk-center spectrum. 
The chromodependence is linear in the near infrared but requires individual curves in the visible range.

Quantitative analysis of line core curvature for shallow lines reveals a slope of $(1.83 \pm 0.08)\times 10^{10}\ \text{m}^{-1}$ for the disk-integrated flux, significantly lower than the $(2.94\pm 0.16)\times 10^{10}\ \text{m}^{-1}$ found at the disk-center. 
This translates to convective velocity values of $\langle v_{\text{conv}}^2 \rangle = 3.15\ (\text{km/s})^2$ and $2.20\ (\text{km/s})^2$, respectively, an unexpected inversion indicating complex convective dynamics. 
For line asymmetry in the $0.3-0.6$ range, bisector slopes decrease with wavelength in the integrated spectrum, with limb darkening causing non-uniform shifts. 
Analysis of line depth-dependence on excitation potential $(2.5$ - $5.0)$ eV shows a maximum displacement at $-200$ m/s for the integrated flux, shifting to $0$ m/s for the quiet sun atlas. 
This work provides a foundational dataset and suggests the development of an adaptive algorithm for future spectroscopic studies.
\end{abstract}
\renewcommand{\abstractname}{Acknowledgements}

\begin{abstract}

% When I was younger, a girl who dreamed of the stars, achieving this goal once felt like a distant image.
% One rarely allowed myself to believe in, I was afraid of throwing away my future on a fantasy. 
% However, every professor and colleague I encountered at the university granted me a new perspective and the courage to persevere in the pursuit of my dreams.
% This works stands as a testament to their belief in me.

% I extend my sincere gratitude and appreciation to my advisor, Professor Benjamin.
% I am deeply thankful not only for his guidance through this project but also, the encouragement he offered during a challenging period in my life.
% I am really grateful for the responses to the questions of a first-semester girl who just wanted to look at the stars and learn about spectrographs, always driving my curiosity to observe more the sky. 
% Literally, for his mentorship along my entire degree, listening to my projects, ideas, and even my mistakes. 
% And even more important, for the confidence to believe in myself.

% I wish to offer a special tribute to Toto, for her profound belief in me, which was the decisive touch I needed to embark on this career.
% Although she is now among the stars, I hope she witnesses the great things she always knew I could achieve.

% To my sister, with the hope that she may see in my achievement a reflection of her own potential to accomplish great things.

% To Daniel, my partner, for his unconditional love and support.
% He was a optimistal believer in my potential, even in my moments of uncertainty, and his support empowered me to pursue paths I once thought beyond my reach.

% To my friend Thomas, for his unwavering support and companionship throughout my degree, standing by me even during the most arduous moments.
% I thank him for his presence and for the many shared coffees that lightened the path.

% To all my family and friends, and to those no longer with us, who provided me with the strength to undertake this endeavor, thank you.

% Finally, to my younger self: \textit{We did it. We can see the stars.}
\end{abstract}
%---------------------------------------------------------------------- TABLE OF CONTENTS
\tableofcontents
%---------------------------------------------------------------------- LIST OF FIGURES
\cleardoublepage\addcontentsline{toc}{chapter}{\listfigurename}
\listoffigures
%---------------------------------------------------------------------- CHAPTERS

\customchapter{Introduction}{The solar granulation patterns}

For decades, the solar spectrum has served as the fundamental reference point for spectroscopic analysis. 
As our nearest star, the Sun enables detailed studies of stellar composition. 
However, advances in optical instrumentation have revealed previously undetected spectral details, providing new insights into fundamental solar properties
One key insight is the granulation pattern caused by convective motion, which is characterized by three distinct hydrodynamic signatures: Line broadening, line bisector asymmetry, and the line-depth dependence of convective blueshift.

Inspired by David Gray's foundational research (\cite{Gray_2009} and references therein), this project aims to extract the granulation pattern from the IAG Solar Flux Atlas to calculate relative velocities and perform a detailed analysis based on the three signatures of convection. 

\section{Convective motion}

The stellar spectrum serves as astronomy's primary source of information, encoding details about a star's composition and velocity. 
However, in some stars this spectrum is modified by fluid motions caused by density variations from temperature fluctuations in the outermost layer, where each spectral line exhibit a distinct velocity shift. 
These persistent convective motions generate a granular structure in the solar photosphere, a phenomenon supported by extensive research (see~\cite{Hamilton_1999, Dravins_1981, Gray_2009}).

The granulation was first observed as a granular structure in movement by Janssen in 1885. 
Later, in 1901, Plaskett associated this pattern with the convective cells observed in Bénard's experiment~\cite{Foukal_1990}, where fluids heated from below produces rising elements of hot gas convecting heat to the surface.
This characteristic configuration gives the name \textit{granulation} to all observable signs of solar convection.

\subsection{The three signatures of convection}
Convective motion produces three distinct signatures of granulation in stellar spectra: Line broadening, line bisector asymmetry, and line-depth dependence of convective blueshift~\cite{Gray_Pugh_2012}.
This project focuses on the third signature, which exhibits a correlation between line depth and relative velocity. 
The method for quantifying this relationship is known as the \textit{granulation pattern}, which means the plot of relative velocities against line depth~\cite{Gray_2009}.

The physical origin of this pattern lies in the convective process itself. 
When the Sun pushes material up through its outer layer, the spectrum exhibits a blueshift. 
As this material subsequently cools and falls back through the atmosphere, it produces a redshift but emits less light, making the blueshift dominant.
Neutral iron lines (Fe I) are optimal for calibrating this phenomenon due to their high abundance, minimal thermal broadening, and limited isotopic variation.
This approach offers the additional advantage of deriving natural wavelengths from a single source, thereby eliminating potential discrepancies.

The study of granulation patterns has been significantly documented by David Gray, whose work has improved the precision of stellar radial velocity measurements (\cite{Gray_2009} and references therein).
This improvement stems primarily from his observation that granulation patterns in solar-type stars closely resemble the solar pattern, differing primarily by a scaling factor. 
This is particularly significant given the considerable challenges of obtaining such precise measurements for other stars, which are often affected by spectral noise, stellar proper motions, and velocity uncertainties~\cite{Gray_2009}.
Furthermore, analyzing solar granulation patterns is crucial for validating photospheric hydrodynamic models~\cite{Dravins_1981}, enabling improved calibration and testing of dynamic atmospheric models.

Following the motivation of treating the Sun as any other star, which does not have the proximity for spectra study, the most accurate solar flux atlas is necessary.

\section{IAG Solar Flux Atlas}

In 2016, Reiners and colaborators published the unprecedented precision \textit{Institut für Astrophysik Göttingen (IAG)} Solar Flux Atlas obtained with the FTS \textit{Fourier Transform Spectrograph} at Göttingen, simultaneously reporting convective blueshifts for a sample of Fe I lines. 
This atlas provides highly precise and accurate data, with radial velocity uncertainties on the order of $\pm 10 m/s$ across the wavelength range of $4050$ to $10650$ $\mathring{A}$. 
In contrast to other FTS atlases, the entire visible wavelength range was observed simultaneously using only one spectrograph setting~\cite{Reiners_2016}.

Despite the exceptional quality of the spectrum, the first derived granulation pattern appeared notably scattered and noisy. 
This was attributable to a rudimentary line position measurement methodology and a poorly line selection, which included numerous blended features, outdated wavelength references, and incomplete spectral coverage.
Consequently, the full potential of the IAG atlas for determining precise convective blueshifts remains unrealized, highlighting the need for the refined analytical methods applied in this project.

\section{IAG Spatially Resolved Quiet Sun Atlas}
In 2023, Ellwarth and colaborators~\cite{Ellwarth_2023} published the \textit{Institut für Astrophysik Göttingen (IAG)} Spatially Resolved Quiet Sun Atlas, also obtained with the FTS \textit{Fourier Transform Spectrograph} at Göttingen. 
This atlas has the advantage of its coverage from the disc center ($\mu = 1.0$) towards the solar limb ($\mu = 0$), where $\mu =\cos(\theta)$.
This spatial resolution enables the study of how convective blueshifts vary across the solar disc due to changing projection angles, a dependency that cannot be observed in other stars.

The principal objective of this project is to characterize the solar granulation pattern by treating the Sun as any other star. 
To achieve this, the disk-integrated IAG Solar Flux Atlas serves as our primary data source. 
However, a precise analysis of the relative velocities in this integrated light requires correcting for the center-to-limb variations that are uniquely quantified by the Spatially Resolved Quiet Sun Atlas.

\section{Motivation}

In the first semester of 2025, Manuel Fuentes a physics student at the Universidad de los Andes, implemented these improvements in a computational project focused on the visible spectral range. 
By developing enhanced measurement techniques and employing a carefully curated line list that fully covers the visible range with updated wavelength references, Fuentes achieved significantly sharper granulation patterns than previous analyses. 
This work demonstrated that proper line selection and modern wavelength standards can reliably extract convective signatures from high-quality solar spectra.

Under the guidance of Professor Benjamin, this project extends the analysis into the near infrared range.
This region contains spectral lines that originate from deeper photospheric layers.
Although these lines are consequently weaker, they represent a rich source of information. 
However their analysis requires an adaptation of the existing measurement methodology. 
A key objective was to adapt these methods to enable the characterization of solar dynamics and line asymmetries via granulation patterns in the near infrared range.

By addressing these challenges, this project aims to produce a robust characterization of the solar granulation pattern and new insights into the photosphere’s dynamic. 
These efforts are guided by the central research question: What are the direct spectroscopic consequences of solar dynamics?



\customchapter{Literature Review}{Convective motion in the Sun}

As previously mentioned, David Gray has significantly advanced the study of granulation patterns in the solar photosphere, with a particular focus on measuring relative velocities with high precision.
This chapter explores the physical origins of the three signatures of convective motion in the solar photosphere, and how this reveals the hydrodynamics on the outermost layer. 

\section{The solar interior and the solar outer atmosphere}

The Sun is classified as a yellow dwarf star of spectral type G2V.
Its chemical structure is primarily composed of a large fraction of ionized hydrogen and a smaller proportion of helium.
What makes the Sun unique in astronomical studies is its proximity to Earth, which allows for detailed observation unmatched by any other star.
Structurally, the Sun is divided into two main regions: The solar interior and the solar outer atmosphere. 

As illustrated in Figure~\ref{fig:Sun interior} the overall structure of the solar interior is core, radiative and convective zone. 
In the core, He nuclei are built from H nuclei in the proton-proton chain as Equation~\eqref{eq:pp chain} refers.
\begin{equation}
    4^{1} \text{H} \rightarrow ^4\text{He} +2e^++2\nu +26.7 \text{MeV} 
    \label{eq:pp chain}
\end{equation}
The proton-proton chain reaction in the core liberates approximately $26.7$  MeV of energy in the form of high energy $\gamma$-rays, and $0.5$  MeV of energy in the form of neutrinos. 
In this zone, standard models estimate a temperature of $1.6\times 10^7$  K and density to $1.6\times 10^5$  kg/$\text{m}^3$.
Moving outward through the layers, both the density and temperature decrease significantly, as the energy is slowly transferred outwards by radiative diffusion~\cite{Foukal_1990}.
This process progressively shifts the wavelength of the radiation from high energy $\gamma$-rays to the visible light that eventually escapes.
The large temperature combined with the high density, allows the absorptions and remissions of photons that make this zone highly opaque and maintain the central material in a plasma state, functioning like a massive nuclear reactor.

\begin{figure}[H]
     \centering
     \begin{subfigure}{0.48\textwidth}
         \includegraphics[width=\textwidth]{Sun interior.jpg}
         \caption{The interior structure of the Sun. The convection zone is responsible for the general movement that characterizes the granulation patterns.}\label{fig:Sun interior}%
     \end{subfigure}
\hfill
     \begin{subfigure}{0.48\textwidth}
         \includegraphics[width=\textwidth]{Sun exterior.jpg}
         \caption{The outer structure of the Sun. The photosphere is the layer of the Sun where the convection cells overshoot from the convection zone.}\label{fig:Sun outer atmosphere}%
     \end{subfigure}

        \caption{The general structure of the Sun. Images modified from~\cite{Priest_1982}.}\label{fig:Sun structure}%
\end{figure}

On the other hand, Figure~\ref{fig:Sun outer atmosphere} illustrates the overall structure of the solar outer atmosphere consisting of the photosphere, chromosphere and corona. 
In these layers, the density decreases rapidly with height above the solar surface. 
However, the temperature decreases to a minimum of approximately $4300$  K in the upper photosphere before rising through the chromosphere and transition region to millions of degrees in the corona.
From that point, the temperature falls slowly expanding outwards as the solar wind. 

Nevertheless, the relevant layers for this project are the photosphere, a thin layer of plasma that emits most of the solar radiation; and the convection zone, in which all the convection process take place.
The radiation chain process results on the emission of a continuous spectrum passes through the overlying photosphere.
Then specific wavelengths are absorbed for this layer, resulting in the characteristic Fraunhofer lines superimposed on the emitted spectrum, which allows observing the convection consequences~\cite{Priest_1982}.

\section{The solar convection zone}

In 1885 Janssen obtained the first clear photograph of photospheric granules (see Figure~\ref{fig:Janssen photography}) providing the initial evidence and the starting point for numerous studies about granulation~\cite{Malherbe_2022}.

\begin{figure}[H]
    \centering
    \includegraphics[width=0.45\linewidth]{First take photosphere.jpg}
    \caption{The first clear photograph of the photosphere where the granulation pattern is visible taken by Janssen in 1885. Image taken from~\cite{Malherbe_2022}.}\label{fig:Janssen photography}%
\end{figure}

Following the history, in 1930, Unsöld theorized that the layers beneath the photosphere should be convective unstable~\cite{Foukal_1990}. 
This hypothesis was later supported by Plaskett when he related the observed granules to the convective cells studied in Bénard's laboratory experiments~\cite{Plaskett_1936}.
In this analogy, a fluid heated from below develops rising elements of hot gas that transport heat to the surface.

The elements of hot gas rising transporting heat are called \textit{convective cells}, and the pattern generated by several cells on the photosphere is the \textit{granulation} with each individual region referred to as a \textit{granule}. 
Typical granules span approximately 700 km and have short lifetimes, lasting between five to ten minutes.

In the solar context, convection takes place in a highly compressible and stratified gas located between $0.86 R_{\astrosun}$ and the surface, affected by a large temperature gradient.
This physical regime leads to determine the conditions required for convection to occur and the resulting dynamics of the granules~\cite{Foukal_1990}. 

\subsection{The Schwarzschild criterion}
Consider an elementary parcel of material displaced in local hydrostatic equilibrium with its surroundings, characterized by radial profiles of pressure $P(r)$, density $\rho(r)$, and temperature $T(r)$.
If the granule temperature is increased to a value $T'$, it will expand adiabatically to maintain pressure equilibrium, thereby decreasing the density relative to its surroundings (see Figure~\ref{fig:parcel argument}).

\begin{figure}[H]
    \centering
    \includegraphics[width=0.7\linewidth]{Parcel argument.jpg}
    \caption{Diagram for the parcel of material displaced so slowly that the only force it feels is the pressure in a direction parallel to itself, keeping it in a constant horizontal movement. Image taken from~\cite{Priest_1982}.}\label{fig:parcel argument}%
\end{figure}

This convective cell experiences a buoyancy force, causing it to rise. 
The buoyancy force persists until the granule's density matches that of its new surroundings after traveling a length $l$.
Let $T_n'$ be the temperature of the rising element and $T_n$ the temperature of its new surroundings. 
The difference between the adiabatic gradient of the element and the radiative gradient of the surroundings governs the convection, as described by Equation~\eqref{eq:adiabatic radiative gradients}
\begin{equation}
    T_n = T + \parens{\frac{dT}{dr}}_R l \quad \big| \quad T_n' = T' + \parens{\frac{dT}{dr}}_{ad} l
    \label{eq:adiabatic radiative gradients}
\end{equation}
Where $R$ refers for radiative temperature gradient and $ad$ for the adiabatic temperature gradient. 
The onset of convection leads to the inequality~\eqref{eq:radiative condition}, where two conditions can arise: The granulation is established when adiabatic gradient exceeds the radiative gradient; otherwise, the layer is stable and energy is transported by radiation.
\begin{equation}
    -\parens{\frac{dT}{dr}}_R > \parens{\frac{dT}{dr}}_{ad}
    \label{eq:radiative condition}
\end{equation}
If the granulation is established, the element continues to expand adiabatically as it rises, driven by buoyancy.
Otherwise, if the layer is stable, the element will contract, becomes heavier than its surroundings and begins to move down toward its original position.

This onset of instability, when the vertical temperature gradient is too large, is known as the Schwarzschild criterion for convection.
Conveniently this criterion is expressed in terms of the relation between $T$, $P$ and $\gamma$ (heat capacity ratio) for an adiabatic change (see Equation~\eqref{eq:Schwarzchild criterion}).

\begin{equation} 
    -\parens{\frac{dT}{dr}}_R  > \frac{\gamma - 1}{\gamma} \parens{\frac{T}{P}}\parens{-\frac{dP}{dr}}_{ad}
    \label{eq:Schwarzchild criterion}
\end{equation}

In this form, the condition establishes that convection can occur when opacity rises rapidly, due to increasing the population of $n=3$ level of hydrogen and $\gamma$ lowered by ionization.
The material and energy transported by this process ended on the low photosphere, where the granules exhibit different properties that will be examined in subsequent sections.

\section{The solar photosphere} 
As mentioned in the previous section, the observation of this layer leads to the study of dynamics and reactions within Sun's outermost layer. 
From surface observations a distinct pattern of granules with dynamic behavior is apparent, where individual cells continuously emerge and disappear (see Figure~\ref{fig:photosphere visual}).

\begin{figure}[H]
    \centering
    \includegraphics[width=0.65\linewidth]{Granulation pattern.jpg}
    \caption{A view of granulation on the Sun's surface. Image taken from~\cite{Samir_pattern}.}\label{fig:photosphere visual}%
\end{figure}

The bright areas of granules correspond to regions where hot gas rises through the solar atmosphere.
As this gas releases energy in the form of photons at the photosphere, it cools and subsequently descends, creating the darker regions of intergranular lanes~\cite{Carroll_Ostlie_2006}.
Furthermore, high-resolution observations reveal that these granules are in continual motion generating asymmetries in absorption line profiles (see Figure~\ref{fig:motion granules}).

\begin{figure}[H]
    \centering
    \includegraphics[width=0.65\linewidth]{Granules motion.jpg}
    \caption{A time sequence showing granule evolution where the time intervals are about a minute. Image taken from~\cite{Foukal_1990}.}\label{fig:motion granules}%
\end{figure}

\subsection{Static photosphere: Limb darkening phenomenon}

EXPLICCAR BREVEMENTE EL NUEVO MODELO, EL TRANQUILO, Y PORQUE S EPUEDE HABLAR DE OSCURECIMIENTO

Because the temperature decreases outward through the photospheric layers, the observed intensity falls off towards the solar limb. 
Discovered by Halm in 1907~\cite{Dravins_1981}, this effect is known as \textit{limb darkening}, which makes the disk intensity profile to appear more squared at increasing wavelength (see Figure~\ref{fig:squared profile}).

\begin{figure}[H]
    \centering
    \includegraphics[width=0.6\linewidth]{Limb darkening.jpg}
    \caption{Squared profile for the disk intensity at increasing wavelengths, where $5\mu m$ refers to the infrared range and $0.32\mu m$ the violet range. Image taken from~\cite{Foukal_1990}.}\label{fig:squared profile}
\end{figure}

The analysis of this effect provides a direct technique for determining the temperature structure of the photosphere as a function of line depth. 

For typical weaker lines, the convective blueshift diminishes toward the limb, with a net velocity change approximately of $400$  m/s.
As explain Ellwarth et al.~observations closer to the limb pass through the atmosphere a shallower angle, resulting in longer optical paths through high atmospheric layers which allows the study of the layers where convective blueshift is less pronounced~\cite{Ellwarth_2023}.
Because of this phenomenon and following the objective of study the line depth-dependence of convective blueshift, the center disk flux spectrum was taken as reference for analysis, where the limb darkening effect is neligible.

\subsection{Dynamic photosphere: The C-curved profile bisector.}
Analysis of changes, contrast and velocity field in the granulation structure have been inferred indirectly from observations of Fraunhofer line profile shapes~\cite{Foukal_1990}.
The observations on absorption lines reveals that velocity of a rising granule decays less rapidly than its excess brightness, resulting in a characteristic C-curved line profile bisector (see Figure~\ref{fig:c curved profile}).

\begin{figure}[H]
    \centering
    \includegraphics[width=0.75\linewidth]{C curved profile bisector.jpg}
    \caption{The C-curved line profile bisector with the corresponding diagram of wavelength displacement due to convective blueshift. Imagen taken from~\cite{Dravins_1981}.}\label{fig:c curved profile}%
\end{figure}

The formation of the C-curved line profile bisector occurs in three stages, corresponding to different heights in the photosphere.

First, the deepest part of the line profile is formed higher up, in a region of decelerated upflow, producing a smaller blueshift.
Then, the mid-depth portion is formed in the brightest upflowing material, resulting in a blueshift.
Finally, the line profile wings where the opacity is lowest, tend to be formed deepest in the cool material, producing a redshift.

This dynamic process induces characteristic asymmetries on the line profile bisector, which becomes an important instrument to measure convection process in the solar atmosphere.

\section{The three signatures of convection}
The three signatures of convection in stars are described by David Gray in his research as the principal characteristics to identify the convective motion through the spectrum (see~\cite{Gray_2009,Gray_Oostra_2018,Gray_Pugh_2012} and references therein).

\subsection{First signature of convection: Line broadening} 
To explain the line broadening is necessary establish the process that creates a spectral line. 
An individual atom making a transition between energy levels emits a photon with certain frequency.
This transition can be represented as a graph of radiance or intensity per unit wavelength against wavelength, what is called \textit{line profile}~\cite{Carroll_Ostlie_2006}.
The radiated intensity can be modeled passing through a hot cloud of gas in thermal equilibrium as Equation~\eqref{eq:radiance absorption line}.
\begin{equation} 
    I_{\nu}(\tau_{\nu}) = I_0 e^{-\tau_{\nu}} + B_{\nu}\parens{1-e^{-\tau_{\nu}}}
    \label{eq:radiance absorption line}
\end{equation}
Where $\tau_{\nu}$ refers to the line depth and $B_{\nu}$ the absorption coefficient for the gas. 
The dynamical and atomic processes on the photosphere causes the thermal, convection, rotation, pressure, Stark, Zeeman and natural broadening effects.
Because we measure the broadening of the line cores, which is affected mostly by Doppler effects, whereas pressure and natural broadening affect the wings, and Zeeman is negligible except in sunspots; only the thermal, rotation and convection broadening effects were studied. 

Atoms in a gas have random motions with temperature dependence, which mean speed is obtained by the relation between kinetic and thermal energy for gasses~\cite{Van_1965}. 
The fraction of atoms in a speed interval between $v$ and $v + \Delta v$ is then given by the Maxwell-Boltzmann distribution in Equation~\eqref{eq:Gaussian distribution}.
\begin{equation} 
    f(v_{\text{rad}}) = \exp\parens{\frac{-mv_{\text{rad}}^2}{2k_{B}T}}
    \label{eq:Gaussian distribution}
\end{equation}
Comparing with a typical Gaussian distribution centered on the origin, we can relate the width $(\sigma^2)$ to the variance of the radial velocity (see Equation~\eqref{eq:velocity variance}).
\begin{equation} 
    	f(x)= \exp\parens{\frac{-x^2}{2\sigma^2}} \quad \rightarrow \quad \langle v_{\text{rad}}^2 \rangle = \sigma^2 = \frac{k_{B}T}{m} 
    \label{eq:velocity variance}
\end{equation}
Using the relation of Doppler effect for the radial velocity and relation~\eqref{eq:velocity variance} the line profile with only Doppler broadening effect is described by Equation~\eqref{eq:Doppler broadening}.
\begin{equation} 
    f(\Delta \lambda) = L_D\exp\parens{\frac{-mc^2}{2\lambda^2k_{B}T}\Delta \lambda^2}
    \label{eq:Doppler broadening}
\end{equation}
Where $L_D$ refers to the line depth, $k_B$ to the Boltzmann constant; and $c\Delta \lambda/\lambda$ to the radial velocity of the observed atom.
The Equation~\eqref{eq:core curvature} describes the line core curvature can be quantified by the second derivative of the line's intensity profile with respect to wavelength, evaluated at observed wavelength.
\begin{equation}
    \lambda_{obs}^2 \parens{\frac{d^2 f(\lambda_{obs})}{d \lambda_{obs}^2}}
    \label{eq:core curvature}
\end{equation}
On a plot of $|f''(0)|\lambda^2/L_D$ against line depth, a theorical slope can be derived from Equation~\eqref{eq:Doppler broadening} using the definition of line core curvature as shows the Relation~\eqref{eq:theory slope only doppler}.
\begin{equation}
    |f''(0)| = L_D \parens{\frac{mc^2}{2\lambda^2k_{B}T}} \quad \rightarrow \quad \frac{|f''(0)|\lambda^2}{L_D} = \frac{mc^2}{2k_{B}T}
    \label{eq:theory slope only doppler}
\end{equation}
Where $\Delta \lambda=0$ due to the origin-centered Gaussian profile.
Equation~\eqref{eq:theory slope only doppler} represents the line core curvature slope for lines which are only affected by the thermal broadening effect.
Directly, assuming the three target effects as independent\footnote{This is not completely true, for it is known that greater temperatures lead to more negative convective velocities, implying a correlation between thermal and convective speeds. For simplicity, the approximation was taken on the project.} the variance of the total radial velocity is the sum of the variances of the thermal, rotational and convective effects.
This leads to the Equation~\eqref{eq:Theory slope} of a theoretical slope including the three broadening effects.
\begin{equation}
    \frac{|f''(0)|\lambda^2}{L_D} = \frac{c^2}{\langle v_{\text{r}}^2 \rangle + \langle v_{\text{T}}^2 \rangle + \langle v_{\text{conv}}^2 \rangle}  
    \label{eq:Theory slope}
\end{equation}
Where $\langle v_{\text{r}}^2 \rangle$ refers to the variance of the rotation velocity; $\langle v_{\text{T}}^2 \rangle$ is the variance of the thermal velocity; and $\langle v_{\text{conv}}^2 \rangle$ refers to the variance of the convection velocity.
Furthermore, the value for Fe atom mass $55.85$ g/mol; the solar effective temperature $5770$ K leads to the values of thermal velocity variance of $0.86$ $(\text{km}/\text{s})^2$.

The variance of rotation velocity is $0.90$ $(\text{km}/\text{s})^2$, this result was derived by professor Benjamin using a spherical and solid model of the Sun.
In the IAG spatially resolved quiet Sun atlas the value of $\langle v_{\text{r}}^2 \rangle$ is neligible.
With these values, an approximation of convection variance of velocity is expected to be calculated for each atlas.

\subsection{Second signature of convection: Line profile asymmetry}
According to Kirchhoff's laws, absorption line formation requires lower temperature conditions, which are found precisely in the Sun's outermost atmospheric layers~\cite{Carroll_Ostlie_2006}. 
These regions not only provide the appropriate temperatures for absorption but also exhibit comparatively higher opacity. 
Those spectral lines from Fe I are particularly valuable for solar granulation studies due to their high abundance, minimal thermal broadening, limited isotopic variation and availability of accurate laboratory measurements of natural wavelengths~\cite{Nieminen_2017}. 
Due to the useful characteristic of this line dataset, asymmetries imprinted can be quantified by analyzing their bisectors. 

A convenient method for this measures involves using the third derivative of the line profile, which provides the slope of the lowest end of the bisector (see Equation~\eqref{eq:third derivative relation}). 
\begin{equation}
    -\frac{c}{\lambda_{obs}}\parens{\frac{1}{3 \frac{d^2 f(\lambda_{obs})}{d \lambda_{obs}^2}}}\parens{\frac{d^3 f(\lambda_{obs})}{d \lambda_{obs}^3}}
    \label{eq:third derivative relation}
\end{equation}

The slope of the line profile is defined as zero when the line profile bisector is vertical and the line core is symmetric.
The relation~\eqref{eq:third derivative relation} is derived in appendix\ref{ap:third derivative}.

As mentioned before, line profile asymmetries are an important instrument to measure convection processes from the solar atmosphere.
Since most of stellar observations are made with lower-resolution spectrographs and often lower signal-to-noise ratios. 


\subsection{Third signature of convection: Line depth-dependent wavelength shifts}
Many studies across the years have detected and observed the phenomenon of wavelength shifts against the line depth, or as it is called, convective blueshift.

\subsection{The third signature plot}
The third signature plot relates relative velocity against line depth, as shown Figure~\ref{fig:granulation pattern ellwarth}.

\begin{figure}[H]
    \centering
    \includegraphics[width=0.65\linewidth]{Granulation pattern Ellwarth.jpg}
    \caption{The third signature plot for the IAG spatially resolved quiet Sun atlas, shows a strong trend blueshift in the shallow lines. Image taken from~\cite{Ellwarth_2023}.}\label{fig:granulation pattern ellwarth}%
\end{figure}

Notice in Figure~\ref{fig:granulation pattern ellwarth} the wavelength-dependence on convective blueshift, which has been extensively documented for several datasets.
The significance of the third signature plot lies in its universality for solar-type stars; their plots closely resemble the Sun's, differing primarily by a scaling factor~\cite{Gray_Pugh_2012}. 
Consequently, a detailed analysis contributes to the understanding and radiation of photospheric hydrodynamic models~\cite{Dravins_1981,Gray_2009}.
However, a precise measurement of this pattern requires a understanding of the convective blueshift phenomenon.

\subsection{Convective Blueshift}
The measured negative redshift resulting from convective motions is known as \textit{convective blueshifts}, which is measured using the Doppler effect (see Equation~\eqref{eq:relative velocity}). 
\begin{equation}
    v_r \approx c \parens{\frac{ \lambda_{obs}- \lambda_{em}}{\lambda_{em}}} -633\text{m/s}
    \label{eq:relative velocity}
\end{equation}
Where the value of $633$ m/s is the correction of gravitational redshift for the Sun; and the $\lambda_{em}$ represents the emitted wavelength.

When the Sun pushes material up through its outer layer, the spectrum exhibits a blueshift. 
As this material subsequently cools and falls back through the atmosphere, it produces a redshift, but emits less light, making the blueshift dominant.
Since the strength of the convective distortions and shifts of spectral lines vary across the H-R diagram, we expect systematic errors in radial velocities~\cite{Gray_2009}.

\subsection{Chromodependence characterization}
Recent emphasis on measuring the third signature plot has led to new interpretations of line depth-dependent wavelength shifts.

Gray and others authors have cualitatively explained the line-depth dependence of convective blueshift.
Shallow lines come preferably from deep photospheric layers where convection is strong, so the convective blueshift is great.
Whereas deep lines come preferably from superficial layers, where gravity and the demise of buoyancy has slowed the convection and the blueshift is small.
However, an explanation of why this diagonal trend depends on color, or chromodependence of the line depth, has not been found.

In 2018, Gray and Oostra established a standard curve determined by a third order polynomial fit to the solar granulation pattern (see Figure~\ref{fig:Gray standard curve}).

\begin{figure}[H]
    \centering
    \includegraphics[width=0.6\linewidth]{Images/Standard curve gray.jpg}
    \caption{Standard curve proposed by Gray and Oostra on the spectral range of $\SI{6020}{\angstrom}$ to $\SI{6340}{\angstrom}$. Image taken from~\cite{Gray_Oostra_2018}.}\label{fig:Gray standard curve}%
\end{figure}

Nevertheless, the authors limited the spectral range of $\SI{6020}{\angstrom}$ to $\SI{6340}{\angstrom}$, avoiding the chromodependence which becomes evident in a wider spectral range.
This limitation motivates the present work to develop a characterization taking into account all wavelength ranges; and give the methodology for dealing with this phenomenona.

A separate theoretical perspective, offered by Hamilton and Lester, attributes aspects of photospheric dynamics to rotation.
The pronounced differential rotation with latitude observed seems to be the result of convective flows driven radially by the buoyancy force and deflected horizontally by the Coriolis force~\cite{Foukal_1990}.

\section{Anomalous chromodependence}
In previous research many authors pointed out the phenomenon that we call chromodependence or wavelength dependence of the signatures, emphasizing on the third granulation plot. 
Because this definition can be ambiguous a clarification is presented below.

In the photosphere model the temperature is higher in the deepest layers than the surface. 
For the formation of an absorption line, the temperature provides the atom with the energy necessary to be prepared in the lowest energy level. 
This is defined as the excitation potential $(\chi)$. 
While lines with higher values of $(\chi)$ are formed in the deepest layers, lines with smaller values of $(\chi)$ are formed near the surface. 

However, we recall the fact mentioned about the opaqueness of the photosphere, the lines formed on the surface are stronger than those produced in deeper layers.   
This results in absorption lines in the violet range, produced when the atom absorbs a high value of energy, formed near the surface implying a lower excitation potential and a small temperature. 
This is a chromodependence, but is not unexpected in the granulation patterns.

In this scheme, atomic and photospheric structure combine to make blue lines stronger and red lines weaker. 
This might be called \textit{Normal chromodependence} and should be characterized by a unique granulation curve, because weaker lines experience also a stronger convection.
What this project tried to characterize is the anomalous chromodependence, the observational fact of granulation patterns showing a different granulation curve for every wavelength range.


\customchapter{Methodology}{The blend-free list of Fe I lines}

As the spectral analysis is complicated cause the convective motion, I take an computational approach leading an observational requirement to lead the research.
This leading by the experience of professor Benjamin and the previous research.
This research endeavor encompass both analytical and computational components. 
For the last part, the programming language Python and Jupyter Notebook was used.
The computational aspect focus on identifying the granulation pattern within the solar spectrum by calculating relative velocities using the wavelengths of Fe I. 
All analysis data and code used in the process was uploaded to a GitHub repository, allowing anyone interested to reproduce the results and verify the authenticity of the conclusions presented. 
Moreover, proper credit will be given to all previous work from other researchers.

\section{Methodology}
We follow the methodology established in previous studies \cite{Dravins_1981,Reiners_2016,Ellwarth_2023} , which utilized a selected list of Fe I lines. 
These lines are ideal for this calibration due to their minimal thermal broadening and reduced susceptibility to other atmospheric affectations.

The line up for the code was identify the Fe I lines in the IAG Solar Flux Atlas and the IAG Spatially Resolved Quiet Sun Atlas using
the blend-free list of Fe I lines.
Second, fit a fourth-grade polynomial fit due to the c-curved line profile bisector and find the observed wavelength.
Then calculate the Doppler velocity, convective blueshift and flux with the fit.
Finally, find the second derivate (core curvature) and the third derivate relation (bisector slope) for the curvature in the observed wavelength. 
This last is develop in detail in the apendix B.


\section{Blend-free Fe I line list}
The previous methodology implemented the Nave et. al list of laboratory measured Fe I lines \cite{Nave_1994}. 
This list classifies lines with a quality rating (A,B,C,D) with A be the most and best quality.
However, not all the listed lines are clearly present in the solar spectrum, and within the near-infrared range, many lines are severely mixed.

In collaboration with Professor Benjamin and Manuel Fuentes, we refined this list. 

\subsection{Selection method for Fe I lines}
This selection method has two different approaches: Computational and visual.
First, we selected only quality A lines and then performed the next filters according to the statement of having an fourth order polynomial fit for the line core.
\begin{itemize}
\item The coefficient of the fourth grade term need to be positive and not too small cause it denotes a curve too big. This can be for a line, specially in the infrared, cause the big cores are related to atmospheric lines.
\item The difference between extreme points of the fit need to be less than the half of the distance, that is a curve and not a slope.
\item The absolute difference between the wavelength observed and the one from Nave list need to be less to $0.025\mathring{A}$.
\end{itemize}

The last part was a visual inspection to discard line mixes or absent from the solar spectrum. This part was divided between porfessor Benjamin and me.
I used the visualizer (explained in detail in the apendix C) and profesor Benjamin use the great years of experience.
Then I combined the list, made a check and profesor Benjamin check again.
The discard part was only using the geometry of the curve and guidance of profesor Benjamin (and God).
%Insert images for examples
% \begin{figure}[H]
%     \centering
%     \includegraphics[width=0.6\linewidth]{Granulation pattern.jpg}
%     \caption{The first clear photgraphy of the photosphere where is visble the granulation pattern. Image taken from \cite{Malherbe_2022}}
%     \label{Janssen photography}
% \end{figure}
The new blend-free list improves our results, leading to cleaner graphics and characterization with less scatter in the data.
Then we perform different graphics for our analysis conform we see it necesary for the characterization.

In case of the IAG Spatially Resolved Quiet Sun Atlas we only use the $\mu=1$ for the effect to eliminate the limb darkening effect in the spectrum.

\customchapter{Results and discussion}{The three signatures of convection}

Our results can be summarized in three principal aspects: Chromodependence on the granulation pattern along the line depth; a detailed view of the characteristic curvature, asymmetries, and sharpness of spectral lines and higher-quality graphs with reduced scatter.

\section{The third signature: Chromodependence on the granulation pattern}
We obtain the granulation patterns for the IAG Solar Flux Atlas for all the wavelength.
\begin{figure}[H]
    \centering
    \includegraphics[width=0.6\linewidth]{GranulationPattern_VIS.png}
    \includegraphics[width=0.6\linewidth]{GranulationPattern_NIR.png}
    \caption{Granulation pattern for the Solar Flux Atlas.}
    \label{GP VIS and NIR}
\end{figure}
The behavior along the line depth is according to literature, in which is clear the chromodependence. 
For a different try into the characterization of this behavior we try to genearte different second order polynomial fits for each color range.
\begin{figure}[H]
    \centering
    \includegraphics[width=0.6\linewidth]{Chromocharacterization_VIS.png}
    \caption{Granulation pattern for the Solar Flux Atlas with color curves.}
    \label{Curves Solar Flux}
\end{figure}
But this is not standard. 
So, we performed an analysis of line depth against wavelength. Our hypothesis was: “If the chromodependence is present only in the Solar Flux Atlas, then rotation could be the cause of this phenomenon”
Surprisly, we observed a chromodependence in both spectral datasets. 
We start taking the granulation pattern of the center disk
\begin{figure}[H]
    \centering
    \includegraphics[width=0.6\linewidth]{GranulationPattern_SPA.png}
    \caption{Granulation pattern for the Spatially Resolved Quiet Sun Atlas at $\mu=1$.}
    \label{GP SPA}
\end{figure}
This was initially unexpected because the rotation is negligible at the disc center.
For the measurement of its velocity we took a 4300-5600 A range and sorted all lines from both atlases into 50 m/s velocity bins.
\begin{figure}[H]
    \centering
    \includegraphics[width=0.6\linewidth]{V_bins_VIS.png}
    \includegraphics[width=0.6\linewidth]{V_bins_SPA.png}
    \caption{Comparision between atlases with velocity bins for the relation between wavelength and line depth.}
    \label{Velocity analysis}
\end{figure}
For the measurement of rotation in the lines we made a first order polynomial fit for each one.
\begin{figure}[H]
    \centering
    \includegraphics[width=0.6\linewidth]{Velocity bins SPA.png}
    \includegraphics[width=0.6\linewidth]{Velocity bins VIS.png}
    \caption{Comparision between atlases with velocity bins for the relation between wavelength and line depth. The first order fit is showed for each velocity bin.}
    \label{Velocity analysis slopes}
\end{figure}
The clue here is the value of slopes in the Spatially Resolved Atlas is greater than the Solar Flux, which contrindicates for complete the initial hypothesis.
The rotation is not the cause.
 
\section{The first and second signature: Detailed view of line broadening and asymmetry}
We made two graphics corresponding to each signature related to the line bisector and core asymmetries.
The slope of line bisector shows the behavior of the c curved line profile bisector.
\begin{figure}[H]
    \centering
    \includegraphics[width=0.6\linewidth]{CBisector_VIS.png}
    \includegraphics[width=0.6\linewidth]{Cbisector_slopes.png}
    \caption{C bisector graph for slopes in ehich is clear the convection movement and how its affected in the profile.}
    \label{C bisector graph}
\end{figure}
The interesting one is the core curvature but multiplied by the squared wavelength, cause it affirms the chromodependence in the weaker lines as the literature says.
\begin{figure}[H]
    \centering
    \includegraphics[width=0.6\linewidth]{Sharpness_VIS.png}
    \includegraphics[width=0.6\linewidth]{Sharpness_NIR.png}
    \caption{Sharpness of the core bisector in the solar flux atlas separated by range.}
    \label{Sharpness}
\end{figure}
We can see a natural division for lines in the infrared range, which is interesting due to corresponding Teluric lines of absorption in the atmosphere.
Theres is a little division in $11400 A$ who separates the lines i one that follow the mean curve and others than doesnt do that.
\begin{figure}[H]
    \centering
    \includegraphics[width=0.6\linewidth]{Sharpness_ALL.png}
    \caption{Sharpness of the core bisector in the solar flux atlas. Is clear that the infrared follow two different behavior due teluric lines.}
    \label{Sharpness}
\end{figure}
This is the graphic which is clear that we have chromodependence in the weaker lines.

\section{Higher quality graphs}
In comparation of Ellwarth graphics, which is the study most recent, there is a better resolution in the different graphs without scattered points.
\begin{figure}[H]
    \centering
    \includegraphics[width=0.6\linewidth]{ConvectiveBlueshift_VIS_Gray.png}
    \caption{We realized the same graphic for the comparision with the Ellwarth article to show the less scattered points.}
    \label{Scattered points}
\end{figure}

\customchapter{Conclusions}{Wavelength and line depth dependences}

%Por cada objetivo va una conclusion

In conclusion, this project characterized the convective blueshift in solar absorption lines and the anomalous chromodependence present across the three signatures of convection. 
First of all, we produced a refined, blend-free list of Fe I lines and derived solar granulation patterns with minimal scatter. 
This list should be published, with the convective blueshift, curvture, bisector slope and dept of each line.
Alsoinclude other parameters such as ionization potential, probability of transition, and damping parameters.
The analysis statistically justified a window of $\SI{1}{\miliangstrom}$ around the line core as the optimal width for examination, underscoring the accuracy and unprecedented precision of the IAG solar flux atlas. 

Furthermore, we provided a detailed description of how granulation patterns vary with wavelength, establishing color-dependent standard granulation curves for the entire spectrum. 
We found several manifestations of wavelength dependence: The expected fact that, in general, deeper lines are located at shorter wavelengths.
But also several anomalous instances of chromodependence: For a given line depth, short-wavelength-lines have greater convective blueshifts, higher (more positive or less negative) core bisector slopes, and norrower profiles (or more precisely, sharper, cores).
Our results determine that rotation is not the cause of the anomalous chromodependence observed in the third signature plot, and we established a general characterization for the full wavelength range of the atlas. 
For the near-infrared, a general trend line describes the chromodependence, while the line bisector slope shift along the line depth is linear. 
In the visible range, we identified individual granulation curves and a description for the coefficients. 


For future work, it is necessary to study atomic broadening effects and model convection to investigate the role of temperature and density fluctuations.



\begin{appendices}
\chapter{Z-score Standardization}\label{ap:zscore}
In the process of calculate the four order polynomial fit the function np.poly.fit() presents an overstimation on the coefficients, due to the large difference of magnitude order between axis.
To deal with this difference a z-score standardization was used on the selected bins of wavelengths around the observed wavelength. 
This process helps to avoid the dominance of certain features over other due to diferences in their scales \cite{Boyd_2014}.

The follow up for the standardization was applied the relation \eqref{z score} on the selected bins for wavelength.

\begin{equation}
\lambda_{scaled}= \frac{\lambda-\mu(\lambda)}{\sigma(\lambda)}
\label{z score}
\end{equation}

Where $\mu(\lambda)$ refers to the mean and $\sigma(\lambda)$ to the standard deviation of the wavelength range. 
As the wavelength was scaled, in terms of calculated derivates for the first and the second signature, a re-scaled for this values was necessary.
Based on the definition for the standardization, the derivates follow the relation \eqref{re scaled derivates}.

\begin{equation}
 \frac{d}{d \lambda} = \frac{1}{\sigma(\lambda)} \frac{d}{d \lambda_{scaled}}
\label{re scaled derivates}
\end{equation}

Taking the derivate of the expresion \eqref{z score} a factor related to the standard deviation appear.
With this, the original values for derivates evaluated in the observed wavelength are expresed in equation \eqref{2 and 3 scaled derivate}

\begin{equation}
 \frac{d^2}{d \lambda^2} = \frac{1}{\sigma(\lambda)^2} \frac{d^2}{d \lambda_{scaled}^2} \quad \quad \frac{d^3}{d \lambda^3} = \frac{1}{\sigma(\lambda)^3} \frac{d^3}{d \lambda_{scaled}^3}
\label{2 and 3 scaled derivate}
\end{equation}

This improved considerably the precision in the fit and there over the precision on the observed wavelength calculated.

\chapter{The third derivate relation}\label{ap:third derivate}

Or called the bisector slope. 
It was multiplied by the relation ($\frac{c}{\lambda}$) to see each clear in the graphic.

\chapter{Visualizer for outliers}\label{ap:visualizer}

For the process of the blend-free list was created an app using the interface Tkinter with the objective to help the visualization of outliers.
Two versions of the visualizer were created.
One just shows the line core and fourth order polynomial fit as shown in the figure ().

This helps for a first process where far separated lines were discarded.
Then, we can perform the different calculations (core curvature, velocity and bisector slope), and use the second version of the visualizer (see figure ()).

In this version it can visualize the three signatures of convection and the line core with the fit. Moreover, was resalted the corresponding Fe I line on each graph to corroborate the behavior.
Thanks to this software the time expended seeing lines was reduced significantly. Specially cause count with his own system to classification, adding lines to a Dataframe and save the image. Following the motivation we present the software on GitHub \href{https://github.com/ccuellarn/Final-Project}{Repository on GitHub where was published the visualizer of mixed lines.} and its explained below.

\section{Test example}

The main code is in the file Visualizer.ipynb and the test example data are test.xlsx, feel free to change the type of data, the important is make a dataframe where the columns are [Wave , Flux] wavelength on Armstrong and flux normalized preferred. Then made another dataframe with the list of lines of Fe I.

The function closer lines select the closer minimums of the FeI lines and save the wave flux of that minimum point and the Fe I line associated. Don't be confused, this is not the observed wavelength. This point is a reference for selecting the bins around the Fe I line. The function discards distances over 0.001A.

Then the function local points select the bins of each corresponding to one index on the closer lines dataframe. Each bin of wavelength is for 0.1mA around the minimum point. 

The function Derivatives find the polynomial fourth order fit and calculate the minimum point with the fit, that is the observed wavelength. This returns a dataframe with the FeI line, flux fit and the minimum observed.

In parallel are calculated the bisectors of each line following the midpoint method, where equal points of flux are selected for comparison.

From this is the first visualizer that receives the local points, the values and the fit. This shows the line core and the fourth polynomial fit, in parallel is shown the bisector of each line in terms of velocity.

We recommend eliminating the lines that follow one of the conditions presented below:
1)The bisector doesn't show a C-curved bisector or it's too affected by the noise.
2)There is no curve or polynomial fit. This can be interpreted as the position on other points to the fit.
3) There's too much noise on the original line.

With this first filter the number of possible lines are reduced for calculating derivatives.

The second part of the code calculates the granulation pattern, core curvature and bisector slope. With these values the visualizer shows all the graphics including the line profile with the polynomial fit. In each graphic of derivatives the corresponding Fe I is resalted, this with the finally to select lines depending on his behavior.
 
This is a code test:
Run the file test, you can adapt this part on your necessaries. The idea is the Dataframe results have the columns Wave (cm), nFlux and Wave A (there's no need for the flux to be normalized, it cannot be, we test this on arturus). 

Then run the nave list, the present github has the table organized on an excel.

Run the cell of closer points and local points, the first needs to be returned a Dataframe and the other a list of dataframes.

For the first filter code you need to have a dataframe with the columns (), the list of local points and the closer lines associated with the Dataframe, the fit values and the covariance values.

Modified this line to call the first visualizer.

Then, this part helps to extract the Dataframe with the lines that don't behave like the condition parameters. This can also help to drop the unnecessary lines

For the second enter the file with the lines resulting in the first filter, and run again the code for closer lines and local points, and then the second big filter.

Modified this line to call the second visualizer.

The next line helps to extract the lines to drop and remove it for the Dataframe.

\section{Conditions justifications}
Two is for observations on polynomial fits that derive for mathematics properties. The third born on the many observations that i realize after the creation of visualizer, I see the parameter of 0.001 for near lines and I put the value

\end{appendices}

%------------------------------------------------------------------- BIBLIOGRAPHY AND CITATIONS
\renewcommand{\refname}{Bibliography}
\bibliographystyle{unsrt}
\bibliography{Bibliography.bib}


\nocite{Aponte,Gray_2010,Gray_Brown_2006,Gray_2005,Stief_2019,Ryden_2003,Cacciani_2006,Griem_2012,Owocki_2024}
\end{document}
