As previously mentioned, David Gray has significantly advanced the study of granulation patterns in the solar photosphere, with a particular focus on measuring relative velocities with high precision.
This chapter explores the three signatures of convective motion in the Sun photosphere, and how this reveals the hydrodynamics on this outermost layer. 

\section{The solar interior and the solar outer atmosphere}

The Sun is classified as a yellow dwarf star of spectral type G2V, a title achieved for a big amount of ionised hydrogen and a small percent of helium.
What makes this star really unique is its proximity to earth and the facility to study from the planet with precision.
In general, the Sun is divided in two fundamental parts: The solar interior and the solar outer atmosphere. 

\begin{figure}[H]
     \centering
     \begin{subfigure}{0.48\textwidth}
         \includegraphics[width=\textwidth]{Sun interior.jpg}
         \caption{The interior structure of the Sun. The convection zone is responsible for the general movement that characterizes the third signature.}

         \label{sun interior}
     \end{subfigure}
\hfill
     \begin{subfigure}{0.48\textwidth}
         \includegraphics[width=\textwidth]{Sun exterior.jpg}
         \caption{The outer structure of the Sun. The photosphere is the layer of the sun where the convection cells overshoot from the convection zone.}
         \label{sun outer atmosphere}
     \end{subfigure}

        \caption{The general structure of the Sun. Image modified from \cite{Priest_1982}.}
        \label{Sun structure}
\end{figure}

As shown in the figure \ref{sun interior} the overall structure of the solar interior is core, radiative and convective zone. 
Across then the density and temperature falls significantly, as the energy is slowly transferred outwards by radiative diffusion.
Some models of the interior structure give a core temperature of $1.6\times 10^7 K$ and density to $1.6\times 10^5 Kg/m^3$, high enough for thermonuclear reactions and remains the central material in plasma like a gigantic atomic reactor. 
This characteristic allows the collisions, absorptions and remissions of photons that make this zone opaque. 
In consequence, there exists an increase of the wavelength from high-energy gamma rays to visible light.

On the other hand, the figure \ref{sun outer atmosphere} shows the overall structure of the solar outer atmosphere which consists of the photosphere, chromosphere and corona. 
In this part the density decreases rather rapidly with height above the solar surface, and the temperature decreases to $4300K$ for then rises through the transition region.
Thereafter, the temperature falls slowly expanding outwards as the solar wind. 
The most relevant layer is the photosphere, a thin layer of plasma that emits most of the solar radiation and emits a continuous spectrum with superimposed dark absorption lines.
Most of these wavelengths are absorbed by the chromosphere, which is transparent\cite{Priest_1982}.
From this the photosphere emits a continuous spectrum with superimposed dark absorption lines where most of these wavelengths are absorbed by the chromosphere \cite{Priest_1982}.

The target layers of this study are the convection zone and the photosphere, which will focus in further sections.

\begin{figure}[H]
    \centering
    \includegraphics[width=0.55\linewidth]{First take photosphere.jpg}
    \caption{The first clear photograph of the photosphere where the granulation pattern is visible. Image taken from \cite{Malherbe_2022}}
    \label{Janssen photography}
\end{figure}; As we shown in their figure \ref{Janssen_1885}, in 1885 Janssen took the first clear photograph of the granules in the photosphere \cite{Malherbe_2022}.
This was the starting point for different studies across the pattern of granules.
In 1930 Unsold pointed out that the layers below the photosphere should be convective unstable \cite{Foukal_1990}. 
Later, Plasketts related these granules with Bernard's laboratory measurements of fluid convection.
This statement is based on fluids heated from below representing hot rising gas elements convecting heat to surface \cite{Plaskett_1936}.
These elements are known as granulation, and each individual region is referred to as a granule. They span approximately 700 km in size and last between five to ten minutes.
The understanding of photospheric granules as convective cells leads to the existence of a zone responsible for convective motion.

\section{The solar convection Zone}
Starting at the $0.86 R_{\astrosun}$ lies the zone where the dynamics processes took place, the great temperature gradient across the layer allows the process of convection \cite{Priest_1982}.

\subsection{The convection movement in the sun}
As we mentioned, the convective movements are based on fluids heated from below representing hot rising gas granules or \textit{convective cells} convecting heat to the photosphere \cite{Plaskett_1936}. 
In this case, solar convection occurs in a highly compressible, stratified gas which leads to determine the conditions under which we expect convection and the dynamics of the granules \cite{Foukal_1990}. 

\subsection{Dynamics of solar convection}
From the core, He nuclei is built from H nuclei in the proton-proton cycle as say equation \eqref{pp chain}.
\begin{equation}
    4^{1}H \rightarrow ^4He +2e^++2\nu +26.7MeV
    \label{pp chain}
\end{equation}
From the H nuclei is liberated a considerable amount of high frequency $\gamma-$rays ($26.7MeV$) and the energy of two neutrinos (0.5MeV).
However, the strong Coulomb repulsion between positively charged nuclei increases as the product of their nuclear charges, so only lightest elements will have appreciable reaction probabilities.
As the electrons recombine with other particles the photons can be absorbed more easily. 
With this, decrease the radiative conductivity and increase the temperature gradient with the opacity \cite{Foukal_1990}. 
When it reaches the low photosphere, some radiation escapes from the sun and the material returns to the convective stability \cite{Priest_1982}.

This inclines to establish an onset of convection: If $T$ is increased to a value $T'$, the granule will expand rapidly to achieve a new pressure equilibrium with its surroundings.
Where is suppose the granules as vertical stratified plasma in hydrostatic equilibrium with pressure ($P(r)$); density ($\rho(r)$); and temperature ($T(r)$), the same as its surroundings at radial distance $r$ from the center of the star.

The lower-density gas experiences a buoyancy force, which will cause it to rise.
The buoyancy force will disappear when the density has dropped to the same value as the new surroundings, after an element has traveled a distance $l$.
Let be $T_n'$ the new temperature and that of its surroundings $T_n$, the adiabatic and radiative gradient for the element follows the equation \eqref{adiabatic radiative gradients}.
\begin{equation}
    T_n = T + \parens{\frac{dT}{dr}}_R l \quad \big| \quad T_n' = T' + \parens{\frac{dT}{dr}}_{ad} l
    \label{adiabatic radiative gradients}
\end{equation}
Where $ad$ references for adiabatic gradient; $R$ is for radiative gradients; and taking $l$ small.
Since we have assumed radiative equilibrium, the R denotes the gradient present in the stellar atmosphere.
Two conditions can arise from here (see equation \eqref{radiative condition}): The radiative gradient is unstable, so convection pattern is established and vice versa.

\begin{equation}
    -\parens{\frac{dT}{dr}}_R > \parens{\frac{dT}{dr}}_{ad}
    \label{radiative condition}
\end{equation}

If the convection pattern is established, the element continues to expand further and rise.
Otherwise, the element begins to contract, becomes heavier, and begins to move down to its original position.

This onset of instability when the vertical temperature gradient is too large is explained by the Schwarzschild condition.

\subsection{The Schwarzschild condition}
Taking the element described before, now suppose an elementary parcel of material displaced so slowly that remains in horizontal pressure equilibrium (see figure \ref{parcel argument}). 

\begin{figure}[H]
    \centering
    \includegraphics[width=0.7\linewidth]{Parcel argument.jpg}
    \caption{Diagram for the parcel of material displaced so slowly that remains in horizontal pressure. Image taken from \cite{Priest_1982}}
    \label{parcel argument}
\end{figure}

If the motion is adiabatic there is no heat exchange with surroundings, the rate between the density and pressure is constant.
This generates a criterion for the presence of convection known as the Schwarzchild condition \eqref{Schwarzchild cond}.
\begin{equation} 
    -\frac{dT}{dr} > \frac{\gamma - 1}{\gamma} \parens{\frac{GM_{\astrosun}m}{r^2k_B}}
    \label{Schwarzchild cond}
\end{equation}
Where $k_B$ refers to Boltzmann constant; $G$ the gravitational constant; $M_{\astrosun}$ the solar mass; $m$ the mass of the granule; and $\gamma$ the degree of freedom of the fluid. 
As the presence of convection reduces the temperature gradient from the higher value it would have assumed under radiative transport alone to the essentially adiabatic value \cite{Foukal_1990}.

All of the material and energy generated by convection ended up to low photosphere, where the granules show different properties examined in after sections.

\section{The Solar Photosphere} 
Since 1874, when Langley gave a detailed description of granulation on the photosphere, astronomers have been studying different motions and reactions on the outermost solar layer \cite{Priest_1982}. 
The distinct pattern of granules with dynamic behavior becomes apparent, where individual areas continuously emerge and disappear (see figure \ref{photosphere visual}).

\begin{figure}[H]
    \centering
    \includegraphics[width=0.7\linewidth]{Granulation pattern.jpg}
    \caption{A view of granulation on the Sun's surface. The central regions exhibit blueshifts while the edges display redshifts. Image taken from \cite{Samir_pattern}}
    \label{photosphere visual}
\end{figure}

The bright areas of granules correspond to regions where hot gas rises through the solar atmosphere. 
As this gas releases energy in the form of photons at the photosphere, it cools and subsequently descends, creating the darker regions of granules \cite{Carroll_Ostlie_2006,Dravins_1981}.

\subsection{Static photosphere: Limb darkening phenomenon}

Because the temperature decreases from the higher layers, the photosphere intensity falls off towards the solar limb. 
Consequently, the disk intensity profile becomes more squared at increasing wavelength (see figure \ref{squared profile}).

\begin{figure}[H]
    \centering
    \includegraphics[width=0.6\linewidth]{Limb darkening.jpg}
    \caption{Squared profile for the disk intensity at increasing wavelengths. Image taken from \cite{Foukal_1990}}
    \label{squared profile}
\end{figure}

This effect is known as \textit{Limb darkening phenomenon}, was discovered in 1907 by Halm \cite{Dravins_1981}.  
The analysis of this effect provides a direct technique for determining the photosphere temperature structure along depth. 

Furthermore, these granules has shown to be in continual motion (see figure \ref{motion granules}).

\begin{figure}[H]
    \centering
    \includegraphics[width=0.7\linewidth]{Granules motion.jpg}
    \caption{A time sequence showing granule evolution where the time intervals are about a minute. Image taken from \cite{Foukal_1990}}
    \label{motion granules}
\end{figure}

This perpetual motion across the layer generates asymmetries on the absorption lines. 
For typical weaker lines the limb effect corresponds to $400m/s$, closer to the limb the gravitational redshift is observed.
As explain Ellwarth et al. observations closer to the limb pass through the atmosphere a shallower angle, resulting in longer optical paths through high atmospheric layers \cite{Ellwarth_2023}.

\subsection{Dynamic photosphere: The C-curved profile bisector.}
Observing the dynamic of the granules, there appears a height dependence of the granular velocities: The velocity of an upward moving granule decays much less rapidly than its excess brightness. 
Changes in the granulation structure, contrast and velocity field around the spots and network have been inferred indirectly from observations of Fraunhofer line profile shapes \cite{Foukal_1990}. 
The result is characteristically C-Curved profile bisector (see figure \ref{c curved profile}).

\begin{figure}[H]
    \centering
    \includegraphics[width=0.75\linewidth]{C curved profile bisector.jpg}
    \caption{The C-curved profile bisector. In the infrared shown to be less pronounced than violet range.}
    \label{c curved profile}
\end{figure}

The process of the creation in this c-curved profile bisector is divided in three stages.
First, the line profile near its mid-depth portion is formed in the most rapidly upflowing bright material (Blueshifted).
Then, the deepest portion of the line core is formed higher in the decelerated upflow (Less Blueshifted).
Finally, the line wings where the opacity is least, tend to be formed deepest in the cool (Redshifted).

This dynamic process induces perturbations in spectral absorption lines, line profile asymmetries, and line depth-dependent wavelength shifts \cite{Gray_Pugh_2012} known as the three signatures of convection.

\section{The three signature of convection}
The signatures of convection in stars are described by Gray in his research \cite{Gray_2009,Gray_Oostra_2018,Gray_Pugh_2012} about the principal characteristics in the spectra to identify and treat the asymmetries due to convective motion.

\subsection{Line broadening}
Due to the limb darkening phenomenon and atmospheric absorption, there exists two types of line cores: Strongest and weaker lines \cite{Gray_Pugh_2012}.
Stronger lines refers to more curved line cores than weaker lines.
This effect is measurable with the second derivative in the observed wavelength, that represents the core curvature (see equation \eqref{core curvature}).

\begin{equation}
    C_c =  \lambda_{obs}^2 \parens{\frac{d^2 p(\lambda_{obs})}{d \lambda_{obs}^2}}
    \label{core curvature}
\end{equation}

Where $p(\lambda)$ is the polynomial fit for the core evaluated on the observed wavelength; and $\lambda_{obs}$ is the minimum of the polynomial fit.
In the near infrared range we can denote natural bands or groups of lines  which are separated for the telluric elements. 
In other words, elements like $O_2$ and $CO_2$ that are absorbed by the atmosphere.
This wavelength range is invisible for the spectra, and it helps us to detect the natural bands in the near infrared curvature profile.

\subsection{Line profile bisector asymmetry}
As demonstrated by Nieminen \cite{Nieminen_2017}, the asymmetry occurs because the c-curved profile bisector reflects velocity variations and a bisector slope (see figure \ref{asymmetries on lines}).

\begin{figure}
    \centering
    \includegraphics[width=0.8\linewidth]{Asymmetries on typical lines.jpg}
    \caption{Asymmetries on an average absorption line. Can be observed the differences in intensity redshift profile. Image taken from \cite{Nieminen_2017}}
    \label{asymmetries on lines}
\end{figure}

According to Kirchhoff's laws, absorption line formation requires lower temperature conditions, which are found precisely in the Sun's outermost atmospheric layers \cite{Carroll_Ostlie_2006}. 
These regions not only provide the appropriate temperatures for absorption line formation but also exhibit comparatively higher opacity. 
Among all available spectral lines, those from neutral iron (Fe I) are particularly valuable for solar granulation studies due to two key characteristics: They display significant opacity and relatively low thermal broadening. 
These properties make Fe I lines excellent tracers of granulation patterns in the solar photosphere \cite{Nieminen_2017}.

These asymmetries can be measured using the third derivative relation for the line bisector (see equation \eqref{third derivative relation}).

\begin{equation}
    \frac{c}{\lambda_{obs}}\parens{\frac{1}{3 C_c^2}}\parens{\frac{d^3 p(\lambda_{obs})}{d \lambda_{obs}^3}}
    \label{third derivate relation}
\end{equation}

Where $C_c$ refers to the core curvature in the equation \eqref{core curvature}.

As the majority of stellar observations are made with lower-resolution spectrographs and often lower signal-to-noise ratios, there exists a necessity of study the granulation pattern.
However, Gray and Oostra shown that exists another form of line profile bisector asymmetry called \textit{flux deficit}. 
This phenomenon can be interpreted as flux imbalances on the redshifted part of the line due to loose radiative energy. 
In consequence the bisector is inclined compared to the original.
To treat with this asymmetry and calculate the flux deficit was suggested by the authors performs a difference beetwwen the altered profile and the original in order to find the flux distribution.
The deficit position implieas an average velocity difference between granules and lanes of $\approx 4 km/s$.
When the altered profile can be seen as the aproximate representation of the velocity distribution, to map the observed line bisector onto the third signature plot \cite{Gray_Oostra_2018}.

\subsection{Line depth-dependent wavelength shifts}
Many studies across the years have detected and observed the phenomena of wavelength shifts against the line depth, or as it is called, chromodependence in the granulation pattern.
This behavior was shown to be more present in the weaker lines which are related to infrared and violet range.

\subsection{Solar granulation pattern}
The solar granulation pattern is a plot of relative velocity against line depth, as shown in the figure \ref{granulation pattern}.

\begin{figure}
    \centering
    \includegraphics[width=0.75\linewidth]{Granulation pattern Ellwarth.jpg}
    \caption{Granulation pattern for the IAG spatially resolved quiet sun atlas, is shown to have a strong chromodependence in the weaker lines. Image taken from \cite{Ellwarth_2023}.}
    \label{granulation pattern}
\end{figure}

The importance of this pattern lies in stars resembling solar patterns, differing only by a scaling factor (see \cite{Gray_Pugh_2012}). 
Furthermore, this analysis contributes to the understanding and radiation of photospheric hydrodynamic models \cite{Dravins_1981,Gray_2009}.
However, for the measurement is necessary the understanding of the convective blueshift.

\subsubsection{Convective Blueshift}
When the Sun pushes material up through its outer layer, the spectrum exhibits a blueshift. 
As this material subsequently cools and falls back through the atmosphere, it produces a redshift, but emits less light, making the blueshift dominant.
That can be measure by the Doppler effect but as the velocities can be significant, the relativistic formulation of this effect must be applied.

The relativistic Doppler effect accounts for length contraction, as predicted by Einstein’s theory of relativity. 
This introduces an additional correction term to the classical shift, which becomes particularly relevant in high-velocity scenarios or strong gravitational fields.
However, the measured negative redshift resulting from convective motions is known as \textit{convective blueshifts} which are measured by the equation \eqref{relative velocity}. 
\begin{equation}
    v_r \approx c \parens{\frac{ \lambda_{obs}- \lambda_{rep}}{\lambda_{rep}}}
    \label{relativevelocity}
\end{equation}

Since the strength of the convective distortions and shifts as spectral lines vary across the H-R diagram, we expect systematic errors in radial velocities \cite{Gray_2009}.

\subsection{Chromodependence characterization}
Recently, the emphasis on the measurement of granulation pattern has opened new interpretations to the line depth-dependent wavelength shifts.
However, there is still no characterization for this phenomenon.

In 2018 Gray and Oostra tried to establish a standard curve determined by the third grade polynomial fit to the solar granulation pattern.
Although, the authors just take into account the range of $4950\mathring{A}$ to $5700 \mathring{A}$ of the spectrum, avoiding the chromodependence in the weaker lines.
This motivates us to search for a strong characterization and give the recipe for dealing with this phenomenon.

Hamilton and Lester, give a little theory that this phenomenon is attributed to the rotation in the photosphere.
The pronounced differential rotation with latitude observed at the photosphere seems to be the result of convective flows driven radially by the buoyancy force and deflected horizontally by the coriolis force due to solar rotation \cite{Foukal_1990}.
This rotation made contributions on angular momentum in two different forms: Meridian circulation and Reynolds stresses.

\begin{figure}
    \centering
    \includegraphics[width=0.7\linewidth]{Angular momentum.jpg}
    \caption{Contributions on angular momentum. Image taken from \cite{Foukal_1990}.}
    \label{angular momentum}
\end{figure}

The first contribution occurs if axisymmetric meridional circulation is present (see figure \ref{angular momentum}). 
In the absence of any other angular momentum transport, a circulation in either sense will tend to spin up the poles and the interior because the fluid carries angular momentum. 

The second contribution tends to enforce solid body rotation, then the meridional circulation drives an equatorial acceleration.
The reason is for equal velocity in the meridional plane, the flux of angular momentum per unit mass across the dashed line will be larger toward the equator than away from it.
This mechanism depends on the existence of nonaxisymmetric convective motion because net fluxes of angular momentum in latitudinal or radial directions are produced without requiring a net mass flux. 
Neither buoyancy forces, which are strictly radial, nor pressure gradients, which must average to zero around the solar circumference, can themselves influence the sun's axisymmetric rotation profile \cite{Foukal_1990}.


