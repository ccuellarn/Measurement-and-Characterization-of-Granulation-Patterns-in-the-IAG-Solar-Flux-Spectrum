As previously mentioned, David Gray\cite{Gray_2009,Gray_Pugh_2012} has significantly advanced the study of granulation patterns in the solar photosphere, with a particular focus on measuring their associated relative velocities with high precision.
These developments have enabled more accurate characterizations of other stars by extrapolating the physical principles observed in the Sun.
This chapter explores the three signatures of convective motion in the Sun photosphere, and how this reveal the hydrodynamic on this outermost layer. 
Furthermore, is given the point of view of different authors respect the reasons behind the chromodependece on the granulation patterns.

\section{The solar interior and the solar outer atmosphere}

The Sun is classified as a yellow dwarf star of spectral type G2V, title achieved for a big amount of hot hydrogen (ionised H in 90 percent) and helium (in 10 percent).
What makes this star really unique is his proximity to earth and the facility to study from the planet with precision.
In general, the Sun is divided in two fundamental parts: The solar interior and the solar outer atmosphere. 

\begin{figure}[H]
    \centering
    \includegraphics[width=0.6\linewidth]{Granulation pattern.jpg}
    \caption{The interior structure of the Sun. The convection zone is the responsible by the general movement that characterize the third signature.}
    \label{Interior structure}
\end{figure}

As shown in the figure \ref{Interior structure} the overall structure of the solar interior is core, radiative and convective zone. 
Across then the density and temperature falls significantly, as the energy is slowly transferred outwards by radiative diffusion.
Some models of the interior structure give a core temperature of $1.6\times 10^7 K$ and density to $1.6\times 10^5 Kg/m^3$, high enough for thermonuclear reactions and remains the central material in plasma like a gigantic atomic reactor. 
This characteristic allow the collisions, absorptions and reemisions of photons that made this zone opaque. 
In consequence, there exists an increase of the wavelength from high-energy gamma rays to visible ligth.

\begin{figure}[H]
    \centering
    \includegraphics[width=0.6\linewidth]{Granulation pattern.jpg}
    \caption{The outer atmosphere of the zone. The photosphere is the layer of the sun where is visible the convection motion.}
    \label{Outer structure}
\end{figure}
 
On the other hand, the figure \ref{Outer structure} shown the overall structure of the solar outer atmosphere which consist in photosphere, chromosphere and corona. 
In this part the density decreases rather rapidly with height above solar surface, and the temperature decrease to $4300$ $K$ for then rises through the transition region.
Thereafter, the temperautre falls slowly expanding outwards as the solar wind. 
From this the photosphere emits a continuous spectrum with superimposed dark absorption lines where most of this wavelengths are absorbed by the chromosphere \cite{Priest_1982}.

The target layers of this study are the conevction zone and the photosphere, which will focus in further sections.

\section{The Solar Convection Zone}
This is a zone ubicated at one solar radius. In this zone the temperature gradient is great for the material to stay in static equilibrium, and is where the magnetic field is generated\cite{Priest_1982}.
Until 1885, when Janssen took a photography of the granules in the photospere and Plasketts relationed these granules with the Bernard laboratory measurements of fluids convection, they took into account the dynamics of the layers in the Sun \cite{Foukal_1990}.

\subsection{The convection movement}
In 1930 Unsold pointed out that the layers below the photosphere should be convective unstable. 
In 1936 Plasketts sugested that granules looked similar to the pattern of convective cells found in Bernards experiments.
This statement are based on fluids heated from below representing hot rising gas elements convecting heat to surface. 
This last is caused by the Janssen observations in 1885, when take a photography of photospere showing pores and granulation \cite{Foukal_1990}.
%Insert a graphic of the convection motion of the fluids 
So, we see the photospheric granules as convective cells. 
Have been shown to be influenced more by surface tension in the shallow near layers than by the bouyancy forces that drive free convection. 
Solar convection occurs in a highly compresible, stratified gas\cite{Foukal_1990}. 
When a liquid layer is heated from below, convection iniciatilly sets in a 2d, horizontal, parallel rolls. 
The existence of a 3d convection pattern of convection cells that is static is limited to fluids whose viscosity $\nu$ is high compared to their thermal conductivity $\kappa$ (the ratio of this two is the Prandtl number P). 
Fluids with $P\approx1$ in Laboratory show little evidence of steady 3D convection, an increase of the gradient temperature leads to a turbulent flow pattern. 
But at te photosphere with $P\approx 10^{-9}$ we expect to find turbulent behavior in the granulation if the temperture gradient is sufficient to drive free convection at the observed velocities and horizontal temperature differences\cite{Foukal_1990}.

\subsection{Dynamics of solar convection}
To determine the conditions under which we expect convection we suppose that a small elements of mass $m$ stay in radiative equilibrium has $P,\rho,T$, the same as its surroundings at radial distance r from the center of the star.
Onset of convection: If T is increased to a value T', the gas in this element will expand rapidly to achieve a new pressure equilibrium with its surroundings.
The lower-density gas will experience a bouyancy force, which will cause it to rise.
The bouyancy force will disappear when the density has dropped to the same value as that of its new surroundings after an element has traveled a distance l.
Let be $T_1'$ the new temperature and that of its surroundings $T_1$ (l is small) So
\begin{equation}
    T_1 = T + \parens{\frac{dT}{dr}}_R l \space T_1' = T' + \parens{\frac{dT}{dr}}_{ad} l
    \label{adiabatic radiative gradients}
\end{equation}
Where the ad is for adiabatic gradient and R is for radiative gradients.
Since we have assumed radiative equilibrium, the R denotes the gradient present in the stellar atmosphere.
Two conditions can arise from here:

First, $T_1'>T_1$ what is like $\Delta T>0$ so equation () follows.
\begin{equation}
    -\parens{\frac{dT}{dr}}_R > \parens{\frac{dT}{dr}}_{ad}
    \label{1 condition}
\end{equation}
This is the element continues to expand further and rise.
So, convection pattern is established and the radiative gradient is unstable.

The other condition, as it be, the element begins to contract, becomes heavier, and begins to move down to its original position.
The radiative gradient is stable, and a displacement element will undergo oscillation.

\subsection{The Schwarzchild Condition}
Convection will tend to take over as the energy transport mechanism wherever the radiative gradient temperature becomes large due to the large local opacity or strong local energy generation.
So, convection can set when opacity rises very rapidly due to increasing population of n=3 level of hydrogen, and when $\gamma$ is lowered by ionization.

The equation \eqref{Schwarzchild cond} stablish the criterion for presence of convection.
As the presence of convection reduces the temperature gradient from the higher value it would have assumed under radiative transport alone to the essentially adiabatic value \cite{Foukal_1990}.

\subsection{The parcel argument}
From the core, He nuclei is built from H nuclei in the proton-proton cycle as say equation\eqref{pp chain}.
\begin{equation}
    4^{1}H \rightarrow ^4He +2e^++2\nu +26.7MeV
    \label{pp chain}
\end{equation}
Where from the H nuclei is liberated an considerable amount of high frecuency $\gamma-$rays ($26.7MeV$) and the energy of 2 neutrinos (0.5MeV).
The strong Coulumb repulsion between positevily charged nuclei increases as the product of their nuclear charges, so only lightest elements will have appreciable reaction probabilities \cite{Foukal_1990}.
As the electrons recombine with others particles the photons can be absorbed more easily. 
With this, decrease the radiative conducity and increases the temperature gradient with the opacity. 
In the boundary of radiative zone the temperautre is drecreasing rapidly, so it begins a convective instability and beyond a convective trubulence.
When reaches the low photosphere, some radiation scape from the sun and the material returns to the convective stability \cite{Priest_1982}.

The onset of instability when the vertical temperautre gradient is too large is explained by the parcel argument and the Schwarzchild condition.
%Insert graphic from priest
Consider a vertical stratified plasma in hydrostatic equilibrium with pressure ($P(r)$); density ($\rho(r)$); and temperature ($T(r)$). 
Now suposse an elementary parcel of the material is displaced so slowly that remains in horizontal pressure equilibrium. 
Then the parcel will feel a bouyancy force and continue raise  if $\delta \rho_i < \delta \rho$. 
By the ideal gas law, the density changes need to follow the equation\eqref{parcel density}.
\begin{equation} 
    \frac{\delta P_i}{P} = \frac{\delta \rho_i}{\rho} + \frac{\delta T_i}{T} 
    \label{parcel density}
\end{equation}
Remains the same for the final state. But for static equilibrium, the conditions \eqref{parcel unstable cond} are gonna make the parcel unstable and the fluid will continue raising.
\begin{equation} 
    -\frac{dT_i}{dr} > \frac{dT}{dr}  \text{and} \delta T_i <-\delta T 
    \label{parcel unstable cond}
\end{equation}
If the motion is adiabatic there is no heat exchange with surroundings. So the rate between the density and pressure is constante, what generates a limit in the adiabatic temperature gradient (see equation \eqref{adiabatic T gradient}).
\begin{equation} 
    \frac{dT_i}{dr} = \frac{\gamma - 1}{\gamma} \parens{\frac{GM_{\bigodot}m}{r^2k_B}}
    \label{adiabatic T gradient}
\end{equation}
Where $k_B$ refers to Boltzman constant, and $\gamma$ the degree freedom of the fluid. With this limit the criterion for conevctive inestability becomes the equation \eqref{Schwarzchild cond}.
\begin{equation} 
    -\frac{dT}{dr} > \frac{\gamma - 1}{\gamma} \parens{\frac{GM_{\bigodot}m}{r^2k_B}}
    \label{Schwarzchild cond}
\end{equation}
When H and He nuclei become ionised, they can absorb energy by more degrees of freedom; this decreases the value of $\gamma$ and the adiabatic temperature gradient, which makes convection easier.
It has been sugested that granulation are driven by the ionization of H and He, and have scales comparable with the depths at wihich processes take place\cite{Priest_1982}.

From surface observations the convection appears to be domination by cells: Granules, mesogranules, supergranules and giant cells.
These granules can perturbe the angular momentum of the sun and the flow patterns in this zone.

\subsection{Contributions on angular momentum}
The pronounced differential rotation with latitude observed at he photosphere seems to be the result of convective flows driven radially by the bouyancy force and deflected horizontally by the coriolis force due to solar rotation (term of $2\rho \vec{\omega }\times \vec{v}$).
Is dificult to datermine wheter the coriollis effects acts on slow global scale axysimmetrical circulations in the meridional plane, on intermediate scale eddies, or on a hierarchy of small eddies \cite{Foukal_1990}.
There are two types of contribution on angular momentum.
%Maybe insert the diagram from foukal where its explained the two contributions
The first contribution is the meridian circulation. 
Occurs if axysimmetric meridional circulation are present. 
Movement of fluids in a vertical plane.
In the absence of any other angular momentum transport, a circulation in either sense will tend to spin up the poles and the interior because the fluid carries angular momentum. 
These regions will be spin up (activating) until the flux of $\omega r^2$ is equal both radius and latitude \cite{Foukal_1990}.

The second contribution is the reynolds stresses. 
These process tends to enforce solid body rotation, then the meridional circulation drive an equatorial aceleration.
The reason is that for equal velocity in the meridional plane, the flux of angular momentum per unit mass across the dashed line will be larger toward the equator than away from it.
This mechanism depends on the existence of nonaxisymetric convective motions. 
Because net fluxes of angular momentum in latitudinal or radials directions are produced without requering a net mass flux. 
They play an important role in the dynamics of turbulent fluids \cite{Foukal_1990}.

Neither buoyancy forces, which are strictly radial, nor presure gradients, which must average to zero around the solar circumference, can themselves influences the suns axysimetric rotation profile\cite{Foukal_1990}.


\section{The Solar Photosphere}
Since 1874, when Langley gives a detailed description of granulation on the photosphere, the astronomers have been studied different motions and reactions across the layers \cite{Priest_1982}. 

Is an extremely thin visible surface layer, is only 100km thick and centered in the region where $T=5000K$. 
In high resolution spectrograph shows a granular struture and a filme shows these brigth granules to be in continual motion. 
These granules represents the top of convective cells that are overshooting the upper convection zone. 
There are composed by hot, rising and horizontally outflowing plasma rather than cooling material. 
This is the region where magnetic flux is concentrated \cite{Priest_1982}.

The solar atmosphere is highly inhomogeneous and turbulent but we can take a start point. 
This is named Harvard-smithsonian reference atmosphere. It cero level is taken as the point where optical depth at a $\lambda=5000\mathring{A}$ is equal to one.
%Insert a graphic for this model and explain why we take it

For the deeper analysis we carried out is better to see separated the static and dynamic photosphere.

\subsection{Static photosphere: Limb darkening phenomenon}
The photosphere intensity falls off towards the limb, cause the temperature decrease of the higher layers as we look nearer the limb. 
The analysis of this effect provides a direct technique for determining the photosphere temperature structure with depth. 
This effect drecreases with increasing wavelength, the disk intensity profile becomes more squared at infrared. 
Since the limb darkening is caused by the temperature gradient we might to expect it to disappear in the infrared or ultraviolet where we observe layers around the $T_{\min}$ \cite{Foukal_1990}.

\subsection{Dynamic photosphere. The C-curved profile bisector.}
There exists a heigth dependence of the granular velocities. 
The velocity of an upward moving granule decays much less rapidly than its excess brigthness. 
The granular material is dark when observed in Fraunhofer lines formed high in the photosphere. 
So the material is cooler caused by its rapid expansion. 
Changes in the granulation structure, contrast and velocity field around the spots and network have been infered indirectly from observations of Fraunhofer line profile shapes. 
The result is characteristically C-Curved profile bisector \cite{Foukal_1990}.
The process of the creation in this c-curved profile bisector is listed before:
\begin{itemize}
    \item The line profile near its mid-depth portion is formed in the most rapidly upflowing bright material. Blueshifted.
    \item The deepest portion of the line core is formed higher in the decelerated upflow. Less Blueshifted.
    \item The line wings, where the opacity is least, tend to be formed deepest in the cool. Redshifted.
\end{itemize}
The bisectors tend to be less c-curved near solar activity maximum that at minimum. 
Near activity maximum the higher packing density of magnetic flux tubes will tend to disrupt granular convection\cite{Foukal_1990}.


\section{Solar granulation pattern}
When observing images of the solar photosphere, a distinct pattern of bright and dark regions with dynamic behavior becomes apparent, where individual areas continuously emerge and disappear (see figure \ref{granulation pattern}).
This phenomenon is known as granulation, and each individual region—spanning approximately 700 km in size and lasting between five to ten minutes—is referred to as a granule.
\begin{figure}[H]
    \centering
    \includegraphics[width=0.6\linewidth]{Images/Granulation pattern.jpg}
    \caption{A view of the granulation pattern on the Sun's surface. The central regions exhibit blueshifts while the edges display redshifts. Image taken from \cite{Samir_pattern}}
    \label{granulation pattern}
\end{figure}

Solar granulation arises due to the "eruption" of the convective zone at the base of the photosphere \cite{Gray_2009}.
Spectroscopic observations of the Sun reveal asymmetries in absorption lines caused by the motion of the solar atmosphere within granules. 
These asymmetries occur because some parts of these convective regions are blueshifted, while others are redshifted. 
The bright areas of granules correspond to regions where hot gas rises through the solar atmosphere, producing blueshifts in absorption lines. 
As this gas releases energy in the form of photons at the photosphere, it cools and subsequently descends, creating the darker regions of granules, which exhibit redshifts in absorption lines \cite{Carroll_Ostlie_2006}.

As we mention before, this granulation pattern arises from convective motions within the solar photosphere. 
These motions consist of an upward flow of matter from hotter inner layers to cooler outer regions—and vice versa. 
This dynamic process induces perturbations in spectral absorption lines, profile asymmetries, and depth-dependent wavelength shifts \cite{Gray_Pugh_2012} known as the three signatures of convection (name given by \cite{Gray_2009}).

Is important because granultion patterns in stars resemble solar patterns, differing only by a scaling factor (see \cite{Gray_Pugh_2012}). 
Furthermore, these analysis contributes to the understanding and radiation of photospheric hydrodynamic models \cite{Dravins_1981,Gray_2009}.

\subsection{Convective Blueshift}
When the Sun pushes material up through its outer layer, the spectrum exhibits a blueshift. 
As this material subsequently cools and falls back through the atmosphere, it produces a redshift, but emits less light, making the blueshift dominant.

\subsubsection{Relativistic Doppler Effect}
The Doppler effect is a wave phenomenon caused by the relative motion between a source and an observer, resulting in a measurable shift in the electromagnetic spectrum compared to laboratory or catalog reference values. 
However, in astronomical contexts—where velocities can be significant—the relativistic formulation of this effect must be applied.

The relativistic Doppler effect accounts for length contraction, as predicted by Einstein’s theory of relativity. 
This introduces an additional correction term to the classical shift, which becomes particularly relevant in high-velocity scenarios or strong gravitational fields.
\begin{equation}
    f_{obs}= f_{rep} \frac{\sqrt{1-v_r/c}}{\sqrt{1+v_r/c}}
    \label{f relativistic}
\end{equation}
As indicated in equation \eqref{f relativistic}, the observed frequency $f_{obs}$ corresponds to the light detected by an observer, while $f_{rep}$ is the frequency emitted by the source, and $v_r$ denotes its radial velocity \cite{Ryden_2003}. 
\begin{equation}
    z = \frac{ \lambda_{obs}- \lambda_{rep}}{\lambda_{rep}} =\sqrt{\frac{1-v_r/c}{1+v_r/c}} -1
    \label{redshift}
\end{equation}
Since cosmic expansion dominates the universe’s large-scale dynamics, the parameter $z$, known as redshift, was introduced. 
This term quantifies the relative recession of distant objects and can be expressed through the relativistic Doppler effect, as shown in equation \eqref{redshift}.

\subsection{Convective Blueshift and Relative Velocity}
However, in our study, we will define this blueshift resulting from specific motions as \textit{convective blueshifts}. 
This phenomenon occurs when spectral lines in stellar light, including the Sun's, appear shifted toward shorter wavelengths due to convective motions in the star's atmosphere \cite{Gray_2009}. 
Taking into account that most astronomical phenomena move with velocities much smaller than light, the equation \eqref{redshift} can be approximated to first order.
\begin{equation}
    \sqrt{1\pm v_r/c} \approx 1 \pm \frac{v_r}{2c} \quad \rightarrow \quad z \approx \frac{v_r}{c}
    \label{first order z}
\end{equation}

From the expression \eqref{first order z}, is advantageous to reform for the equation \eqref{velocity} with the approximation of relative velocity.
\begin{equation}
    v_r \approx c \frac{ \lambda_{obs}- \lambda_{rep}}{\lambda_{rep}}
    \label{velocity}
\end{equation}
With this equation the calculus of relative velocity is straightforward; therefore, the perturbations in spectral absorption lines.

This phenomena is a impediment to determing true radial velocities of stars to accuracies better than a few hundred m/s. 
Since the strength of the convective distortions and shifts os spectral lines varies across the H-R diagram, we expect the systematic errors in radial velocities \cite{Gray_2009}.

In resume is a graphic of doppler shift against line depth, it shows that weaker lines are more blueshifted. 

\section{The three signature of convective motion}
With all the mentioned about convection, there are thre signatures for identified convection in the outer atmosphere for a start. In our case we are just take the sun.

\subsection{Line broadening}
Due to the limb darkening phenomen the photospheric intensity falls off toward the limb, cause the temperature decrease of the higher layers seen as we look nearer the limb \cite{Foukal_1990}.
This effect decreases with increasing wavelength, the disk intensity profile becomes more squared  at infrared. Or as we call weaker lines.

This is measurable with the core curvature or the second derivate in the observed wavelength.

\subsection{Line profile bisector asymmetry}
As mention \cite{Hamilton_1999,Dravins_1981} we treat with a C-curved profile bisector assymetry due to convective blueshift. 
So we expect a difference in the bisector slope (thir derivate relation proved in the apendix).

Absorption spectra from the solar photosphere exhibit asymmetries that may be measured using the bisector method. 
This technique involves tracing a line connecting the midpoints between the spectral profile's wings at different intensity levels. 
The phenomenon is particularly prominent in spectral lines of pure elements; for this study, we specifically focus on neutral iron lines (Fe I) due to their low sensitivity to thermal motion, and other virtues, as mention below.

The solar spectrum reveals that our star emits radiation in the visible and infrared ranges following approximately a blackbody curve. 
This behavior indicates the presence of a continuous opacity source across the observed electromagnetic spectrum. 
This continuum opacity is primarily due to $H^{+}$ ions in the photosphere, which are responsible for hydrogen's contributions to the spectral continuum. 
According to Kirchhoff's laws, absorption line formation requires lower temperature conditions, which are found precisely in the Sun's outermost atmospheric layers \cite{Carroll_Ostlie_2006}. 
These regions not only provide the appropriate temperatures for absorption line formation but also exhibit comparatively higher opacity. 
Among all available spectral lines, those from neutral iron are particularly valuable for solar granulation studies due to two key characteristics: they display significant opacity and relatively low thermal broadening. 
These properties make Fe I lines excellent tracers of granulation patterns in the solar photosphere \cite{Nieminen_2017}.
\begin{figure}
    \centering
    \includegraphics[width=0.7\linewidth]{Images/asymmetries.png}
    \caption{Asymmetries on an average absorption line. Can be observed the differences in intensity redshift profile. Image taken from \cite{Nieminen_2017}}
    \label{asymmetries on lines}
\end{figure}
These asymmetries are physically significant because absorption line profiles should theoretically be symmetric under ideal conditions. 
Their detection directly reveals the influence of photospheric granulation patterns. 
As demonstrated by Nieminen \cite{Nieminen_2017}, the asymmetry occurs because each individual profile component contributes a distinct convective blueshift, reflecting velocity variations between atmospheric layers where convection dominates and the more stable interior regions (see figure \ref{asymmetries on lines}).

Defining the base of what we recognize as the photosphere is not a trivial task, primarily due to the dependence of optical depth on wavelength. 
This means that the observed radiation must traverse varying densities within the solar medium, where the temperature gradient across this atmospheric layer significantly influences its propagation.

\subsection{Line-depth dependent shifts}
Many studies across the yers have detected and observed the phenomena of wavelength shifts againts the line depth, or as it is called, chromodependence in the granulation pattern.
This behavior was showed to be more present in the weaker lines (less depth) which are related to infrered and violet range.
However, the many studies didn´t present a general characterization of this phenomena, just describe the problem.

The article which try to stablish a "standard" curve is \cite{Gray_Oostra_2018} where is presented a second grade polynomial fot to the solar granulation pattern.
But, the authors just take the green range of the spectrum, avoiding the chromodependece in the weaker lines.
This motivate us to search a strong characterization and give the recipe for dealing with this phenomenon.

\section{Teluric Absorption Lines}
In the near infrared range we can denote a natural bands or groups of lines  which are separated for the teluric elements, on others words, elements like oxygen and CO2 that are absorbed by the atmosphere.
Tis range are invisible for the spectra, and it helps us to detected the natural bands in the near infrered curvature profile.