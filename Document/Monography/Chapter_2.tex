As previously mentioned, David Gray has significantly advanced the study of granulation patterns in the solar photosphere, with a particular focus on measuring relative velocities with high precision.
This chapter explores the three signatures of convective motion in the Sun photosphere, and how this reveals the hydrodynamics on this outermost layer. 

\section{The solar interior and the solar outer atmosphere}

The Sun is classified as a yellow dwarf star of spectral type G2V.
Its chemical structure is primarily composed of a large fraction of ionized hydrogen and a smaller proportion of helium.
What makes the Sun unique in astronomical studies is its proximity to Earth,which allows for detailed observation unmatched for any other star.
Structurally, the Sun is divided into two main regions: The solar interior and the solar outer atmosphere. 

\begin{figure}[H]
     \centering
     \begin{subfigure}{0.48\textwidth}
         \includegraphics[width=\textwidth]{Sun interior.jpg}
         \caption{The interior structure of the Sun. The convection zone is responsible for the general movement that characterizes the third signature.}\label{fig:Sun interior}%
     \end{subfigure}
\hfill
     \begin{subfigure}{0.48\textwidth}
         \includegraphics[width=\textwidth]{Sun exterior.jpg}
         \caption{The outer structure of the Sun. The photosphere is the layer of the sun where the convection cells overshoot from the convection zone.}\label{fig:Sun outer atmosphere}%
     \end{subfigure}

        \caption{The general structure of the Sun. Images modified from~\cite{Priest_1982}.}\label{fig:Sun structure}%
\end{figure}

As illustrated in Figure~\ref{fig:Sun interior} the overall structure of the solar interior is core, radiative and convective zone. 
Moving outward through these layers, both the density and temperature decrease significantly, as the energy is slowly transferred outwards by radiative diffusion.
In the core, where energy is generated by thermonuclear fusion, standard models estimate a temperature of $1.6\times 10^7$  K and density to $1.6\times 10^5$  kg/$\text{m}^3$.

The extreme conditions in this region maintain the central material in a plasma state, functioning like a massive nuclear reactor.

This characteristic high density allows the absorptions and remissions of photons that make this zone highly opaque. 
This process, known as radiative diffusion, slowly transfers energy outward and progressively shifts the wavelength of the radiation from high-energy $\gamma-$rays to the visible light that eventually escapes.


On the other hand, Figure~\ref{fig:Sun outer atmosphere} illustrates the overall structure of the solar outer atmosphere which consists of the photosphere, chromosphere and corona. 
In these layers, the density decreases rapidly with height above the solar surface. 
The temperature decreases to a minimum of approximately $4300$  K in the upper photosphere before rising through the chromosphere and transition region to millions of degrees in the corona.
Thereafter, the temperature falls slowly expanding outwards as the solar wind. 

The most relevant layer for this project is the photosphere.
This thin layer of plasma that emits most of the solar radiation and a continuous spectrum.
When the continuous light passes through the overlying chromosphere, specific wavelengths are absorbed, resulting in the characteristic Fraunhofer lines superimposed on the continuum~\cite{Priest_1982}.
This characteristic allows observing the convection in the spectrum.

The target layers of this study are the convection zone and the photosphere, which will focus in further sections.

\begin{figure}[H]
    \centering
    \includegraphics[width=0.45\linewidth]{First take photosphere.jpg}
    \caption{The first clear photograph of the photosphere where the granulation pattern is visible. Image taken from~\cite{Malherbe_2022}.}\label{fig:Janssen photography}%
\end{figure}

In 1885 Janssen obtained the first clear photograph of photospheric granules~\cite{Malherbe_2022} providing the initial evidence for this granularar pattern (see Figure~\ref{fig:Janssen photography})
This was the starting point for different studies across the pattern of these granules.
In 1930, Unsöld theorized that the layers beneath the photosphere should be convective unstable~\cite{Foukal_1990}. 
This hypothesis was later supported when Plaskett related the observed granules to the convective cells studied in Bénard's laboratory experiments~\cite{Plaskett_1936}.
In this analogy, a fluid heated from below develops rising elements of hot gas that transport heat to the surface.

These convective elements are known as granulation, with each individual region referred to as a granule. 
Typical granules span approximately 700 km and have short lifetimes, lasting between five to ten minutes.
The understanding of granules as convective cells provided direct evidence for the existence of a zone responsible for convective motion beneath the photosphere.

\section{The solar convection Zone}
The convective zone, where dynamic plasma motions occur, begins at approximately $0.86 R_{\astrosun}$. 
The great temperature gradient across this layer allows the convective process~\cite{Priest_1982}.

As previously mentioned, convection is driven by fluids heated from below representing hot rising gas elements or \textit{convective cells} transporting heat to the photosphere~\cite{Plaskett_1936}. 
In the solar context, convection takes place in a highly compressible and stratified gas.
This physical regime leads to determine the conditions required for convection to occur and the resulting dynamics of the granules~\cite{Foukal_1990}. 

\subsection{Dynamics of solar convection}
From the core, He nuclei is built from H nuclei in the proton-proton chain as Equation~\eqref{eq:pp chain} refers.

\begin{equation}
    4^{1} \text{H} \rightarrow ^4\text{He} +2e^++2\nu +26.7 \text{MeV} 
    \label{eq:pp chain}
\end{equation}

The proton-proton chain reaction in the core liberates approximately $26.7$  MeV of energy in the form of high-frequency $\gamma$-rays, and $0.5$  MeV of energy in the form of neutrinos. 
However, the strong Coulomb repulsion between positively charged nuclei, which increases with the product of their nuclear charges, means that only the lightest elements have appreciable fusion probabilities.

As energy is transported outward, photons are frequently absorbed and re-emitted. 
This process reduces the radiative conductivity, which in turn increases the temperature gradient~\cite{Foukal_1990}. 
When the transported energy reaches the low photosphere, a portion of the radiation escapes into space, and the plasma returns to a state of convective stability~\cite{Priest_1982}.
These conditions establish the onset of convection. 

Consider a granule of plasma in local hydrostatic equilibrium with its surroundings, characterized by radial profiles of pressure $P(r)$, density $\rho(r)$, and temperature $T(r)$.
If the granule's temperature is increased to a value $T'$, it will expand adiabatically to maintain pressure equilibrium, thereby decreasing its density relative to its surroundings.

This lower-density gas then experiences a buoyancy force, causing it to rise. 
The buoyancy force persists until the granule's density matches that of its new surroundings after traveling a mixing length $l$.
Let $T_n'$ be the temperature of the rising element and $T_n$ the temperature of its new surroundings. 
The difference between the adiabatic gradient of the element and the radiative gradient of the surroundings governs the convection, as described by Equation~\eqref{eq:adiabatic radiative gradients}

\begin{equation}
    T_n = T + \parens{\frac{dT}{dr}}_R l \quad \big| \quad T_n' = T' + \parens{\frac{dT}{dr}}_{ad} l
    \label{eq:adiabatic radiative gradients}
\end{equation}

Where $R$ refers for radiative temperature gradient and $ad$ for the adiabatic temperature gradient. 
Convection occurs when the radiative gradient becomes steeper than the adiabatic gradient.
Since we have assumed radiative equilibrium, the $R$ denotes the gradient present in the stellar atmosphere.

The onset of convection leads to the inequality~\eqref{eq:radiative condition}, where two conditions can arise: The convection pattern is established when adiabatic gradient exceeds the radiative gradient; otherwise, the layer is stable and energy is transported by radiation.

\begin{equation}
    -\parens{\frac{dT}{dr}}_R > \parens{\frac{dT}{dr}}_{ad}
    \label{eq:radiative condition}
\end{equation}

If the convective pattern is established, the element continues to expand adiabatically as it rises, driven by buoyancy.
Otherwise, if the layer is stable, the element will contract, becomes heavier than its surroundings and begins to move down toward its original position.

This onset of instability, when the vertical temperature gradient is too large, is formally described by the Schwarzschild criterion for convection.

\subsection{The Schwarzschild criterion}
Taking the element described before, now suppose an elementary parcel of material displaced so slowly that remains in horizontal pressure equilibrium (see Figure~\ref{fig:parcel argument}). 

\begin{figure}[H]
    \centering
    \includegraphics[width=0.7\linewidth]{Parcel argument.jpg}
    \caption{Diagram for the parcel of material displaced so slowly that remains in horizontal pressure. Image taken from~\cite{Priest_1982}}\label{fig:parcel argument}%
\end{figure}

If the motion is adiabatic there is no heat exchange with surroundings, the pressure and density of the rising element are adiabatic.
This generates a criterion for the presence of convection known as the Schwarzchild criterion~\eqref{eq:Schwarzchild criterion}.

\begin{equation} 
    -\frac{dT}{dr} > \frac{\gamma - 1}{\gamma} \parens{\frac{GM_{\astrosun}m}{r^2k_B}}
    \label{eq:Schwarzchild criterion}
\end{equation}

Where $k_B$ refers to Boltzmann constant; $G$ the gravitational constant; $M_{\astrosun}$ the solar mass; $m$ the mass of the granule; and $\gamma$ the adiabatic index. 
The presence of convection reduces the temperature gradient from the higher value required for purely radiative transport to a adiabatic value~\cite{Foukal_1990}.
The material and energy transported by this process ended on the low photosphere, where the granules exhibit different properties that will be examined in subsequent sections.

\section{The Solar Photosphere} 
Since 1874, when Langley gave a detailed description of granulation on the photosphere, astronomers have studied the dynamics and reactions within Sun's outermost layer~\cite{Priest_1982}. 
A distinct pattern of granules with dynamic behavior is apparent, where individual cells continuously emerge and disappear (see Figure~\ref{fig:photosphere visual}).

\begin{figure}[H]
    \centering
    \includegraphics[width=0.6\linewidth]{Granulation pattern.jpg}
    \caption{A view of granulation on the Sun's surface. The central regions exhibit blueshifts while the edges display redshifts. Image taken from~\cite{Samir_pattern}.}\label{fig:photosphere visual}%
\end{figure}

The bright areas of granules correspond to regions where hot gas rises through the solar atmosphere. 
As this gas releases energy in the form of photons at the photosphere, it cools and subsequently descends, creating the darker regions of intergranular lanes~\cite{Carroll_Ostlie_2006}.

Furthermore, high-resolution observations reveal that these granules are in continual motion (see Figure~\ref{fig:motion granules}).

\begin{figure}[H]
    \centering
    \includegraphics[width=0.45\linewidth]{Granules motion.jpg}
    \caption{A time sequence showing granule evolution where the time intervals are about a minute. Image taken from~\cite{Foukal_1990}.}\label{fig:motion granules}%
\end{figure}

This perpetual motion across the photosphere generates asymmetries in absorption line profiles. 

\subsection{Static photosphere: Limb darkening phenomenon}

Because the temperature decreases outward through the photospheric layers, the observed intensity falls off towards the solar limb. 
Discovered by Halm in 1907~\cite{Dravins_1981}, this effect is known as \textit{Limb darkening phenomenon}, which make the disk intensity profile to appear more squared at increasing wavelength (see Figure~\ref{fig:squared profile}).

\begin{figure}[H]
    \centering
    \includegraphics[width=0.55\linewidth]{Limb darkening.jpg}
    \caption{Squared profile for the disk intensity at increasing wavelengths. Image taken from~\cite{Foukal_1990}.}\label{fig:squared profile}
\end{figure}

The analysis of this effect provides a direct technique for determining the temperature structure of the photosphere as a function of depth. 

For typical weaker lines, the convective blueshift diminishes toward the limb, with a net velocity change approximately of $400$  m/s.
As explain Ellwarth et al.~observations closer to the limb pass through the atmosphere a shallower angle.
This results in longer optical paths through high atmospheric layers, allowing the study of the layers where convective blueshift is less~\cite{Ellwarth_2023}.

\subsection{Dynamic photosphere: The C-curved profile bisector.}
Analysis of granule dynamics reveals a height dependence of convective velocities: The vertical velocity of a rising granule decays less rapidly than its excess brightness. 
Changes in the granulation structure, contrast and velocity field around the spots and magnetic network have been inferred indirectly from observations of Fraunhofer line profile shapes~\cite{Foukal_1990}. 
A direct result of the correlation between brightness and velocity of granules is the characteristic C-curved line profile bisector observed in photospheric absorption lines (see Figure~\ref{fig:c curved profile}).

\begin{figure}[H]
    \centering
    \includegraphics[width=0.75\linewidth]{C curved profile bisector.jpg}
    \caption{The C-curved line profile bisector. In the infrared shown to be less pronounced than violet range. Imagen taken from~\cite{Dravins_1981}.}\label{fig:c curved profile}%
\end{figure}

The formation of the C-curved line profile bisector occurs in three stages, corresponding to different heights in the photosphere.

First, the mid-depth portion of the line profile is formed in the brightest upflowing material, resulting in a blueshift.
Then, the deepest part of the line core is formed higher up, in a region of decelerated upflow, producing a smaller blueshift.
Finally, the line wings where the opacity is lowest, tend to be formed deepest in the cool material, resulting in a redshift.

This dynamic process induces characteristic perturbations in spectral lines, manifesting as line broadenings, line profile asymmetries, and line depth-dependent wavelength shifts known as the three signatures of convection~\cite{Gray_Pugh_2012}.

\section{The three signatures of convection}
The signatures of convection in stars are described by Gray in his research (see~\cite{Gray_2009,Gray_Oostra_2018,Gray_Pugh_2012} and references therein) as the principal characteristics to identify and treat the asymmetries due to convective motion.

\subsection{First signature of convection: Line broadening}
Due to the limb darkening phenomenon and atmospheric absorption, there exists two types of line cores: Strongest and weaker lines~\cite{Gray_Pugh_2012}.
In general, stronger absorption lines exhibit more pronounced core curvature than weaker lines.
This line core curvature $C_c$ can be quantified by the second derivative of the line's intensity profile with respect to wavelength, evaluated at observed wavelength (see Equation~\eqref{eq:core curvature}).
\begin{equation}
    C_c =  \lambda_{obs}^2 \parens{\frac{d^2 p(\lambda_{obs})}{d \lambda_{obs}^2}}
    \label{eq:core curvature}
\end{equation}

In the near infrared range, the solar spectrum contains natural bands or groups of lines that are separated by regions of strong absorption from telluric elements. 
In other words, elements like $\text{O}_2$ and $\text{CO}_2$ that are absorbed by the atmosphere.
This phenomenon modify the line core curvature of determining lines profiles, specially in the near infrared range.

\subsection{Second signature of convection: Line profile asymmetry}
As demonstrated by Nieminen~\cite{Nieminen_2017}, the asymmetry occurs because the c-curved profile bisector reflects velocity variations and a bisector slope (see Figure~\ref{fig:asymmetries on lines}).

\begin{figure}[H]
    \centering
    \includegraphics[width=0.7\linewidth]{Asymmetries on typical lines.jpg}
    \caption{Asymmetries on an average absorption line. Can be observed the differences in intensity redshift profile. Image taken from~\cite{Nieminen_2017}.}\label{fig:asymmetries on lines}%
\end{figure}

According to Kirchhoff's laws, absorption line formation requires lower temperature conditions, which are found precisely in the Sun's outermost atmospheric layers~\cite{Carroll_Ostlie_2006}. 
These regions not only provide the appropriate temperatures for absorption but also exhibit comparatively higher opacity. 
Those spectral lines from Fe I are particularly valuable for solar granulation studies due to their high abundance, minimal thermal broadening, limited isotopic variation and availability of accurate laboratory measurements of natural wavelengths~\cite{Nieminen_2017}. 

These asymmetries imprinted on the Fe I lines can be quantified by analyzing their bisectors. 
A convenient method for this measures involves using the third derivative of the line profile, which provides the slope of the lowest end of the bisector (see Equation~\eqref{eq:third derivative relation}). 
\begin{equation}
    -\frac{c}{\lambda_{obs}}\parens{\frac{1}{3 C_c^2}}\parens{\frac{d^3 p(\lambda_{obs})}{d \lambda_{obs}^3}}
    \label{eq:third derivative relation}
\end{equation}

The slope of the line profile is defined as zero when the line profile bisector is vertical and the line core is symmetric.
The relation is derived in appendix\ref{ap:third derivative}.

Since most of stellar observations are made with lower-resolution spectrographs and often lower signal-to-noise ratios, there is a need to study granulation through its signatures.
Gray and Oostra identified a form of line profile asymmetry observable under these conditions, known as the \textit{flux deficit}. 
This phenomenon can be interpreted as flux imbalances on the redshifted side of the line profile, caused by a net loss of radiative energy. 

To quantify this asymmetry, the authors proposed calculating the flux deficit by taking the difference between the observed, altered profile and a the granulation curve.
By treating the altered line profile as an approximate representation of the third signature of convection, one can observe the flux difference interpreted as radiation~\cite{Gray_Oostra_2018}.

\subsection{Third signature of convection: Line depth-dependent wavelength shifts}
Many studies across the years have detected and observed the phenomenon of wavelength shifts against the line depth, or as it is called, convective redshift.
This behavior was shown to be more present in the weaker lines which are related to infrared and violet range.

\subsection{Solar granulation pattern}
The solar granulation pattern is a plot of relative velocity against line depth, as shown Figure~\ref{fig:granulation pattern ellwarth}.

\begin{figure}[H]
    \centering
    \includegraphics[width=0.75\linewidth]{Granulation pattern Ellwarth.jpg}
    \caption{Granulation pattern for the IAG spatially resolved quiet sun atlas, is shown to have a strong blueshift in the weaker lines. Image taken from~\cite{Ellwarth_2023}.}\label{fig:granulation pattern ellwarth}%
\end{figure}

Notice in Figure~\ref{fig:granulation pattern ellwarth} the wavelength-dependence os the pattern.
The significance of the granulation pattern lies in its universality for solar-type stars; their patterns closely resemble the Sun's, differing primarily by a scaling factor (see~\cite{Gray_Pugh_2012}). 
Consequently, a detailed analysis contributes to the understanding and radiation of photospheric hydrodynamic models~\cite{Dravins_1981,Gray_2009}.
However, a precise measurement of the granulation pattern requires a understanding of the convective blueshift phenomenon.


\subsection{Convective Blueshift}
When the Sun pushes material up through its outer layer, the spectrum exhibits a blueshift. 
As this material subsequently cools and falls back through the atmosphere, it produces a redshift, but emits less light, making the blueshift dominant.

That can be measured by the relativistic Doppler effect. This effect accounts for length contraction, as predicted by Einstein’s theory of relativity. 
However, the measured negative redshift resulting from convective motions is known as \textit{convective blueshifts} which are measured by the Equation~\eqref{eq:relative velocity}. 
\begin{equation}
    v_r \approx c \parens{\frac{ \lambda_{obs}- \lambda_{rep}}{\lambda_{rep}}}
    \label{eq:relative velocity}
\end{equation}

Since the strength of the convective distortions and shifts of spectral lines vary across the H-R diagram, we expect systematic errors in radial velocities~\cite{Gray_2009}.

\subsection{Chromodependence characterization}
Recent emphasis on measuring the granulation pattern has led to new interpretations of line depth-dependent wavelength shifts.

Gray and others authors have cualitatively explained the line-depth dependence on convective blueshift.
Shallow lines come preferably from deep photospheric layers where convection is strong, so the convective blueshift is great.
As long as deep lines come preferably from superficial layers, where gravity has slowed the convection and the blueshift is small.
However, an explanation of why diagonal trend depends on color, or chromodependence along te line depth, has not been found.

In 2018, Gray and Oostra attempted to establish a standard curve determined by a third order polynomial fit to the solar granulation pattern (see Figure~\ref{fig:Gray standard curve}).

\begin{figure}[H]
    \centering
    \includegraphics[width=0.6\linewidth]{Images/Standard curve gray.jpg}
    \caption{Standard curve proposed by Gray and Oostra on the spectral range the spectral range of $\SI{6020}{\angstrom}$ to $\SI{6340}{\angstrom}$. Image taken from~\cite{Gray_Oostra_2018}.}\label{fig:Gray standard curve}%
\end{figure}

Nevertheless, the authors limited the spectral range of $\SI{6020}{\angstrom}$ to $\SI{6340}{\angstrom}$, avoiding the chromodependence which becomes evident in a wider spectral range.
This limitation motivates the present work to develop a strong characterization and give the methodology for dealing with this phenomenon.

A separate theoretical perspective, offered by Hamilton and Lester, attributes aspects of photospheric dynamics to rotation.
The pronounced differential rotation with latitude observed seems to be the result of convective flows driven radially by the buoyancy force and deflected horizontally by the Coriolis force~\cite{Foukal_1990}.
This rotation contributes to angular momentum through two forms: Meridional circulation and Reynolds stresses.

\begin{figure}[H]
    \centering
    \includegraphics[width=0.6\linewidth]{Angular momentum.jpg}
    \caption{Contributions on angular momentum. Image taken from~\cite{Foukal_1990}.}\label{fig:angular momentum}%
\end{figure}

The first contribution involves axisymmetric meridional circulation (see Figure~\ref{fig:angular momentum}). 
In the absence of other transport mechanism, such a circulation would tend to spin up the poles and the interior, as the moving fluid conserves its angular momentum.

The second contribution, which tends to enforce solid body rotation, arises from Reynolds stresses associated with nonaxisymmetric convective motions.
The mechanism can be understood by considering that, for equal velocities in the meridional plane, the flux of angular momentum per unit mass is greater toward the equator than away from it. 
This process depends critically on nonaxisymmetric convection, as it generates net fluxes of angular momentum in latitudinal and radial directions without a corresponding net mass flux.
Neither buoyancy forces, which are strictly radial, nor pressure gradients, which must average to zero around the solar circumference, can directly influence the sun's axisymmetric rotation profile~\cite{Foukal_1990}.
Therefore, the combined action of meridional circulation and Reynolds stresses is essential for shaping the observed differential rotation.

\textcolor{Miku}{ADD BROADENING EFFECTS AND EXPLAIN THEM}