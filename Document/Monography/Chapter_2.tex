As previously mentioned, David Gray has significantly advanced the study of granulation patterns in the solar photosphere, with a particular focus on measuring relative velocities with high precision.
This chapter explores the physical origins of the three signatures of convective motion in the solar photosphere, and how this reveals the hydrodynamics on the outermost layer. 

\section{The solar interior and the solar outer atmosphere}

The Sun is classified as a yellow dwarf star of spectral type G2V.
Its chemical structure is primarily composed of a large fraction of ionized hydrogen and a smaller proportion of helium.
What makes the Sun unique in astronomical studies is its proximity to Earth, which allows for detailed observation unmatched by any other star.
Structurally, the Sun is divided into two main regions: The solar interior and the solar outer atmosphere. 

As illustrated in Figure~\ref{fig:Sun interior} the overall structure of the solar interior is core, radiative and convective zone. 
In the core, He nuclei are built from H nuclei in the proton-proton chain as Equation~\eqref{eq:pp chain} refers.
\begin{equation}
    4^{1} \text{H} \rightarrow ^4\text{He} +2e^++2\nu +26.7 \text{MeV} 
    \label{eq:pp chain}
\end{equation}
The proton-proton chain reaction in the core liberates approximately $26.7$  MeV of energy in the form of high energy $\gamma$-rays, and $0.5$  MeV of energy in the form of neutrinos. 
In this zone, standard models estimate a temperature of $1.6\times 10^7$  K and density to $1.6\times 10^5$  kg/$\text{m}^3$.
Moving outward through the layers, both the density and temperature decrease significantly, as the energy is slowly transferred outwards by radiative diffusion~\cite{Foukal_1990}.
This process progressively shifts the wavelength of the radiation from high energy $\gamma$-rays to the visible light that eventually escapes.
The large temperature combined with the high density, allows the absorptions and remissions of photons that make this zone highly opaque and maintain the central material in a plasma state, functioning like a massive nuclear reactor.

\begin{figure}[H]
     \centering
     \begin{subfigure}{0.48\textwidth}
         \includegraphics[width=\textwidth]{Sun interior.jpg}
         \caption{The interior structure of the Sun. The convection zone is responsible for the general movement that characterizes the granulation patterns.}\label{fig:Sun interior}%
     \end{subfigure}
\hfill
     \begin{subfigure}{0.48\textwidth}
         \includegraphics[width=\textwidth]{Sun exterior.jpg}
         \caption{The outer structure of the Sun. The photosphere is the layer of the Sun where the convection cells overshoot from the convection zone.}\label{fig:Sun outer atmosphere}%
     \end{subfigure}

        \caption{The general structure of the Sun. Images modified from~\cite{Priest_1982}.}\label{fig:Sun structure}%
\end{figure}

On the other hand, Figure~\ref{fig:Sun outer atmosphere} illustrates the overall structure of the solar outer atmosphere consisting of the photosphere, chromosphere and corona. 
In these layers, the density decreases rapidly with height above the solar surface. 
However, the temperature decreases to a minimum of approximately $4300$  K in the upper photosphere before rising through the chromosphere and transition region to millions of degrees in the corona.
From that point, the temperature falls slowly expanding outwards as the solar wind. 

Nevertheless, the relevant layers for this project are the photosphere, a thin layer of plasma that emits most of the solar radiation; and the convection zone, in which all the convection process take place.
The radiation chain process results on the emission of a continuous spectrum passes through the overlying photosphere.
Then specific wavelengths are absorbed for this layer, resulting in the characteristic Fraunhofer lines superimposed on the emitted spectrum, which allows observing the convection consequences~\cite{Priest_1982}.

\section{The solar convection zone}

In 1885 Janssen obtained the first clear photograph of photospheric granules (see Figure~\ref{fig:Janssen photography}) providing the initial evidence and the starting point for numerous studies about granulation~\cite{Malherbe_2022}.

\begin{figure}[H]
    \centering
    \includegraphics[width=0.45\linewidth]{First take photosphere.jpg}
    \caption{The first clear photograph of the photosphere where the granulation pattern is visible taken by Janssen in 1885. Image taken from~\cite{Malherbe_2022}.}\label{fig:Janssen photography}%
\end{figure}

Following the history, in 1930, Unsöld theorized that the layers beneath the photosphere should be convective unstable~\cite{Foukal_1990}. 
This hypothesis was later supported by Plaskett when he related the observed granules to the convective cells studied in Bénard's laboratory experiments~\cite{Plaskett_1936}.
In this analogy, a fluid heated from below develops rising elements of hot gas that transport heat to the surface.

The elements of hot gas rising transporting heat are called \textit{convective cells}, and the pattern generated by several cells on the photosphere is the \textit{granulation} with each individual region referred to as a \textit{granule}. 
Typical granules span approximately 700 km and have short lifetimes, lasting between five to ten minutes.

In the solar context, convection takes place in a highly compressible and stratified gas located between $0.86 R_{\astrosun}$ and the surface, affected by a large temperature gradient.
This physical regime leads to determine the conditions required for convection to occur and the resulting dynamics of the granules~\cite{Foukal_1990}. 

\subsection{The Schwarzschild criterion}
Consider an elementary parcel of material displaced in local hydrostatic equilibrium with its surroundings, characterized by radial profiles of pressure $P(r)$, density $\rho(r)$, and temperature $T(r)$.
If the granule temperature is increased to a value $T'$, it will expand adiabatically to maintain pressure equilibrium, thereby decreasing the density relative to its surroundings (see Figure~\ref{fig:parcel argument}).

\begin{figure}[H]
    \centering
    \includegraphics[width=0.7\linewidth]{Parcel argument.jpg}
    \caption{Diagram for the parcel of material displaced so slowly that the only force it feels is the pressure in a direction parallel to itself, keeping it in a constant horizontal movement. Image taken from~\cite{Priest_1982}.}\label{fig:parcel argument}%
\end{figure}

This convective cell experiences a buoyancy force, causing it to rise. 
The buoyancy force persists until the granule's density matches that of its new surroundings after traveling a length $l$.
Let $T_n'$ be the temperature of the rising element and $T_n$ the temperature of its new surroundings. 
The difference between the adiabatic gradient of the element and the radiative gradient of the surroundings governs the convection, as described by Equation~\eqref{eq:adiabatic radiative gradients}
\begin{equation}
    T_n = T + \parens{\frac{dT}{dr}}_R l \quad \big| \quad T_n' = T' + \parens{\frac{dT}{dr}}_{ad} l
    \label{eq:adiabatic radiative gradients}
\end{equation}
Where $R$ refers for radiative temperature gradient and $ad$ for the adiabatic temperature gradient. 
The onset of convection leads to the inequality~\eqref{eq:radiative condition}, where two conditions can arise: The granulation is established when adiabatic gradient exceeds the radiative gradient; otherwise, the layer is stable and energy is transported by radiation.
\begin{equation}
    -\parens{\frac{dT}{dr}}_R > \parens{\frac{dT}{dr}}_{ad}
    \label{eq:radiative condition}
\end{equation}
If the granulation is established, the element continues to expand adiabatically as it rises, driven by buoyancy.
Otherwise, if the layer is stable, the element will contract, becomes heavier than its surroundings and begins to move down toward its original position.

This onset of instability, when the vertical temperature gradient is too large, is known as the Schwarzschild criterion for convection.
Conveniently this criterion is expressed in terms of the relation between $T$, $P$ and $\gamma$ (heat capacity ratio) for an adiabatic change (see Equation~\eqref{eq:Schwarzchild criterion}).

\begin{equation} 
    -\parens{\frac{dT}{dr}}_R  > \frac{\gamma - 1}{\gamma} \parens{\frac{T}{P}}\parens{-\frac{dP}{dr}}_{ad}
    \label{eq:Schwarzchild criterion}
\end{equation}

In this form, the condition establishes that convection can occur when opacity rises rapidly, due to increasing the population of $n=3$ level of hydrogen and $\gamma$ lowered by ionization.
The material and energy transported by this process ended on the low photosphere, where the granules exhibit different properties that will be examined in subsequent sections.

\section{The solar photosphere} 
As mentioned in the previous section, the observation of this layer leads to the study of dynamics and reactions within Sun's outermost layer. 
From surface observations a distinct pattern of granules with dynamic behavior is apparent, where individual cells continuously emerge and disappear (see Figure~\ref{fig:photosphere visual}).

\begin{figure}[H]
    \centering
    \includegraphics[width=0.65\linewidth]{Granulation pattern.jpg}
    \caption{A view of granulation on the Sun's surface. Image taken from~\cite{Samir_pattern}.}\label{fig:photosphere visual}%
\end{figure}

The bright areas of granules correspond to regions where hot gas rises through the solar atmosphere.
As this gas releases energy in the form of photons at the photosphere, it cools and subsequently descends, creating the darker regions of intergranular lanes~\cite{Carroll_Ostlie_2006}.
Furthermore, high-resolution observations reveal that these granules are in continual motion generating asymmetries in absorption line profiles (see Figure~\ref{fig:motion granules}).

\begin{figure}[H]
    \centering
    \includegraphics[width=0.65\linewidth]{Granules motion.jpg}
    \caption{A time sequence showing granule evolution where the time intervals are about a minute. Image taken from~\cite{Foukal_1990}.}\label{fig:motion granules}%
\end{figure}

\subsection{Static photosphere: Limb darkening phenomenon}

EXPLICCAR BREVEMENTE EL NUEVO MODELO, EL TRANQUILO, Y PORQUE S EPUEDE HABLAR DE OSCURECIMIENTO

Because the temperature decreases outward through the photospheric layers, the observed intensity falls off towards the solar limb. 
Discovered by Halm in 1907~\cite{Dravins_1981}, this effect is known as \textit{limb darkening}, which makes the disk intensity profile to appear more squared at increasing wavelength (see Figure~\ref{fig:squared profile}).

\begin{figure}[H]
    \centering
    \includegraphics[width=0.6\linewidth]{Limb darkening.jpg}
    \caption{Squared profile for the disk intensity at increasing wavelengths, where $5\mu m$ refers to the infrared range and $0.32\mu m$ the violet range. Image taken from~\cite{Foukal_1990}.}\label{fig:squared profile}
\end{figure}

The analysis of this effect provides a direct technique for determining the temperature structure of the photosphere as a function of line depth. 

For typical weaker lines, the convective blueshift diminishes toward the limb, with a net velocity change approximately of $400$  m/s.
As explain Ellwarth et al.~observations closer to the limb pass through the atmosphere a shallower angle, resulting in longer optical paths through high atmospheric layers which allows the study of the layers where convective blueshift is less pronounced~\cite{Ellwarth_2023}.
Because of this phenomenon and following the objective of study the line depth-dependence of convective blueshift, the center disk flux spectrum was taken as reference for analysis, where the limb darkening effect is neligible.

\subsection{Dynamic photosphere: The C-curved profile bisector.}
Analysis of changes, contrast and velocity field in the granulation structure have been inferred indirectly from observations of Fraunhofer line profile shapes~\cite{Foukal_1990}.
The observations on absorption lines reveals that velocity of a rising granule decays less rapidly than its excess brightness, resulting in a characteristic C-curved line profile bisector (see Figure~\ref{fig:c curved profile}).

\begin{figure}[H]
    \centering
    \includegraphics[width=0.75\linewidth]{C curved profile bisector.jpg}
    \caption{The C-curved line profile bisector with the corresponding diagram of wavelength displacement due to convective blueshift. Imagen taken from~\cite{Dravins_1981}.}\label{fig:c curved profile}%
\end{figure}

The formation of the C-curved line profile bisector occurs in three stages, corresponding to different heights in the photosphere.

First, the deepest part of the line profile is formed higher up, in a region of decelerated upflow, producing a smaller blueshift.
Then, the mid-depth portion is formed in the brightest upflowing material, resulting in a blueshift.
Finally, the line profile wings where the opacity is lowest, tend to be formed deepest in the cool material, producing a redshift.

This dynamic process induces characteristic asymmetries on the line profile bisector, which becomes an important instrument to measure convection process in the solar atmosphere.

\section{The three signatures of convection}
The three signatures of convection in stars are described by David Gray in his research as the principal characteristics to identify the convective motion through the spectrum (see~\cite{Gray_2009,Gray_Oostra_2018,Gray_Pugh_2012} and references therein).

\subsection{First signature of convection: Line broadening} 
To explain the line broadening is necessary establish the process that creates a spectral line. 
An individual atom making a transition between energy levels emits a photon with certain frequency.
This transition can be represented as a graph of radiance or intensity per unit wavelength against wavelength, what is called \textit{line profile}~\cite{Carroll_Ostlie_2006}.
The radiated intensity can be modeled passing through a hot cloud of gas in thermal equilibrium as Equation~\eqref{eq:radiance absorption line}.
\begin{equation} 
    I_{\nu}(\tau_{\nu}) = I_0 e^{-\tau_{\nu}} + B_{\nu}\parens{1-e^{-\tau_{\nu}}}
    \label{eq:radiance absorption line}
\end{equation}
Where $\tau_{\nu}$ refers to the line depth and $B_{\nu}$ the absorption coefficient for the gas. 
The dynamical and atomic processes on the photosphere causes the thermal, convection, rotation, pressure, Stark, Zeeman and natural broadening effects.
Because we measure the broadening of the line cores, which is affected mostly by Doppler effects, whereas pressure and natural broadening affect the wings, and Zeeman is negligible except in sunspots; only the thermal, rotation and convection broadening effects were studied. 

Atoms in a gas have random motions with temperature dependence, which mean speed is obtained by the relation between kinetic and thermal energy for gasses~\cite{Van_1965}. 
The fraction of atoms in a speed interval between $v$ and $v + \Delta v$ is then given by the Maxwell-Boltzmann distribution in Equation~\eqref{eq:Gaussian distribution}.
\begin{equation} 
    f(v_{\text{rad}}) = \exp\parens{\frac{-mv_{\text{rad}}^2}{2k_{B}T}}
    \label{eq:Gaussian distribution}
\end{equation}
Comparing with a typical Gaussian distribution centered on the origin, we can relate the width $(\sigma^2)$ to the variance of the radial velocity (see Equation~\eqref{eq:velocity variance}).
\begin{equation} 
    	f(x)= \exp\parens{\frac{-x^2}{2\sigma^2}} \quad \rightarrow \quad \langle v_{\text{rad}}^2 \rangle = \sigma^2 = \frac{k_{B}T}{m} 
    \label{eq:velocity variance}
\end{equation}
Using the relation of Doppler effect for the radial velocity and relation~\eqref{eq:velocity variance} the line profile with only Doppler broadening effect is described by Equation~\eqref{eq:Doppler broadening}.
\begin{equation} 
    f(\Delta \lambda) = L_D\exp\parens{\frac{-mc^2}{2\lambda^2k_{B}T}\Delta \lambda^2}
    \label{eq:Doppler broadening}
\end{equation}
Where $L_D$ refers to the line depth, $k_B$ to the Boltzmann constant; and $c\Delta \lambda/\lambda$ to the radial velocity of the observed atom.
The Equation~\eqref{eq:core curvature} describes the line core curvature can be quantified by the second derivative of the line's intensity profile with respect to wavelength, evaluated at observed wavelength.
\begin{equation}
    \lambda_{obs}^2 \parens{\frac{d^2 f(\lambda_{obs})}{d \lambda_{obs}^2}}
    \label{eq:core curvature}
\end{equation}
On a plot of $|f''(0)|\lambda^2/L_D$ against line depth, a theorical slope can be derived from Equation~\eqref{eq:Doppler broadening} using the definition of line core curvature as shows the Relation~\eqref{eq:theory slope only doppler}.
\begin{equation}
    |f''(0)| = L_D \parens{\frac{mc^2}{2\lambda^2k_{B}T}} \quad \rightarrow \quad \frac{|f''(0)|\lambda^2}{L_D} = \frac{mc^2}{2k_{B}T}
    \label{eq:theory slope only doppler}
\end{equation}
Where $\Delta \lambda=0$ due to the origin-centered Gaussian profile.
Equation~\eqref{eq:theory slope only doppler} represents the line core curvature slope for lines which are only affected by the thermal broadening effect.
Directly, assuming the three target effects as independent\footnote{This is not completely true, for it is known that greater temperatures lead to more negative convective velocities, implying a correlation between thermal and convective speeds. For simplicity, the approximation was taken on the project.} the variance of the total radial velocity is the sum of the variances of the thermal, rotational and convective effects.
This leads to the Equation~\eqref{eq:Theory slope} of a theoretical slope including the three broadening effects.
\begin{equation}
    \frac{|f''(0)|\lambda^2}{L_D} = \frac{c^2}{\langle v_{\text{r}}^2 \rangle + \langle v_{\text{T}}^2 \rangle + \langle v_{\text{conv}}^2 \rangle}  
    \label{eq:Theory slope}
\end{equation}
Where $\langle v_{\text{r}}^2 \rangle$ refers to the variance of the rotation velocity; $\langle v_{\text{T}}^2 \rangle$ is the variance of the thermal velocity; and $\langle v_{\text{conv}}^2 \rangle$ refers to the variance of the convection velocity.
Furthermore, the value for Fe atom mass $55.85$ g/mol; the solar effective temperature $5770$ K leads to the values of thermal velocity variance of $0.86$ $(\text{km}/\text{s})^2$.

The variance of rotation velocity is $0.90$ $(\text{km}/\text{s})^2$, this result was derived by professor Benjamin using a spherical and solid model of the Sun.
In the IAG spatially resolved quiet Sun atlas the value of $\langle v_{\text{r}}^2 \rangle$ is neligible.
With these values, an approximation of convection variance of velocity is expected to be calculated for each atlas.

\subsection{Second signature of convection: Line profile asymmetry}
According to Kirchhoff's laws, absorption line formation requires lower temperature conditions, which are found precisely in the Sun's outermost atmospheric layers~\cite{Carroll_Ostlie_2006}. 
These regions not only provide the appropriate temperatures for absorption but also exhibit comparatively higher opacity. 
Those spectral lines from Fe I are particularly valuable for solar granulation studies due to their high abundance, minimal thermal broadening, limited isotopic variation and availability of accurate laboratory measurements of natural wavelengths~\cite{Nieminen_2017}. 
Due to the useful characteristic of this line dataset, asymmetries imprinted can be quantified by analyzing their bisectors. 

A convenient method for this measures involves using the third derivative of the line profile, which provides the slope of the lowest end of the bisector (see Equation~\eqref{eq:third derivative relation}). 
\begin{equation}
    -\frac{c}{\lambda_{obs}}\parens{\frac{1}{3 \frac{d^2 f(\lambda_{obs})}{d \lambda_{obs}^2}}}\parens{\frac{d^3 f(\lambda_{obs})}{d \lambda_{obs}^3}}
    \label{eq:third derivative relation}
\end{equation}

The slope of the line profile is defined as zero when the line profile bisector is vertical and the line core is symmetric.
The relation~\eqref{eq:third derivative relation} is derived in appendix\ref{ap:third derivative}.

As mentioned before, line profile asymmetries are an important instrument to measure convection processes from the solar atmosphere.
Since most of stellar observations are made with lower-resolution spectrographs and often lower signal-to-noise ratios. 


\subsection{Third signature of convection: Line depth-dependent wavelength shifts}
Many studies across the years have detected and observed the phenomenon of wavelength shifts against the line depth, or as it is called, convective blueshift.

\subsection{The third signature plot}
The third signature plot relates relative velocity against line depth, as shown Figure~\ref{fig:granulation pattern ellwarth}.

\begin{figure}[H]
    \centering
    \includegraphics[width=0.65\linewidth]{Granulation pattern Ellwarth.jpg}
    \caption{The third signature plot for the IAG spatially resolved quiet Sun atlas, shows a strong trend blueshift in the shallow lines. Image taken from~\cite{Ellwarth_2023}.}\label{fig:granulation pattern ellwarth}%
\end{figure}

Notice in Figure~\ref{fig:granulation pattern ellwarth} the wavelength-dependence on convective blueshift, which has been extensively documented for several datasets.
The significance of the third signature plot lies in its universality for solar-type stars; their plots closely resemble the Sun's, differing primarily by a scaling factor~\cite{Gray_Pugh_2012}. 
Consequently, a detailed analysis contributes to the understanding and radiation of photospheric hydrodynamic models~\cite{Dravins_1981,Gray_2009}.
However, a precise measurement of this pattern requires a understanding of the convective blueshift phenomenon.

\subsection{Convective Blueshift}
The measured negative redshift resulting from convective motions is known as \textit{convective blueshifts}, which is measured using the Doppler effect (see Equation~\eqref{eq:relative velocity}). 
\begin{equation}
    v_r \approx c \parens{\frac{ \lambda_{obs}- \lambda_{em}}{\lambda_{em}}} -633\text{m/s}
    \label{eq:relative velocity}
\end{equation}
Where the value of $633$ m/s is the correction of gravitational redshift for the Sun; and the $\lambda_{em}$ represents the emitted wavelength.

When the Sun pushes material up through its outer layer, the spectrum exhibits a blueshift. 
As this material subsequently cools and falls back through the atmosphere, it produces a redshift, but emits less light, making the blueshift dominant.
Since the strength of the convective distortions and shifts of spectral lines vary across the H-R diagram, we expect systematic errors in radial velocities~\cite{Gray_2009}.

\subsection{Chromodependence characterization}
Recent emphasis on measuring the third signature plot has led to new interpretations of line depth-dependent wavelength shifts.

Gray and others authors have cualitatively explained the line-depth dependence of convective blueshift.
Shallow lines come preferably from deep photospheric layers where convection is strong, so the convective blueshift is great.
Whereas deep lines come preferably from superficial layers, where gravity and the demise of buoyancy has slowed the convection and the blueshift is small.
However, an explanation of why this diagonal trend depends on color, or chromodependence of the line depth, has not been found.

In 2018, Gray and Oostra established a standard curve determined by a third order polynomial fit to the solar granulation pattern (see Figure~\ref{fig:Gray standard curve}).

\begin{figure}[H]
    \centering
    \includegraphics[width=0.6\linewidth]{Images/Standard curve gray.jpg}
    \caption{Standard curve proposed by Gray and Oostra on the spectral range of $\SI{6020}{\angstrom}$ to $\SI{6340}{\angstrom}$. Image taken from~\cite{Gray_Oostra_2018}.}\label{fig:Gray standard curve}%
\end{figure}

Nevertheless, the authors limited the spectral range of $\SI{6020}{\angstrom}$ to $\SI{6340}{\angstrom}$, avoiding the chromodependence which becomes evident in a wider spectral range.
This limitation motivates the present work to develop a characterization taking into account all wavelength ranges; and give the methodology for dealing with this phenomenona.

A separate theoretical perspective, offered by Hamilton and Lester, attributes aspects of photospheric dynamics to rotation.
The pronounced differential rotation with latitude observed seems to be the result of convective flows driven radially by the buoyancy force and deflected horizontally by the Coriolis force~\cite{Foukal_1990}.

\section{Anomalous chromodependence}
In previous research many authors pointed out the phenomenon that we call chromodependence or wavelength dependence of the signatures, emphasizing on the third granulation plot. 
Because this definition can be ambiguous a clarification is presented below.

In the photosphere model the temperature is higher in the deepest layers than the surface. 
For the formation of an absorption line, the temperature provides the atom with the energy necessary to be prepared in the lowest energy level. 
This is defined as the excitation potential $(\chi)$. 
While lines with higher values of $(\chi)$ are formed in the deepest layers, lines with smaller values of $(\chi)$ are formed near the surface. 

However, we recall the fact mentioned about the opaqueness of the photosphere, the lines formed on the surface are stronger than those produced in deeper layers.   
This results in absorption lines in the violet range, produced when the atom absorbs a high value of energy, formed near the surface implying a lower excitation potential and a small temperature. 
This is a chromodependence, but is not unexpected in the granulation patterns.

In this scheme, atomic and photospheric structure combine to make blue lines stronger and red lines weaker. 
This might be called \textit{Normal chromodependence} and should be characterized by a unique granulation curve, because weaker lines experience also a stronger convection.
What this project tried to characterize is the anomalous chromodependence, the observational fact of granulation patterns showing a different granulation curve for every wavelength range.
