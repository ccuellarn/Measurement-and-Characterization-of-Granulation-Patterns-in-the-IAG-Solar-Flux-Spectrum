As previously mentioned, David Gray \cite{Gray_2009,Gray_Pugh_2012,Gray_Oostra_2018} has significantly advanced the study of granulation patterns in the solar photosphere, with a particular focus on measuring their associated relative velocities with high precision.
These developments have enabled more accurate characterizations of other stars by extrapolating the physical principles observed in the Sun.
This chapter explores the three signatures of convective motion in the Sun photosphere, and how this reveal the hydrodynamic on this outermost layer. 
Furthermore, is given the point of view of different authors respect the reasons behind the chromodependece on the granulation patterns.

\section{The solar interior and the solar outer atmosphere}

The Sun is classified as a yellow dwarf star of spectral type G2V, title achieved for a big amount of hot hydrogen (ionised H in 90 percent) and helium (in 10 percent).
What makes this star really unique is his proximity to earth and the facility to study from the planet with precision.
In general, the Sun is divided in two fundamental parts: The solar interior and the solar outer atmosphere. 

\begin{figure}[H]
    \centering
    \includegraphics[width=0.6\linewidth]{Sun interior structure.jpg}
    \caption{The interior structure of the Sun. The convection zone is the responsible by the general movement that characterize the third signature. Image taken from \cite{Priest_1982}.}
    \label{Interior structure}
\end{figure}

As shown in the figure \ref{Interior structure} the overall structure of the solar interior is core, radiative and convective zone. 
Across then the density and temperature falls significantly, as the energy is slowly transferred outwards by radiative diffusion.
Some models of the interior structure give a core temperature of $1.6\times 10^7 K$ and density to $1.6\times 10^5 Kg/m^3$, high enough for thermonuclear reactions and remains the central material in plasma like a gigantic atomic reactor. 
This characteristic allow the collisions, absorptions and reemisions of photons that made this zone opaque. 
In consequence, there exists an increase of the wavelength from high-energy gamma rays to visible ligth.

\begin{figure}[H]
    \centering
    \includegraphics[width=0.6\linewidth]{Sun interior structure.jpg}
    \caption{The outer structure of the Sun. The photosphere is the layer of the sun where is visible the convection cells overshooting from the convection zone. Image taken from \cite{Priest_1982}.}
    \label{Outer structure}
\end{figure}

On the other hand, the figure \ref{Outer structure} shown the overall structure of the solar outer atmosphere which consist in photosphere, chromosphere and corona. 
In this part the density decreases rather rapidly with height above solar surface, and the temperature decrease to $4300$ $K$ for then rises through the transition region.
Thereafter, the temperautre falls slowly expanding outwards as the solar wind. 
The most relevant layer is the photosphere, a thin layer of plasma that emits most of the solar radiation and emits a continuos spectrum with superimposed dark absorption lines.
Most of this wavelengths are absorbed by the chromosphere, which is transparent\cite{Priest_1982}.
From this the photosphere emits a continuous spectrum with superimposed dark absorption lines where most of this wavelengths are absorbed by the chromosphere \cite{Priest_1982}.

The target layers of this study are the conevction zone and the photosphere, which will focus in further sections.

For this research, the two target layers of the Sun are the convection zone and the photosphere. 

\begin{figure}[H]
    \centering
    \includegraphics[width=0.6\linewidth]{First take photosphere.jpg}
    \caption{The first clear photgraphy of the photosphere where is visble the granulation pattern. Image taken from \cite{Malherbe_2022}}
    \label{Janssen photography}
\end{figure}

As we shown in the figure \ref{Janssen photography}, in 1877 Janssen took the first clear photography of the granules in the photosphere \cite{Malherbe_2022}.
This was the starting point for different studies across the pattern of granules.
In 1930 Unsold pointed out that the layers below the photosphere should be convective unstable \cite{Foukal_1990}. 
Later, Plasketts relationed these granules with the Bernard's laboratory measurements of fluids convection.
This statement are based on fluids heated from below representing hot rising gas elements convecting heat to surface \cite{Plaskett_1936}.
This hot rising gas elements are known as granulation, and each individual region—spanning approximately 700 km in size and lasting between five to ten minutes—is referred to as a granule.
The understanding of photosperic granules as convective cells leads the thought to the existence of a zone responsible of the convective motion.

\section{The solar convection Zone}
Starting at the $0.86 R_{\astrosun}$ lies the zone where the dynamics processes took place, the great temperature gradient across the layer allow the process of convection \cite{Priest_1982}.

\subsection{The convection movement in the sun}
As we mentioned, the convective movement are based on fluids heated from below representing hot rising gas granules or \textit{convective cells} convecting heat to the photosphere \cite{Plaskett_1936}. 
In this case, solar convection occurs in a highly compresible, stratified gas which leads to determine the conditions under which we expect convection and the dynamics of the granules \cite{Foukal_1990}. 

\subsection{Dynamics of solar convection}
From the core, He nuclei is built from H nuclei in the proton-proton cycle as say equation \eqref{pp chain}.
\begin{equation}
    4^{1}H \rightarrow ^4He +2e^++2\nu +26.7MeV
    \label{pp chain}
\end{equation}
Where from the H nuclei is liberated an considerable amount of high frecuency $\gamma-$rays ($26.7MeV$) and the energy of two neutrinos (0.5MeV).
However, the strong Coulumb repulsion between positevily charged nuclei increases as the product of their nuclear charges, so only lightest elements will have appreciable reaction probabilities.
As the electrons recombine with others particles the photons can be absorbed more easily. 
With this, decrease the radiative conducity and increases the temperature gradient with the opacity \cite{Foukal_1990}. 
The decrease of temperature allow a convective instability and beyond a convective turbulence.
When reaches the low photosphere, some radiation scape from the sun and the material returns to the convective stability \cite{Priest_1982}.

This inclines to stablish an onset of convection: If $T$ is increased to a value $T'$, the granule will expand rapidly to achieve a new pressure equilibrium with its surroundings.
Where is suppose the granules as vertical stratified plasma in hydrostatic equilibrium with pressure ($P(r)$); density ($\rho(r)$); and temperature ($T(r)$), the same as its surroundings at radial distance $r$ from the center of the star.

The lower-density gas experiences a bouyancy force, which will cause it to rise.
The bouyancy force will disappear when the density has dropped to the same value as the new surroundings, after an element has traveled a distance $l$.
Let be $T_n'$ the new temperature and that of its surroundings $T_n$, the adiabatic and radiative gradient for the element follows the equation \eqref{adiabatic radiative gradients}.
\begin{equation}
    T_n = T + \parens{\frac{dT}{dr}}_R l \quad \big| \quad T_n' = T' + \parens{\frac{dT}{dr}}_{ad} l
    \label{adiabatic radiative gradients}
\end{equation}
Where $ad$ references for adiabatic gradient; $R$ is for radiative gradients; and taking $l$ small.
Since we have assumed radiative equilibrium, the R denotes the gradient present in the stellar atmosphere.
Two conditions can arise from here (see equation \eqref{radiative condition}): The radiative gradient is unstable, so convection pattern is stablish and viceversa.

\begin{equation}
    -\parens{\frac{dT}{dr}}_R > \parens{\frac{dT}{dr}}_{ad}
    \label{radiative condition}
\end{equation}

If the convection pattern is stablish, the element continues to expand further and rise.
Otherwise, the element begins to contract, becomes heavier, and begins to move down to its original position.

This onset of instability when the vertical temperature gradient is too large is explained by the Schwarzchild condition.

\subsection{The Schwarzchild condition}
Taking the element described before, now suposse an elementary parcel of material displaced so slowly that remains in horizontal pressure equilibrium (see figure \ref{parcel argument}). 

\begin{figure}[H]
    \centering
    \includegraphics[width=0.6\linewidth]{Parcel argument.jpg}
    \caption{Diagram for the parcel of material displaced so slowly that remains in horizontal pressure. Image taken from \cite{Priest_1982}}
    \label{parcel argument}
\end{figure}

If the motion is adiabatic there is no heat exchange with surroundings, the rate between the density and pressure is constant.
This generates a criterion for the presence of convection known as the Schwarzchild condition \eqref{Schwarzchild cond}.
\begin{equation} 
    -\frac{dT}{dr} > \frac{\gamma - 1}{\gamma} \parens{\frac{GM_{\astrosun}m}{r^2k_B}}
    \label{Schwarzchild cond}
\end{equation}
Where $k_B$ refers to Boltzman constant; $G$ the gravitational constant; $M_{\astrosun}$ the solar mass; $m$ the mass of the granule; and $\gamma$ the degree freedom of the fluid. 
As the presence of convection reduces the temperature gradient from the higher value it would have assumed under radiative transport alone to the essentially adiabatic value \cite{Foukal_1990}.

All of the material and energy generated by convection ended up to low photosphere, where the granules shows different properties examined in after sections.

\section{The Solar Photosphere} 
Since 1874, when Langley gives a detailed description of granulation on the photosphere, the astronomers have been studied different motions and reactions on the outermost solar layer \cite{Priest_1982}. 
The distinct pattern of granules with dynamic behavior becomes apparent, where individual areas continuously emerge and disappear (see figure \ref{photosphere visual}).

\begin{figure}[H]
    \centering
    \includegraphics[width=0.6\linewidth]{Granulation pattern.jpg}
    \caption{A view of granulation on the Sun's surface. The central regions exhibit blueshifts while the edges display redshifts. Image taken from \cite{Samir_pattern}}
    \label{photosphere visual}
\end{figure}

The bright areas of granules correspond to regions where hot gas rises through the solar atmosphere. 
As this gas releases energy in the form of photons at the photosphere, it cools and subsequently descends, creating the darker regions of granules \cite{Carroll_Ostlie_2006,Dravins_1981}.

\subsection{Static photosphere: Limb darkening phenomenon}

Cause the temperature decrease from the higher layers, the photosphere intensity falls off towards the solar limb. 
Consequently, the disk intensity profile becomes more squared at increasing wavelength (see figure \ref{squared profile}).

\begin{figure}[H]
    \centering
    \includegraphics[width=0.6\linewidth]{Limb darkening.jpg}
    \caption{Squared profile for the disk intensity at increasing wavelengths. Image taken from \cite{Foukal_1990}}
    \label{squared profile}
\end{figure}

This effect is known as \textit{Limb darkening phenomenon}.
The analysis of this effect provides a direct technique for determining the photosphere temperature structure along depth. 

Furthermore, these granules has shown to be in continual motion (see figure \ref{motion granules}).

\begin{figure}[H]
    \centering
    \includegraphics[width=0.6\linewidth]{Granules motion.jpg}
    \caption{A time sequence showing granule evolution where the time intervals are about a minute. Image taken from \cite{Foukal_1990}}
    \label{motion granules}
\end{figure}

This perpetual motion across the layer generates asymmetries on the absorption lines. 

\subsection{Dynamic photosphere: The C-curved profile bisector.}
Observing the dynamic of the granules, there appear a heigth dependence of the granular velocities: The velocity of an upward moving granule decays much less rapidly than its excess brigthness. 
Changes in the granulation structure, contrast and velocity field around the spots and network have been infered indirectly from observations of Fraunhofer line profile shapes \cite{Foukal_1990}. 
The result is characteristically C-Curved profile bisector (see figure \ref{c curved profile}).

\begin{figure}[H]
    \centering
    \includegraphics[width=0.6\linewidth]{C curved profile bisector.jpg}
    \caption{The C-curved profile bisector. In the infrared shown to be less pronounced than violet range.}
    \label{c curved profile}
\end{figure}

The process of the creation in this c-curved profile bisector is divided in three stages.
First, the line profile near its mid-depth portion is formed in the most rapidly upflowing bright material (Blueshifted).
Then, the deepest portion of the line core is formed higher in the decelerated upflow (Less Blueshifted).
Finally, the line wings where the opacity is least, tend to be formed deepest in the cool (Redshifted).

This dynamic process induces perturbations in spectral absorption lines, line profile asymmetries, and line depth-dependent wavelength shifts \cite{Gray_Pugh_2012} known as the three signatures of convection.

\section{The three signature of convection}
The signatures of convection in stars are described by Gray in his research \cite{Gray_2009,Gray_Oostra_2018,Gray_Pugh_2012} about the principal characteristics in the spectra to identified and treat the asymmetries due to convective motion.

\subsection{Line broadening}
Due to the limb darkening phenomenon and atmospheric absorption, there exists two types of line cores: Strongest and weaker lines \cite{Gray_Pugh_2012}.
Where strongest refers to more curved than weaker lines as is shown in the figure \ref{line broadening}.

\begin{figure}[H]
    \centering
    \includegraphics[width=0.6\linewidth]{Granulation pattern.jpg}
    \caption{In the figure is comparised the strongest and the weaker line core shapes.}
    \label{line broadening}
\end{figure}

This effect is measurable with the second derivate in the observed wavelength, that representing the core curvature (see equation \eqref{core curvature}).

\begin{equation}
    C_c =  \lambda^2 \parens{\frac{d^2 p(\lambda)}{d \lambda^2}}
    \label{core curvature}
\end{equation}

Where $p(\lambda)$ is the polinomial fit for the core evaluated on the observed wavelength.
In the near infrared range we can denote a natural bands or groups of lines  which are separated for the teluric elements, on others words, elements like oxygen and CO2 that are absorbed by the atmosphere.
Tis range are invisible for the spectra, and it helps us to detected the natural bands in the near infrered curvature profile.

\subsection{Line profile bisector asymmetry}
As demonstrated by Nieminen \cite{Nieminen_2017}, the asymmetry occurs because the c-curved profile bisector reflects velocity variations and a bisector slope (see figure \ref{asymmetries on lines}).

\begin{figure}
    \centering
    \includegraphics[width=0.7\linewidth]{Asymmetries on typical lines.jpg}
    \caption{Asymmetries on an average absorption line. Can be observed the differences in intensity redshift profile. Image taken from \cite{Nieminen_2017}}
    \label{asymmetries on lines}
\end{figure}

According to Kirchhoff's laws, absorption line formation requires lower temperature conditions, which are found precisely in the Sun's outermost atmospheric layers \cite{Carroll_Ostlie_2006}. 
These regions not only provide the appropriate temperatures for absorption line formation but also exhibit comparatively higher opacity. 
Among all available spectral lines, those from neutral iron (Fe I) are particularly valuable for solar granulation studies due to two key characteristics: they display significant opacity and relatively low thermal broadening. 
These properties make Fe I lines excellent tracers of granulation patterns in the solar photosphere \cite{Nieminen_2017}.

These asymmetries can be measured using the third derivate relation for the line bisector (see equation \eqref{third derivate relation}).

\begin{equation}
    \frac{c}{\lambda}\parens{\frac{ \displaystyle \frac{d^3 p(\lambda)}{d \lambda^3}}{3 C_c^2}}
    \label{third derivate relation}
\end{equation}

Where $C_c$ refers to the core curvature in the equation \eqref{core curvature}.

\subsection{Line depth-dependent wavelength shifts}
Many studies across the years have detected and observed the phenomena of wavelength shifts againts the line depth, or as it is called, chromodependence in the granulation pattern.
This behavior was showed to be more present in the weaker lines which are related to infrared and violet range.

\subsection{Solar granulation pattern}
The solar granulation pattern is a plot of relative velocity against line depth, as shown in the figure \ref{granulation pattern}.

\begin{figure}
    \centering
    \includegraphics[width=0.7\linewidth]{Granulation pattern Ellwarth.jpg}
    \caption{Granulation pattern for the IAG spatially resolved quiet sun atlas, is shown to have a strong chromodependece in the weaker lines. Image taken from \cite{Ellwarth_2023}.}
    \label{granulation pattern}
\end{figure}

The importance of this pattern lies in stars resemble solar patterns, differing only by a scaling factor (see \cite{Gray_Pugh_2012}). 
Furthermore, these analysis contributes to the understanding and radiation of photospheric hydrodynamic models \cite{Dravins_1981,Gray_2009}.
However, for the measurement is necesary the understanding of the convective blueshift.

\subsubsection{Convective Blueshift}
When the Sun pushes material up through its outer layer, the spectrum exhibits a blueshift. 
As this material subsequently cools and falls back through the atmosphere, it produces a redshift, but emits less light, making the blueshift dominant.
That can be measure by the Doppler effect but as the velocities can be significant, the relativistic formulation of this effect must be applied.

The relativistic Doppler effect accounts for length contraction, as predicted by Einstein’s theory of relativity. 
This introduces an additional correction term to the classical shift, which becomes particularly relevant in high-velocity scenarios or strong gravitational fields.
However, the measured negative redshift resulting from convective motions is known as \textit{convective blueshifts} which are measured by the equation \eqref{relative velocity}. 
\begin{equation}
    v_r \approx c \parens{\frac{ \lambda_{obs}- \lambda_{rep}}{\lambda_{rep}}}
    \label{relativevelocity}
\end{equation}

Since the strength of the convective distortions and shifts as spectral lines varies across the H-R diagram, we expect the systematic errors in radial velocities \cite{Gray_2009}.

\subsection{Chromodependence characterization}
Recently, the emphasis on the measurement of granulation pattern has opened new interpretations to the line depth-dependent wavelength shifts.
However, there is still no characterization for this phenomenon.

In 2018 Gray and Oostra try to stablish a standard curve determined by the second grade polynomial fit to the solar granulation pattern.
Although, the authors just take into account the range of $4950\mathring{A}$ to $5700 \mathring{A}$ of the spectrum, avoiding the chromodependence in the weaker lines.
This motivate to search a strong characterization and give the recipe for dealing with this phenomenon.

Hamilton and Lester, give a little theory that this phenomenon is atribute to the rotation in the photosphere.
The pronounced differential rotation with latitude observed at the photosphere seems to be the result of convective flows driven radially by the bouyancy force and deflected horizontally by the coriolis force due to solar rotation \cite{Foukal_1990}.
This rotation made contributions on angular momentum in two different forms: Meridian circulation and Reynolds stresses.

\begin{figure}
    \centering
    \includegraphics[width=0.7\linewidth]{Angular momentum.jpg}
    \caption{Contributions on angular momentum. Image taken from \cite{Foukal_1990}.}
    \label{angular momentum}
\end{figure}

The first contribution occurs if axysimmetric meridional circulation are present (see figure \ref{angular momentum}). 
In the absence of any other angular momentum transport, a circulation in either sense will tend to spin up the poles and the interior because the fluid carries angular momentum. 

The second contribution tends to enforce solid body rotation, then the meridional circulation drive an equatorial aceleration.
The reason is for equal velocity in the meridional plane, the flux of angular momentum per unit mass across the dashed line will be larger toward the equator than away from it.
This mechanism depends on the existence of nonaxisymetric convective motion because net fluxes of angular momentum in latitudinal or radials directions are produced without requering a net mass flux. 
Neither buoyancy forces, which are strictly radial, nor presure gradients, which must average to zero around the solar circumference, can themselves influences the suns axysimetric rotation profile\cite{Foukal_1990}.

